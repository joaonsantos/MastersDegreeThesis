%!TEX program = xelatex
%
% ================================================================
%	Tipo de dissertação:
%		escolher entre "doutoramento" ou "mestrado"
%
%	Área científica:
%		escolher entre
%			- "ct" (ciências e tecnologia, final); "ctR" (ciências e tecnologia, rascunho);
%			- "csh" (ciências sociais e humanas, final); "cshR" (ciências sociais e humanas, rascunho);
%			- "artes" (artes, final); "artesR" (artes, rascunhos)
%
% ================================================================
%
\documentclass[mestrado,ct,12pt]{teseue}
%
%
% ================================================================
%	DOCUMENTO:
%		 
%		Língua, Título, Nome do Candidato, Curso, etc
%		Estrutura
% ================================================================
%
% ----------------------------------------------------------------
%
%	LÍNGUA DA TESE
%
%	Opções atuais:
%	- PT: Português (novo acordo ortográfico)
%	- EN: Inglês
%
\tueLINGUA{EN}
%
% ----------------------------------------------------------------
%
%	TÍTULO DA TESE
%
%	Em Português e Inglês.
%
\tueTITULO
{Identity Management in Healthcare Using Blockchain Technology}
{}
%
% ----------------------------------------------------------------
%
%	SUBTÍTULO DA TESE
%
%	Em Português e Inglês.
%
\tueSUBTITULO
{}
{}
%
% ----------------------------------------------------------------
%
%	CANDIDATO
%
%	Nome completo.
%		
\tueCANDIDATO
{João Pedro Nunes dos Santos}
%
% ----------------------------------------------------------------
%
%	TÍTULO E NOME DO/A ORIENTADOR/A
%
%	Designação oficial e nome do orientador/a.
%	Em geral, "Orientador" ou "Orientadora".
%
\tueORIENTADOR
{Orientador}
{Pedro Salgueiro}
%
% ----------------------------------------------------------------
%
%	SEGUNDO ORIENTADOR/A (se aplicável)
%
%	Designação oficial e nome do segundo orientador/a.
%	Em geral, "Co-orientador" ou "Co-orientadora".
%
\tueSEGUNDOORIENTADOR
{Co-Orientador}
{Vítor Beires Nogueira}
%
% ----------------------------------------------------------------
%
%	TERCEIRO ORIENTADOR/A (se aplicável)
%
%	Designação oficial e nome do terceiro orientador/a.
%	Em geral, "Co-orientador" ou "Co-orientadora".
%
%\tueTERCEIROORIENTADOR
%{Co-Orientador}
%{António Inácio Norberto}
%
% ----------------------------------------------------------------
%
%	CURSO
%
%	Nome do curso em que se enquadra esta tese.
%
\tueCURSO
{Engenharia Informática}
%
% ----------------------------------------------------------------
%
%	ESPECIALIDADE (se aplicável)
%
%	Nome da especialidade em que se enquadra esta tese.
%
%\tueESPECIALIDADE
%{Coordenação de Recursos Naturais}
%
% ----------------------------------------------------------------
%
%	DEPARTAMENTO
%
%	Departamento anfitrião do curso.
%
\tueDEPARTAMENTO
{Departamento de Informática}
%
% ----------------------------------------------------------------
%
%	ESCOLA
%
%	Escola a que pertence o departamento.
%
\tueESCOLA
{Escola de Ciências e Tecnologia}
%
% ----------------------------------------------------------------
%
%	PALAVRAS CHAVE
%
%
\tuePALAVRASCHAVE
{Blockchain, Saúde, Identidade, Hyperledger Fabric, Smart Contracts}
{Blockchain, Healthcare, Identity, Hyperledger Fabric, Smart Contracts}
%
% ----------------------------------------------------------------
%
%	DATA
%
%	Data de submissão da tese.
%
\tueDATA
{\today}
%
% ----------------------------------------------------------------
%
%	DEDICATÓRIA
%
\tueDEDICATORIA
{To My Family}
%
% ----------------------------------------------------------------
%
%	PREAMBULO
%
%	Comandos e definições para o LaTeX que devem estar **antes**
%	do texto do documento.
%
\tuePREAMBULOLATEX{
	\usepackage[figureright]{rotating}
}
%
% ----------------------------------------------------------------
%
%	PREAMBULO
%
%	Texto até à página 1. 
%
%	Por omissão os conteúdos estão definidos nos ficheiros
%		- prefacio.tex
%		- agradecimentos.tex
%		- acronimos.tex
%		- sumario.tex
%		- abstract.tex
%
\tuePREAMBULO {
  %!TEX root = main.tex
\chapter*{Acknowledgements}

Lorem ipsum dolor sit amet, consectetur adipiscing elit. Vivamus vitae est vitae risus varius malesuada et eget velit. Morbi tincidunt venenatis tellus, in volutpat ante varius et. Fusce congue maximus velit ac dignissim. Integer hendrerit pharetra libero, at vehicula odio vestibulum eget. Etiam eget fringilla leo, sit amet posuere nisl. Aenean at tincidunt felis. Cras rhoncus mauris libero, a vestibulum risus faucibus quis. Aenean malesuada vitae nibh ut dapibus. Pellentesque vel blandit odio.

Maecenas massa leo, egestas id augue at, aliquam iaculis leo. Etiam ac lacus tempus, malesuada dolor vel, mattis leo. Duis tortor mi, accumsan vitae ligula eu, luctus accumsan diam. Etiam venenatis elit non magna aliquam eleifend. Phasellus in nunc at arcu iaculis ultrices sed sed ante. Nullam in velit a metus convallis vestibulum a vitae turpis. Proin fringilla dui tempor, ultrices metus nec, lobortis elit. Sed at posuere augue. Phasellus ac massa fringilla, convallis urna nec, aliquet orci. Mauris placerat tellus vel scelerisque tempus. Donec lacinia tincidunt mattis. Donec congue, augue sed ullamcorper placerat, erat nunc vestibulum tellus, vel consequat sem diam in magna. Vivamus ac dolor lacinia magna pharetra maximus. Nulla congue feugiat vehicula. Praesent luctus purus ac justo tempor eleifend.
  \tableofcontents
  \listoffigures
  \listoftables
  %!TEX root = main.tex
\begin{tueACRONIMOS}
	\begin{acronym}[IEEE]
		\acro{EHR}{\emph{Electronic Health Record}}
		\acro{HL7}{\emph{Health Level 7}}
		\acro{DDOS}{\emph{Distributed Denial of Service}}
		\acro{EU}{\emph{European Union}}
		\acro{BFT}{\emph{Byzantine Fault Tolerant}}
		\acro{GDPR}{\emph{General Data Protection Regulation}}
		\acro{DLP}{\emph{Distributed Ledger Platform}}
		\acro{EVM}{\emph{Ethereum Virtual Machine}}
		\acro{SDK}{\emph{Software Development Kit}}
		\acro{MSP}{\emph{Membership Service Provider}}
		\acro{HLF}{\emph{Hyperledger Fabric}}
		\acro{CA}{\emph{Certificate Authority}}
		\acro{FHIR}{\emph{Fast Healthcare Interoperability Resources}}
		\acro{API}{\emph{Application Programming Interface}}
		\acro{JSON}{\emph{JavaScript Object Notation}}
		\acro{FHIR}{\emph{Fast Healthcare Interoperability Resources}}
		\acro{CIA}{\emph{Confidentiality, Integrity, and Availability}}
		\acro{SHA}{\emph{Secure Hash Algorithm}}
		\acro{TLS}{\emph{Transport Layer Security}}
	\end{acronym}
\end{tueACRONIMOS}

  \begin{tueABSTRACT}

  Bitcoin served as the catalyst for creating a solution to secure digital
  transactions without requiring a trusted third party to be involved. To solve
  this problem, the mechanisms now associated with a Blockchain were
  conceptualized and implemented to serve as the backbone for the Bitcoin
  network. More specifically it was used as a security tool to try and make
  Bitcoin a more transparent and reliable form of cash, a digital currency.
  Even though today, it is clear that it currently, cannot fulfill its intended
  original purpose, it nonetheless, enabled further avenues for innovation and
  creativity used to solve both new sets of problems as well as old problems in
  a new way.

  Blockchain was initially used as the backbone for various cryptocurrencies
  networks and was eventually extended to provide a platform that allows the
  execution of code in an autonomous manner exactly as it was programmed, with
  no human intervention. These smart contracts can be used to solve yet another
  set of problems due to their transparent behaviour, lack of human
  intervention and distributed nature. 

  Blockchain technology allows the creation of systems that introduce a number
  of benefits over traditional data handling used in today's Healthcare
  Information Systems. Costs and risks associated with these systems can be
  reduced and information can become transparent and trustworthy to all
  participants. In this article the technological foundations that enable this
  change are explored and analysed. The Hyperledger Fabric Network with true
  private transactions and advanced security mechanisms was used to serve as
  the basis for this system. An application was created that uses smart
  contracts to manipulate the ledger. In this paper we present this system and
  its impact in Healthcare.

\end{tueABSTRACT}

  \begin{tueSUMARIO}

  A criptomoeda Bitcoin foi essencial para criar uma solução para transacções
  digitais seguras, sem requerer a participação de um terceiro interveniente
  fidedigno para ambas as partes.  Para resolver este problema, os mecanismos
  que hoje são associados com a tecnologia Blockchain, foram concebidos e
  implementados para servir como base para a rede da Bitcoin. Mais
  especificamente, esta foi utilizada como um mecanismo de segurança, de forma
  a tornar a Bitcoin uma forma de dinheiro mais transparente e estável, uma
  moeda criptográfica. Mesmo que a Bitcoin não tenha conseguido cumprir o seu
  propósito original, a tecnologia Blockchain despoletou novas inovações e
  permitiu maior criatividade.

  A Blockchain tem sido, desde então, a base tecnológica de várias
  criptomoedas. Algumas implementações desta tecnologia permitem a execução de
  código de uma forma autónoma exactamente como foi programado, sem intervenção
  humana.  Habitualmente chamados \textit{smart contracts}, estes podem ser
  usados para resolver um novo conjunto de problemas devido ao seu
  comportamento transparente, ausência de intervenção humana e devido à sua
  natureza distribuida. 

  A Blockchain é uma tecnologia que permite a criação de sistemas que
  introduzem um conjunto de beneficios em relação aos sistemas tradicionais de
  armazenamento de dados utilizados nos serviços de saúde. Custos e riscos
  associados a estes sistemas podem ser reduzidos e a informação pode ser mais
  transparente e fidedigna para todos os participantes.

  A rede Hyperledger Fabric com transacções privadas e mecanismos avançados de
  segurança foi usada como base para a criação do sistema proposto nesta
  dissertação. Adicionalmente, uma aplicação foi criada que usa \textit{smart
  contracts} para manipular o \textit{ledger} da Blockchain.

  O trabalho apresentado nesta dissertação mostra que um sistema baseado em
  Blockchain, neste caso em Hyperledger Fabric, é adequado a gerir a identidade
  de utentes,  em organizações prestadoras de cuidados de saúde. Apesar das
  funcionalidades apresentadas por esta plataforma serem focadas em privacidade
  e segurança, algumas medidas adicionais em torno da confidencialidade dos
  dados tiveram de ser implementadas. Independentemente disso, o sistema foi
  construido com sucesso e conseguiu cumprir os requerimentos que foram
  definidos. A implementação deste sistema em serviços de saúde traria
  tranparência, imutabilidade e segurança adicional para utentes e
  profissionais de saúde.

\end{tueSUMARIO}

}
%
% ----------------------------------------------------------------
%
%	CONTEÚDO
%
%	Texto principal da tese.
%
\tueCONTEUDO  % A partir da página 1
{
	\chapter{Introduction}\label{introduction}

Health is intrinsically linked with technology as new technologies enable safer
and better treatments. It is also worth noting that computing devices and
networks are now easily available and widespread to the general population in
more develop countries. Healthcare organizations now store patients data on a
digital format. The Electronic Health Record (\textbf{ehr}) is an abstract
concept representing the patients digitally stored clinical data and their
identity in a medical context.

Electronic Health Records (\textbf{ehr}) for many years have lacked a standard,
even tough, more recently there has been some progress regarding this matter.
Standards are an important step to implement because they allow
interoperability between different organizations. The Health Level 7
(\textbf{hl7}) standard, being developed by Health Level Seven organization, is
growing in use and is represented internationally. Health Level Seven Portugal
is starting its operations and is building a community to support the use of
this standard in Portugal~\cite{HealthLevel7}.

This paper provides an overview into the design and implementation of a
Blockchain based system for managing the identity of a patient in a Healthcare
context. Such a system could be used as a complement to current Information
Systems in order to provide a higher degree of resiliency and trust. The
patient should be able to manage his data and control its access. It can be
used to handle the patient’s medical identity, for example, in hospitals or
clinics and would help solve the aforementioned problems in how data is handled
in the Information Systems available in nowadays regular medical environments.

Blockchain is often known as the technology behind the Bitcoin Cryptocurrency.
The phenomenon we know as Bitcoin depends on two complementary technologies,
digital tokens and blockchain, that together facilitate digital identity,
ownership, contracts, and trust~\cite{Evans2016}.

A Blockchain runs on a network of computers and has a list of records that are
replicated across the participating peers. Blockchain as we know today was
conceptualized as the public ledger for the Bitcoin cryptocurrency in 2008 by
Satoshi Nakamoto, a pen name of, a still unknown to this day, individual or
organization of individuals. The network was implemented in 2009 and many are
now finding it has a much broader potential across many fields, with some
implementations even resembling a programming platform to execute code in an
autonomous manner~\cite{Nakamoto2008}.

Traditional databases and architectures can become vulnerable and a target to
groups of malicious actors that possess the technical expertise to deny
services with Distributed Denial of Service (\textbf{ddos})
attacks~\footnote[1]{A Distributed Denial of Service (\textbf{ddos}) attack is
an attempt to make an online service unavailable by overwhelming it with
traffic from multiple sources.} or cause a data breach~\footnote[2]{A data
breach is the intentional or unintentional release of secure or
private/confidential information to an untrusted environment.}. 

Making matters worse other problems spring to mind. The data that forms the
identity of a patient is often fragmented across multiple Healthcare
organizations, in such a way that, to get a true overview of the patients
history and diagnosis you would need to merge together all the pieces of
information stored in data systems that are hosted in architecturally different
and sometimes competing Healthcare Information Systems. Transparency is also a
concern, as a patient does not currently possess the means to track how his
medical data is being handled by the medical services he used.

As more information becomes available, new insights can be extracted by health
professionals that lead to an overall improvement of the patients interaction
with the Healthcare ecosystem. However, maintaining this huge amount of data
secure is a costly and risky matter for every party involved. Security and
privacy should be a top concern regarding this sensitive data. 

In this article different Blockchain implementations are explored to get an
overview of their capabilities and purposes. Also, some use cases of Blockchain
based systems used in the Healthcare field are presented. More precisely, in
Section~\ref{background}, a brief introduction to Blockchain is made followed
by an introduction to its most prominent implementations. Then a number of
real-world use cases of this technology in the healthcare field are explored.
In Section~\ref{HLFHealthcare} technical details of the system will be
presented. Finally, in Section~\ref{conclusion}, some conclusions are observed
regarding the change enabled by these advances.

	\chapter{Background}\label{background}


\begin{quote} \emph{"This project aims to build a Blockchain based system to
  manage the identity of patients and investigating the suitability of creating
  such a system in the Healthcare environment according to objective criteria.
  While Blockchain is not a new concept at this point, it is an evolving
  technology that is being used to solve old problems with new approaches while
  at the same time creating new application fields and challenging old
  conventions and methodologies. This Chapter will provide an overview of this
  technology and some of its most prominent implementations. Finally, some
  context is given to how technology has been helping the Healthcare industry
  to enable better management of their patients identity and how the current
  information systems of Healthcare establishments handle this task."}
\end{quote}

\section{Blockchain Technology}

  The concept of Blockchain is abstract. It is a collection of technologies
  orchestrated to work together. In this sense the concept can be used to refer
  to the Bitcoin's Blockchain, alternative implementations or even forks of the
  Bitcoin Blockchain called Altchains~\cite{Lewis2015} that share many
  characteristics but may have different features and purposes. It can even
  refer to platforms that allow execution of code in an autonomous manner,
  exactly as it was programmed, with no human intervention.  A Blockchain is,
  generally speaking, a continuously growing list of records being written in
  the ledger, a structure where all records are written and stored, that is
  constantly being replicated across a network of peers, in opposition to
  having a single central record history, making it a good example of a
  distributed database, thus avoiding having a single central point of failure
  that can be easily targetable~\cite{Barclay2017}.

  The purpose of a Blockchain is to maintain integrity in a network of
  distributed systems~\cite{Drescher2017}. To fulfill this purpose it uses
  cryptographic techniques and digital signatures to not only verify the
  authenticity of records but also as a way to manage read or write access to
  the network and as proof that a record was written in the ledger and was
  never tampered with, creating an immutable history of records, that benefits
  various use cases as discussed later in this document.

  Unlike a conventional database system running in a server, where only a
  single entity keeps a copy of the underlying database, making it centralized
  by design, the ledger of the Blockchain is constantly replicated across any
  number of participating nodes in the network~\cite{Lewis2015} in a regular
  cadence defined at the genesis of the network. In some implementations, not
  every participant has the same ability to interact with the ledger and in
  this respect a Blockchain can be permissionless or permissioned. Generally
  speaking, in a permissionless Blockchain every node of the network can write
  in the ledger whereas in a permissioned Blockchain only a select group of
  entities have access to writing in the ledger, making the permissioned
  version, by default secure, if the entities themselves who manage the network
  are considered secure and trustworthy by the participants in the
  network~\cite{Lewis2015,Valenta2017}.

  But then, how does a permissionless Blockchain maintain security if every
  participant in the network has access to writing on it, including potentially
  malicious parties?

  Given that participating nodes in a public network can belong to different
  and often competing parties, there is no implied trust between them, so the
  Blockchain needs a mechanism to ensure the integrity of the ledger and
  prevent malicious meddling from interested parties and avoiding the need for
  a central authority~\cite{Barclay2017}.  Take for example the Bitcoin
  Blockchain that uses a peer-to-peer network to avoid the requirement of a
  third party being involved in a financial transaction such as a financial
  institution or a middle man, which must be trusted with the details of a
  transaction to see it through~\cite{Nakamoto2008}.

  Consensus is a mechanism employed by the Blockchain to solve this problem.
  Even though consensus mechanisms can behave vastly different, depending on
  its implementation and purpose, they are at the core a solution to create
  immutability and ensure resiliency by ensuring the majority of the network
  agrees upon the sequence of events.  For example, in the Bitcoin's Blockchain
  case, consensus is reached by the longest chain rule where the longest chain
  of blocks not only serves as proof of the sequence of events witnessed, but
  as proof that it came from the largest pool of computing power, as it uses a
  proof of work (\textbf{pow}) algorithm that relies on brute force to solve a
  complex mathematical puzzle, making the longest chain of blocks the one with
  the most computing power behind it and therefore agreed upon by the majority
  of the network~\cite{Baars2016,Wood2017} making it the most likely to be the
  one that represents the sequence of events witnessed.

  While the Blockchain, we now know today, was conceptualized as the public
  ledger for the Bitcoin cryptocurrency in 2008 by Satoshi Nakamoto and
  implemented in 2009, many are now using it as a foundation across many
  application areas such as traceability and asset management~\cite{MIT2016}.
  Thanks to the roaring success of Bitcoin and the increasingly apparent use
  cases that the Blockchain can provide, the public and the various industries
  interest in this technological advance is rising and it is quickly becoming a
  technological foundation in our economic and social systems~\cite{Zago2018,
  Marr2018,Long2018}.

  \subsection{Ethereum}

  Due to Bitcoin getting extensive media coverage, the average public awareness
  in cryptocurrencies is shown to be rising~\cite{BitAwareness2017}. While
  Blockchain is used as a means to increase the resiliency of the Bitcoin
  cryptocurrency from malicious parties, a token is used to represent the coin. 
  
  Just like a Dollar it has no value by itself, it has value only because we
  agree to trade goods and services in exchange for a higher amount of the
  currency under our control and we believe others will do the same
  \cite{aliessi2016}. Through the years Blockchain has evolved to be capable of
  being an independent development platform using the token as a means to
  reward those who maintain the consensus by spending electricity and
  computation power in the network. In some networks like Ethereum one can
  build upon the network to create Decentralized Applications (\textbf{Ðapps})
  that allow logic to be executed in an autonomous manner~\cite{Wood2017}. 
  
  In the same manner that the Bitcoin Blockchain can be seen as an adding
  machine, the Ethereum Blockchain can be seen as a computer able to execute
  programs designed for it~\cite{Wood2015}.

  Ethereum is an open-source platform based on the Blockchain technology that
  enables developers to build and deploy \textbf{Ðapps}. Ethereum is being
  developed by the Ethereum Foundation and was first discussed by Buterin in
  2013.  Ethereum intends to provide a Blockchain with a built-in programming
  language that is used to create \textit{Smart contracts}~\cite{Wood2017}.

  These are used to describe the logic of any system that developers can
  imagine and, when created, can be deployed to the Blockchain where they
  execute as “autonomous agents”.  Thanks to these tools it is safe to say that
  long gone are the days where building Blockchain applications required a
  complex background in coding cryptography, mathematics as well as significant
  resources~\cite{Wood2017,BlockGeeks2017}.

  The Ethereum Blockchain is a permissionless Blockchain, and thus, it must
  have a consensus mechanism to ensure the validation process of every record
  and, in turn, ensure resiliency and immutability. While other implementations
  of the Blockchain have different consensus mechanics, in Ethereum’s case, all
  participants have to reach consensus over the order of all transactions that
  have taken place. If a definitive order cannot be established then a
  double-spend~\footnote{Double-spending is a potential flaw in a digital cash
  scheme in which the same single digital token can be spent more than once.
  This is possible because a digital token consists of a digital file that can
  be duplicated or falsified. As with counterfeit money, such double-spending
  leads to inflation by creating a new amount of fraudulent currency that did
  not previously exist. This devalues the currency relative to other monetary
  units, and diminishes user trust as well as the circulation and retention of
  the currency.} might have occurred and the transaction is
  rejected~\cite{Wood2017}.

  \subsection{Fabric}

  Hyperledger Fabric (\textbf{hlf}) is part of the Hyperledger project started
  in December 2015 by the Linux Foundation, and is an open-source
  developer-focused community of communities focused on the development of
  enterprise-grade, open-source Blockchain-based solutions.  Fabric is an
  implementation of a Distributed Ledger Platform (\textbf{dlp}) under the
  Hyperledger umbrella~\cite{Cachin2016}.

  Hyperledger Fabric’s initial commit was contributed by IBM and written in the
  Go programming language.  It is a permissioned Blockchain and its main design
  goal was to surpass previous Blockchain implementation limitations, such as,
  lack of true private transactions and confidential contracts.

  These goals are achieved thanks to assigning peers in the network three
  distinct roles and by offering the ability to create channels each with its
  own private ledger.  A peer can have the role of endorser, committer or
  consenter or sometimes multiple roles.  Hyperledger Fabric is intended as a
  foundation for developing applications in a modular fashion, opting for a
  plug-and-play approach to its various components as well as its consensus
  mechanism~\cite{HyperledgerFabricDocs2017}.

  Hyperledger Fabric, as discussed, also allows the creation of smart contracts
  which can be written in Chaincode.  Given that this Blockchain's key
  operational requirement is privacy, featuring true private transactions and
  confidential contracts, it makes this technology a great asset for a business
  environment where sensitive information must be handled with care and
  disclosed on a case by case basis.  Thanks to its modular approach consensus
  protocols are no longer hard-coded and trust models can be repurposed.

  \subsection{Burrow}

  Hyperledger Burrow (\textbf{hlb}) is also part of the Hyperledger project and
  its development started in 2014 by Monax and sponsored by Intel. It is a
  permissionable smart contract machine written in Go and offers a modular
  Blockchain client with a permissioned smart contract interpreter built, in
  part, to the specification of the Ethereum Virtual Machine (\textbf{evm})
  with the client having, essentially, three main components, the consensus
  engine, the permissioned \textbf{evm} and the Remote Procedure Call
  (\textbf{rpc}) gateway~\cite{Kuhlman2017,HyperledgerBurrow2017}.

  Hyperledger Burrow has its own Consensus Engine, the Byzantine fault-tolerant
  Tendermint protocol.  The Tendermint protocol is an open-source effort that
  allows high performance in solving the consensus problem and also has a
  flexible interface for building arbitrary applications above the consensus,
  as well as, a suite of tools for deployments and their
  management~\cite{Buchman2016}.
  
  \section{Identity in Healthcare}
  
  Originally records of a patient were stored in paper, a physical format.
  Thanks to the advent of the computers more and more records are stored on a
  digital format and the Electronic Health Record (EHR) was created.
  \cite{Marquez2017}  This benefits handling of information between the patient
  and the medical professionals and medical
  institutions.\cite{ONCoordinator2017} But first we must discuss what is
  defined as identity in this specific case.
  
  Identity is a construct that depends on the context.  Identity can be defined
  as the characteristics determining who or what a person is.  In this paper we
  define identity as the set of characteristics that determine who is the
  patient in the given Healthcare ecosystem they belong to, such as, the name,
  the age, the cellphone number, the gender and the birth date of the patient.
  Electronic Health Records encapsulate this information in digital format,
  however, they are usually represented in a format according to the
  Information System they were designed to work with.
  
  To enable interoperability, standards for EHRs were created and many failed
  to bring the much needed consensus that was required for interoperability
  between different Information Systems in different institutions.
  \cite{Eichelberg2006} Health Level 7 has done much work to be recognized in
  many countries and is quickly being implemented in many countries to allow
  for joint efforts between organizations. \cite{HL7Anual2016}
  
  Even with these advances in mind, the nature of many clinics and hospitals
  Information Systems makes the management of their patients identity a very
  cumbersome, costly and risky affair to handle.  Security in a connected age,
  where internet is easily available, is lagging behind and presenting some
  problems.  There is also the question of transparent use of information by
  the organizations that store it.

	\chapter{Blockchain: Properties and Use Cases}

\begin{quote} \emph{"The Blockchain, initially used to solve problems in
  centralized financial systems, has inherent characteristics that, depending
  on the implementation, make it suitable for a variety of use cases. This
  Chapter will provide a more in-depth exploration of the Blockchain technology
  showing some of the considerations that were taken into account, after
  analyzing the alternative implementations previously mentioned in
  Chapter~\ref{background}, to choose the Hyperledger Fabric Distributed Ledger
  Platform to achieve the purpose mentioned in Chapter~\ref{introduction}.
  Finally some pratical implementations of Blockchain based solutions created
  to solve problems in the Healthcare field or problems managing entities are
  presented that provide context and show the current state of this technology
  in a production environment, providing an overview of both its shortcomings
  and successes thus far."} \end{quote}

\section{Trusting the Network}

Blockchain implementations are an emerging structure for distributed computing
systems, and provide an accurate and unchangeable history of transactions
written to a publicly available ledger or record, even when there is no trust
relationship between the parties involved~\cite{Barclay2017}.

Banks used to keep track of their financial transactions by writing on a book
called the ledger. The ledger would be written on when a new transaction
occurred, storing all details of the transaction that occurred between the bank
and other entities. Nowadays the ledger is not a book, instead being a
centralized database with the same function of recording all the transactions
that are made.

Imagine the following, Joe is on vacation and needs to borrow money from Jane,
his wife. Joe calls Jane to ask for some money and Jane tells him it will send
the money right away. Jane then proceeds to call her account manager in the
bank to transfer money to Joe. Finally Jane calls Joe to tell him the transfer
went through.  As seen on Figure~\ref{fig:centralizedvsdescentralized} Joe and
Jane need to use and trust the bank as a middle man in order to complete this
transaction. If the bank was ever to be unavailable, the database was corrupted
or if someone with  privileged access to the central database and malicious
intent was able to intercept the transactions from inside the bank then all
transactions between Joe and Jane would fail creating additional costs to all
parties involved. 

\begin{figure}[h]
  \centering
  \includegraphics[width=1\linewidth]{imgs/blockchainvscentralizedNetwork.png}
  \caption{\label{fig:centralizedvsdescentralized} A comparison between a
  Centralized Banking System and a Distributed Ledger. (Source: Finance \&
  Development, 2016)}
\end{figure}

It was for a long time necessary, to establish trust between two entities, a
middle-man with a neutral stake in the transaction. While the ledger is also at
the core of the Blockchain, this technology aims to solve the dependency placed
upon third parties using decentralization and aims to make two different
entities trust each other through constant replication and through a process
called consensus.  Consensus is a mechanism that establishes a set of rules
that define if a chain of blocks is considered valid or not. In order to
properly explain how consensus works and why it is of such importance then
Blockchain network implementations must be categorized into two distinct
categories.

\section{Permissionless and Permissioned Networks}

As mentioned in this Chapter, Blockchain was created to solve problems that were
displayed in centralized financial systems. A centralized system is one that is
governed by a hierarchical authority; examples of such being banks, credit card
company’s, etc. If you want to use a Visa card you must request access from
Visa and be approved. At any time your access to that line of credit and your
funds may be made unavailable to you and your access permanently revoked
\cite{Dreifuerst2018}. The problem is that these centralized systems are
singular in number and therefore easily targetable.

In a decentralized network, participants work together to keep the network
corruption free. A decentralized network means that no single node can make a
decision individually, instead relying on other nodes to approve that change
and permanently add it to the record list. A decentralized network is a
subset of a distributed network.


%TODO

\section{Dealing with Identity Using Blockchain}
%TODO

\section{Blockchain Applied to Healthcare}

Some companies have already started developing Blockchain applications in the
Healthcare field and established some key partnerships.

Many Blockchain-based solutions are still very early on development or
deployment.  One exception is Guardtime, that has fully deployed their system
in 2008, started cooperating in 2011 and in 2016 announced a partnership with
the Estonian Government, where a million patient records are now secured by the
strategy and, until today, still proves the resilience of the Blockchain
technology, as well as, other advances in cryptography.  Now other companies
like Verizon are becoming interested in this technology for their own
purposes~\cite{GuardTime2018,EstonianGovernmentGuardTime2016}.

Another company, Gem, is collaborating with Phillips Healthcare to explore
options in this area, and is opting to solve the interoperability problem with
an additional layer of abstraction they call GemOS.  Factom, another
Blockchain-based service, has also announced a partnership with a major US
medical services provider
HealthNautica~\cite{BlockchainCompHealth2017,FactomPartnership2017}.

The use of the Blockchain technology in the health field is expanding. Just
recently a new platform appeared, called Medichain that allows patients to
store their own data in a secure way and give anonymized access to this data to
specialists. Giving data allows for users to gain tokens that represent
value~\cite{MediChain2018}.

\section{Hyperledger Fabric}
%TODO

	\chapter{Designing and Building the System} \label{HLFHealthcare}  

The goal of this thesis is to evaluate the suitability of Blockchain technology
in managing the identity of patients in an Healthcare organization by
conceptualizing and implementing a system that fulfills this role. In order to
fulfill this goal, the development part of this thesis was primarily divided
into four steps with each step building upon the previous ones. This Chapter
provides an insight into how the desired functionality was achieved by the
system starting at the conceptualization and its associated challenges all the
way to the implementation of said system. This system was built using some of
the technologies and concepts described in the previous section.

\section{First Step - Defining Requirements and Choosing a 
	Platform}\label{choosingHyperledger}

After investigating the various Blockchain platforms some criteria was needed
to serve as reference. As such, the first step consisted in defining a set of
key points that the built system had to fulfill. Defining the requirements
proved helpful to choose the most appropriate platform for the objectives as
explained later.

\subsection{Requirement Definition}
The requirements for this project were deemed to be as follows:

\renewcommand{\labelenumi}{\Roman{enumi}.}
\begin{enumerate}
  \item The system must allow a patient to opt into the network and register as
    a participant.
  \item The system must allow a patient to record his medical data under the
    approval of an administrator.
  \item The system must keep information confidential, transparent and have
    high availability.
  \item The system must provide the patient with the ability to share his data
    with another entity participating in the network, for example sharing
    information with a doctor.
  \item The system must allow the deletion of a patient's data in some manner,
    if he wishes to do so, in order to comply with European privacy laws.
\end{enumerate}

These requirements were chosen in order to create a system that is interesting
to an organization while still respecting the patients data and their access
right to it. 

After defining the requirements it was nececessary to choose the Blockchain
platform that best fulfills these requirements.


\subsection{Choosing a Platform}\label{choosePlatform}

Blockchain platforms often have different goals even tough they normally
originate from the realization that full centralization has major drawbacks.
Ranging from open networks, such as Ethereum which anyone can join and use, to
permissioned distributed ledgers, which can be run publicly or privately but
are only open to access and participation through a membership service, such as
Hyperledger Fabric and Hyperledger Indy.

Ethereum is a popular platform on its own right and has certainly paved the way
for Blockchain to be used as a platform that can be extended and built upon. It
has a growing learning ecosystem and community. It is easy to start interacting
with the network as anyone is able to simply download a client and connect to
it.  Thanks to the Solidity smart contract language being targeted for the
specific purpose of authoring smart contracts it is a platform easy to develop
for after the initial learning barrier of the \textbf{dsl}. 

Ethereum is being used in a great deal of projects around the world proving its
stability and suitability in a wide variety of use cases. On the other hand,
handling patients medical data is a great responsibility due to the private and
personal nature of this data. Also hospitals and clinics must obey the
regulatory laws regarding privacy and usage of this data.

It is also worth noting that while Ethereum can handle private data exchange by
building upon it, as shown by Barclay, it was not designed with this intent in
mind, therefore these middle ground solutions can prove to be unwise to use at
scale given Ethereum's and the whole Blockchain's ecosystem past problems with
scalability.  

Fabric, like Ethereum, was built with the intention of being a general purpose
use Blockchain. It provides developers with the tools needed to build any
system they can imagine. However, the latter is clearly focused on making
organizations feel more at ease by being auditable as it offers an identity
service and a known environment due to using a membership service provider and
a certificate authority, therefore avoiding the same fate as \textbf{IoT}
devices where the lack of security regulations and ambiguity in how data
collected by these devices is handled has stopped these to be used in any
official capacity.

Fabric also has good amount of development tools that are now maturing and a
good learning environment with ample documentation about every aspect important
for a developer looking to get started into it. Fabric is being backed by the
Linux Foundation and IBM, lending credibility to the project and ensuring that
this platform is supported and developed in the foreseeable future, being
governed by a diverse technical steering committee and by a diverse set of
maintainers from multiple organizations. In regards to performance the
Hyperledger community is appointed a Performance and Scale working group to
improve performance as well as implementation of a benchmarking framework
called Hyperledger Caliper.

Regarding Fabric's features, it lends itself very well to fulfill the project
requirements. With Fabric's channels and private data segregation at peer level
it lends itself well to fulfill all the requirements that were laid out for
this project. Adding to this, many Blockchain based projects in the Healthcare
field are using permissioned networks due to the concerns regarding the privacy
of the patients while retaining the key benefits of Blockchain such as
immutability and decentralization.

Ultimately it was decided to use Hyperledger Fabric as the platform on which to
build the prototype project upon.


\section{Working knowledge of the Platform}

After choosing to work with Hyperledger Fabric it became necessary to
understand in further detail what are the components that form a network and
the tools to manage these components. This section discusses the main
components of a Fabric network and the tools required to create and maintain a
Fabric network. These components often interact with one another and provide
the technical infrastructure that comprises this technology. This knowledge
proved essential in building the solution proposed by this Thesis.

\subsection{Hyperledger Fabric Components}

An Hyperledger Fabric (HLF) network is defined as the technical infrastructure
that provides ledger and smart contract services to applications. Smart
contracts are used to generate transactions and interact with the ledger. The
network is comprised of several components. 

The ledger is one of these components, composed by a world state and a
Blockchain. The world state is a database that holds the current values of
ledger states. States are, by default, expressed as key-value pairs. The world
state is useful because it makes it easy for a program to get the current value
of these states, instead of having to traverse the entire transaction log. The
Blockchain holds the transaction logs that record the history of changes that
have resulted in the current world state. Transactions are collected and
recorded in an immutable sequence of blocks. Each block contains a set of
ordered transactions. There is one logical ledger in a Hyperledger Fabric
network, even tough in reality, the network maintains multiple copies of a
ledger that are synchronized through consensus. 

Another component is the set of peers participating in the network. A Peer is a
node that hosts a copy of the multiple ledgers and smart contracts. HLF opts to
allow multiple ledgers in a network to achieve different goals of a greater
purpose. This allows the creation of channels of information between trusted
parties, for example, a channel of secure and private information between the
clinical staff of an hospital and a patient as discussed on
Chapter~\ref{background}. 

Through a peer connection, applications execute chaincode that queries or
updates a ledger. Peers have at least one of the three different roles assigned
to them, as seen on Section~\ref{distributedLedgerPlatform}. Applications
always connect to peers when they need to access ledgers and smart contracts.
Every peer in the network is assigned a digital certificate by an administrator
from its owning organization. The mapping of a peer's identity in an
organization is provided through the membership service provider. 

In fact peers, applications, end users (clients), administrators, channels and
organizations must have an identity provided by the MSP in order to be able to
interact with the network. Each of these actors has a digital identity
encapsulated in an X.509 digital certificate standard. These determine the
exact permissions these have over resources and access to information in the
network. The MSP issues these certificates through the built-in CA component,
the Fabric Certificate Authority (CA). The Fabric CA is a private root CA
provider that consists in a CA server and a CA client. The MSP also supports
Certificate Revocation Lists (CRL).

\subsection{Administrating a HLF Networks}


As discussed, a HLF network must have an administrator. HLF provides the
\textit{cryptogen}, \textit{configtxgen}, \textit{configtxlator} and
\textit{peer} tools that are used to configure the network to suit different
needs and use cases.

The \textit{cryptogen} tool generates cryptographic data consuming the file
\textit{crypto-config.yaml}.  HLF uses an abstraction layer for certification
and authority called Membership Service Provider (MSP) that defines the rules
by which entities are governed and authenticated and it must be unique for
every participating entity.

The \textit{configtxgen} tool generates the genesis block for the orderer
services and the initial transactions.  This tool consumes the file
\textit{configtx.yaml} that defines configuration parameters for channels, the
genesis block and the orderer service.

The \textit{configtxlator} tool is also used to generate channel
configurations.  Finally the \textit{peer} tool is used to manage the
participating peers in the HLF network.

These tools are used to create and maintain the topology of the network and are
invoked when a change to the network is made, for example, when permissions to
certain records are changed or a new user is enrolled in the network and are
very much intertwined with the Fabric Certificate Authority (CA) discussed in
subsection .

\section{Building the System}

After considering the project goals of investigating the suitability of a
Blockchain based system to manage patients in Healthcare, the third step was to
build a prototype of a system that would represent the network, albeit on a
smaller scale. The insights gained from developing a simple working system
would enable benefits and risks of the approach to be identified, and
opportunities for further research to be laid out.

\subsection{Designing the System}

In order to build a system it is a good practice to conceptualize the
architecture first. After some consideration the concept was determined to be
as follows:

The information that defines the patients identity is a key requirement to
build a system that recognizes patients across the Healthcare environment, as
discussed in Chapter~\ref{background}. An asset can be created that represents
the concept of the patient's identity in this network.

To aid in interoperability with other systems, as seen in Figure
\ref{fig:interoperability}, the Fast Healthcare Interoperability Resources
(FHIR) standard by the Health Level 7 organization was used as basis for the
representation of a patient.  Each field of the structure that represents the
patients identity, defined in the smart contract, is linked to a field of the
\href{http://www.hl7.org/fhir/patient.html}{patient structure as presented in
the FHIR standard}.

\begin{figure}[ht] \centering
\includegraphics[width=0.7\linewidth]{imgs/interoperability.png}
\caption{\label{fig:interoperability}An Example of Interoperability with the
Blockchain Network} \end{figure}

% Falar sobre channels, organizations, peers, permissões. Falar sobre 
% precupações dúvidas, dificuldades, objectivos.

The most simple case of an interaction in an Healthcare service is the
interaction between a patient and a doctor. In HLF this situation translates to
two organizations and two peers. Each peer belongs to an organization. One
organization represents the patients while the other represents the hospital
where the doctor works. 

To establish a communication between the two participating peers a channel can
be created ensuring information exchanged between the two on the channel is
private and does not exist on the rest of the network.  If a third organization
with another peer representing another health clinic joined the network then
another channel could be created between the patient's organization and this
new organization. If the patient were to insert his data into the channel then
the clinic would be able to view it everytime they wished. 

Starting in version 1.1 of Fabric the MSP allows Attribute Based Access Control
meaning access flow to the data can depend on the value of a certain attribute
of the certificate. Also it is possible to encrypt data and insert it into the
channel and then require a key to decrypt the data. In this case the patient
could give a key to the doctor to be able to access only his data. In the
current version of Fabric, version 1.2, private data collections was
introduced, meaning that some data can be marked as private on a channel while
other data can be public.


\subsection{Creating the System}

To create an interactive system that can manage the patients identity in an
Healthcare environment an application was built that the user interacts with.
This application interfaces with smart contracts through the Hyperledger Fabric
Software Development Kit and the chaincode was built using the Hyperledger
Fabric Shim for node.js.

The identity of a patient was recorded on the ledger of the HLF network as a
structure via chaincode deployed to the network that interacts directly with
the ledger.  This structure contains the necessary fields to identify the
patient such as its name and birth date, for example, as well as some other
information necessary to manage this data. 

The application is accessed by the user and calls upon the smart contract.  The
smart contract will handle the assets part of the system.  A smart contract to
represent and manipulate identity was built and interfaces with the network to
write and read records to the appropriate ledger. The overview of the
architecture for this system is represented on Figure \ref{fig:appOverview}.
The smart contract also initializes and manages the ledger state through
transactions as well as the world state.

\begin{figure}[ht] \centering
  \includegraphics[width=1\linewidth]{imgs/hyperledgerAppOverview.png}
  \caption{\label{fig:appOverview}An Overview of the System Architecture
  (Source:
  \href{http://hyperledger-fabric.readthedocs.io/en/latest/write_first_app.html}{HLF
  Fabric Documentation})} \end{figure}

The application allows for user enrollment to create a new identity in the
network.  When a new user of the application enters the network; the function,
in the smart contract, that initializes the creation of the user and writes the
user to the ledger as a new participating identity is called. Due to the
security mechanisms this specific transaction is automatically signed by the
administrator of the network and is verified by the CA servers.

The smart contract also provides the application with several operations to
manage the identity object as seen on Figure \ref{fig:smartContractOverview}.
These operations form an Application Programming Interface (API) that return a
payload in JSON format with identity information from the network.  This API
allows a query to be made to the network that returns the patients information,
changing incorrect or outdated information or disabling the identity structure
of someone who is not participating in the network actively anymore in order
for that information to be read-only from that point on, for example, with more
available.  Depending on the operation only certain users can access the
information or manipulate the already existing one.  This system architecture
leads to a modular as well as extensible approach regarding the availability of
new operations that become available as soon as new versions of the smart
contract are deployed.  

\begin{figure}[ht] 
  \centering
  \includegraphics[width=1\linewidth]{imgs/smartContractOverview.png}
  \caption{\label{fig:smartContractOverview}Smart Contract Operations Example
  (Original:
  \href{http://hyperledger-fabric.readthedocs.io/en/latest/write_first_app.html}{HLF
  Fabric Documentation})} 
\end{figure}

	\chapter{Experiments and System Evaluation} 
\label{experiments}

\begin{quote}
\emph{This Chapter presents the experiments using the solution created for
  managing patients identity data in a Healthcare context. Then the solution is
  evaluated against a security model and goals set by this dissertation are
  evaluated.}
\end{quote}

To properly evaluate this solution a number of experiments were conducted, as
follows. First some data would be present on the channel when the user
interacts that represents his identity in a certain clinic. The patient would
query the Blockchain for his data and receive his data if everything worked
accordingly. The second experiment was making the patient share his data with
the doctor in the channel. The last experiment consisted in the patient trying
to query data of another patient that was inserted at the genesis of the
network and seeing if the data was encrypted or was easily readable. The
outcomes of these experiments can shape the development of the solution as it
could take these results into consideration and highlight possible problems.

\section{Testing the Built Solution}\label{tests}

With the network in place and the peers set up and registered the experiments
proposed have now their requirements fulfilled.

The patient used the function provided by the application to query the network
for his information. He searched for his patient number and was shown his
information successfully. This shows that the information was recorded with
success when the chaincode was deployed. The simple way to query personal
information with an assigned patient number also proved successful and shows
that this system can be used to store patient's identity data and retrieve it.

Then the patient had to share his information with the doctor. To proceed, it
was necessary to assume that the patient had given his patient number to the
doctor so that he could use the application built to query for that patient
number. The doctor queried the network for the patient's information and was
able to access it successfully. This proves that this platform allows, to very
easily share information between a patient and a doctor using a smart contract
in a simple way.

Finally the patient tried to access another patient's data. It was necessary to
assume that he was given the patient number by the respective patient. When he
queried the network for that patient's data it became clear what already had
arose suspicions in previous experiments. He was actually able to access that
data without a problem. This would be okay if the number was willingly given to
him. However if the number was obtained unwillingly it could prove a problem.
This meant that the solution currently, did not meet the requirement of the
information being confidential that was defined previously, even tough it is
transparent and has high availability since the information was spread through
multiple peers and could be on multiple channels. It became clear that some
additional data security measures was needed. After implementing data
encryption (see Section~\ref{confidentiality}), data was effectively encrypted
and could not be seen, fulfilling the original goal.

\section{Evaluation of the Built Solution}

It was determined that to evaluate the effectiveness of this system in regards
to security, a standard for these types of solution was needed. 

After careful consideration, the international standard for information
security known as the Confidentiality, Integrity, and Availability (CIA) triad
model was used and the solution was evaluated against this standard, in order
to draw further conclusions and evaluate how secure the built system is, in
regards to data security, which is a critical concern in this particular field. 

The three pillars that form this standard are the preservation of
confidentiality, information availability and ensuring information integrity.
The evaluation of the system against this model is presented over the following
Section.

\subsection{Confidentiality}

Confidentiality of the information stored in the network was considered a key
requirement when the requirements were presented. The Hyperledger Fabric was a
prime candidate for building the solution upon due to its focus on privacy and
a more enterprise approach to Blockchain development. While Hyperledger Fabric
offers many features such as channels that truly do segregate information in a
way that many equivalent platforms cannot do at the moment it is also true that
by default data will be stored in plain text. 

To solve this problem it was necessary to implement data encryption on top of
the network using chaincode. This way, even if someone was able to access the
underlying database or if someone used a tool like Hyperledger Explorer to
explore the network, all it would see is encrypted data that would require a
key to decrypt and become human readable. With these considerations in mind, it
can be said that the built system provides a confidential data storage.

\subsection{Integrity}

One of the key aspects of a Blockchain system is the immutability of data. This
means that once information is written, it cannot be changed or erased. The
transaction logs assure that the specific version of that asset is recorded
permanently in the network. In order to comply with privacy regulations some
data can become only visible as an hash but it still remains there. Therefore
the integrity of data on this Blockchain platform and solution is also
preserved.

\subsection{Availability}

Even though Fabric is a permissioned Distributed Ledger Platform and as such it
is administrated by an administrator it is also distributed and therefore
avoids having a single point of attack. By default, it is more available than a
simple informational system that is centralized. In this aspect it can be said
that, the more the network scales, the more robust it becomes and therefore
more availability it provides as information redundancy also increases.

\subsection{Review of the System Goals}

Using Hyperledger Fabric a system was built that successfully can create,
manage and disable patients data. Information can be shared in a secure manner
and interoperability eases organization into adopting this system. This system
provides benefits to the medical staff as well as the patients due to
transparency in how data is handled and secured.

However the costs of deploying this system in a production ready environment
would be higher compared to a more traditional approach. Since this system is
built upon a Permissioned Platform, machines to host the central services need
to be acquired and an administrator of the platform is necessary for the
necessary maintenance. As the network grows it would become more resilient and
additional servers could be used to expand the core availability of Blockchain
components.

There is also the question of scalability. Even though a Permissioned
Blockchain is always faster in relation to a Permissionless variant it still is
far from matching the scalability and performance of the Electronic Payment
Management System created by SIBS~\footnote{The Sociedade Interbancária de
Serviços is a company that manages all the debit card payment system in
Portugal and that operates with all banks. The company is responsible for the
Multibanco network.  The network is comprised by the store payment machines and
the automated banking machines that offer money withdrawal and payment
services, for example.  As of December 2014, the network had an average of more
than 75 million operations every month.} for banking transactions, for example.
If this system was intended for global use, then additional approaches would
need to be taken regarding this matter.

With this said, the pace of development has been relatively fast with new
releases on a quarterly basis that focus on the issues of scalability and
privacy, two important features pertaining to the system this Thesis proposed.

	\chapter{Conclusion and Future Work}\label{Conclusion}

Here will be the conclusion.

\section{Conclusion}

Lorem ipsum dolor sit amet, consectetur adipiscing elit. Vivamus vitae est
vitae risus varius malesuada et eget velit. Morbi tincidunt venenatis tellus,
in volutpat ante varius et. Fusce congue maximus velit ac dignissim. Integer
hendrerit pharetra libero, at vehicula odio vestibulum eget. Etiam eget
fringilla leo, sit amet posuere nisl. Aenean at tincidunt felis. Cras rhoncus
mauris libero, a vestibulum risus faucibus quis. Aenean malesuada vitae nibh
ut dapibus. Pellentesque vel blandit odio.

Maecenas massa leo, egestas id augue at, aliquam iaculis leo. Etiam ac lacus
tempus, malesuada dolor vel, mattis leo. Duis tortor mi, accumsan vitae
ligula eu, luctus accumsan diam. Etiam venenatis elit non magna aliquam
eleifend. Phasellus in nunc at arcu iaculis ultrices sed sed ante. Nullam in
velit a metus convallis vestibulum a vitae turpis. Proin fringilla dui
tempor, ultrices metus nec, lobortis elit. Sed at posuere augue. Phasellus ac
massa fringilla, convallis urna nec, aliquet orci. Mauris placerat tellus vel
scelerisque tempus. Donec lacinia tincidunt mattis. Donec congue, augue sed
ullamcorper placerat, erat nunc vestibulum tellus, vel consequat sem diam in
magna. Vivamus ac dolor lacinia magna pharetra maximus. Nulla congue feugiat
vehicula. Praesent luctus purus ac justo tempor eleifend.

Nunc eu ex vel ipsum ultrices molestie. In eget sodales turpis. Donec egestas
facilisis nulla id feugiat. Duis gravida lorem quis porttitor interdum. Sed
turpis leo, aliquet non metus a, vulputate volutpat ante. Donec neque metus,
volutpat quis congue non, aliquam sed nunc. Curabitur erat mauris, elementum
id rhoncus quis, condimentum eu felis. Quisque porta gravida velit a congue.
Nulla gravida suscipit pulvinar. Sed sed erat ut turpis consequat sagittis.
Sed scelerisque, massa ac tincidunt rutrum, libero dolor suscipit lorem,
interdum dignissim massa enim a purus. Aliquam porta orci non urna
sollicitudin, sed lobortis nibh ullamcorper. Aliquam erat volutpat. Phasellus
ac purus in massa aliquet ultricies non sit amet justo.

Quisque placerat lobortis risus. Vestibulum ante ipsum primis in faucibus orci
luctus et ultrices posuere cubilia Curae; Pellentesque eget odio sed lectus
sollicitudin consectetur et ornare libero. Aliquam et ullamcorper arcu. Fusce
mollis euismod purus, vitae auctor quam lobortis eu. Nunc mollis, velit eu
cursus feugiat, nunc neque pellentesque arcu, a suscipit tellus nunc quis
quam. Cras diam est, fermentum a rutrum sed, pretium eu tortor.

}
%
% ----------------------------------------------------------------
%
%	APÊNDICES
%
%	Texto complementar da tese.
%
\tueAPENDICES % Material de suporte
{
	% Incluir o código futuramente
}
%
% ----------------------------------------------------------------
%
%	BIBLIOGRAFIA
%
%
\tueBIBLIOGRAFIA{
	\let\oldaddcontentsline\addcontentsline% Store \addcontentsline
	\renewcommand{\addcontentsline}[3]{}% Make \addcontentsline a no-op
  % Adicionar "-" à quebra de links
  \def\UrlBreaks{\do\/\do-}
  %
  \bibliographystyle{alpha}
  \bibliography{bibliography.bib}
  %
  \let\addcontentsline\oldaddcontentsline% Restore \addcontentsline
}
%
% ----------------------------------------------------------------
%
%	ÍNDICE REMISSIVO
%
%\tueINDICEREMISSIVO{}
%
% ----------------------------------------------------------------
%
% ================================================================
%
%	Modo ORGANIZAÇÃO DA DISSETAÇÃO COMPLETA.
%
%	Prevê que
%		- a informação sobre título, autor, orientadores, etc está definida acima e que
%		- a obra tem a seguinte estrutura:
%
%			prefácio
%			agradecimentos
%			tabela de conteúdos
%			lista de figuras
%			lista de tabelas
%			lista de acrónimos
%			sumário
%			tradução do sumário
%			------------------------------
%			CONTEÚDO (vários capítulos)
%			APÊNDICES (vários capítulos)
%			------------------------------
%			bibliografia
%			índice remissivo
%
% ================================================================
%
\tueDOCUMENTO
%
% ================================================================
%	Modo CAPA, CONTRA-CAPA e LOMBADAS.
%
%	Prevê que a informação sobre título, autor, orientadores, etc está definida acima.
%
% ================================================================
%
%\tueCAPAS
