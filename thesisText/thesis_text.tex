%!TEX program = xelatex
%
% ================================================================
%	Tipo de dissertação:
%		escolher entre "doutoramento" ou "mestrado"
%
%	Área científica:
%		escolher entre
%			- "ct" (ciências e tecnologia, final); "ctR" (ciências e tecnologia, rascunho);
%			- "csh" (ciências sociais e humanas, final); "cshR" (ciências sociais e humanas, rascunho);
%			- "artes" (artes, final); "artesR" (artes, rascunhos)
%
% ================================================================
%
\documentclass[mestrado,ctR,12pt]{teseue}
%
%
% ================================================================
%	DOCUMENTO:
%		 
%		Língua, Título, Nome do Candidato, Curso, etc
%		Estrutura
% ================================================================
%
% ----------------------------------------------------------------
%
%	LÍNGUA DA TESE
%
%	Opções atuais:
%	- PT: Português (novo acordo ortográfico)
%	- EN: Inglês
%
\tueLINGUA{EN}
%
% ----------------------------------------------------------------
%
%	TÍTULO DA TESE
%
%	Em Português e Inglês.
%
\tueTITULO
{Identity Management in Healthcare Using Blockchain Technology}
{}
%
% ----------------------------------------------------------------
%
%	SUBTÍTULO DA TESE
%
%	Em Português e Inglês.
%
\tueSUBTITULO
{}
{}
%
% ----------------------------------------------------------------
%
%	CANDIDATO
%
%	Nome completo.
%		
\tueCANDIDATO
{João Pedro Nunes dos Santos}
%
% ----------------------------------------------------------------
%
%	TÍTULO E NOME DO/A ORIENTADOR/A
%
%	Designação oficial e nome do orientador/a.
%	Em geral, "Orientador" ou "Orientadora".
%
\tueORIENTADOR
{Orientador}
{Pedro ... Salgueiro}
%
% ----------------------------------------------------------------
%
%	SEGUNDO ORIENTADOR/A (se aplicável)
%
%	Designação oficial e nome do segundo orientador/a.
%	Em geral, "Co-orientador" ou "Co-orientadora".
%
\tueSEGUNDOORIENTADOR
{Orientadora}
{Vítor ... Nogueira}
%
% ----------------------------------------------------------------
%
%	TERCEIRO ORIENTADOR/A (se aplicável)
%
%	Designação oficial e nome do terceiro orientador/a.
%	Em geral, "Co-orientador" ou "Co-orientadora".
%
%\tueTERCEIROORIENTADOR
%{Co-Orientador}
%{António Inácio Norberto}
%
% ----------------------------------------------------------------
%
%	CURSO
%
%	Nome do curso em que se enquadra esta tese.
%
\tueCURSO
{Engenharia Informática}
%
% ----------------------------------------------------------------
%
%	ESPECIALIDADE (se aplicável)
%
%	Nome da especialidade em que se enquadra esta tese.
%
%\tueESPECIALIDADE
%{Coordenação de Recursos Naturais}
%
% ----------------------------------------------------------------
%
%	DEPARTAMENTO
%
%	Departamento anfitrião do curso.
%
\tueDEPARTAMENTO
{Departamento de Informática}
%
% ----------------------------------------------------------------
%
%	ESCOLA
%
%	Escola a que pertence o departamento.
%
\tueESCOLA
{Escola de Ciências e Tecnologia}
%
% ----------------------------------------------------------------
%
%	PALAVRAS CHAVE
%
%	Data de submissão da tese.
%
\tuePALAVRASCHAVE
{Blockchain, Health, Identity, Big Data}
{Blockchain, Saúde, Identidade, Big Data}
%
% ----------------------------------------------------------------
%
%	DATA
%
%	Data de submissão da tese.
%
\tueDATA
{\today}
%
% ----------------------------------------------------------------
%
%	DEDICATÓRIA
%
\tueDEDICATORIA
{To My Family}
%
% ----------------------------------------------------------------
%
%	PREAMBULO
%
%	Comandos e definições para o LaTeX que devem estar **antes**
%	do texto do documento.
%
\tuePREAMBULOLATEX{
	\usepackage[figureright]{rotating}
}
%
% ----------------------------------------------------------------
%
%	PREAMBULO
%
%	Texto até à página 1. 
%
%	Por omissão os conteúdos estão definidos nos ficheiros
%		- prefacio.tex
%		- agradecimentos.tex
%		- acronimos.tex
%		- sumario.tex
%		- abstract.tex
%
\tuePREAMBULO {
  \chapter*{Acknowledgements}

Firstly, I want to thank my dissertation advisors, Pedro Salgueiro and Vítor
Beires Nogueira, for being patient with me, for their availability and for
their dedication in this project.

I want to thank my family who helped me finish this project, with their
unending support, words of wisdom and for always pushing me to do better.

I also want to thank my work colleagues, who always supported me, helped me
grow as a professional and person, challenged me to improve, and whom I
consider as a second family.

I need to thank my close friends for always believing in me and for always
being available when I needed the most, showing they are true friends.

Lastly, every person who supported me in some manner, I truly am grateful for
your support.

  %!TEX root = main.tex
\begin{tueACRONIMOS}
	\begin{acronym}[IEEE]
		\acro{EHR}{\emph{Electronic Health Record}}
		\acro{HL7}{\emph{Health Level 7}}
		\acro{DDOS}{\emph{Distributed Denial of Service}}
		\acro{BFT}{\emph{Byzantine Fault Tolerant}}
		\acro{GDPR}{\emph{General Data Protection Rule}}
		\acro{DLP}{\emph{Distributed Ledger Platform}}
		\acro{EVM}{\emph{Ethereum Virtual Machine}}
		\acro{SDK}{\emph{Software Development Kit}}
		\acro{MSP}{\emph{Membership Service Provider}}
		\acro{HLF}{\emph{Hyperledger Fabric}}
		\acro{CA}{\emph{Certificate Authority}}
		\acro{FHIR}{\emph{Fast Healthcare Interoperability Resources}}
		\acro{API}{\emph{Application Programming Interface}}
		\acro{JSON}{\emph{JavaScript Object Notation}}
		\acro{FHIR}{\emph{Fast Healthcare Interoperability Resources}}
		\acro{CIA}{\emph{Confidentiality, Integrity, and Availability}}
		\acro{SHA}{\emph{Secure Hash Algorithm}}
		\acro{TLS}{\emph{Transport Layer Security}}
	\end{acronym}
\end{tueACRONIMOS}

  %!TEX root = main.tex
\begin{tueSUMARIO}

  Lorem ipsum dolor sit amet, consectetur adipiscing elit. Vivamus vitae est
  vitae risus varius malesuada et eget velit. Morbi tincidunt venenatis tellus,
  in volutpat ante varius et. Fusce congue maximus velit ac dignissim. Integer
  hendrerit pharetra libero, at vehicula odio vestibulum eget. Etiam eget
  fringilla leo, sit amet posuere nisl. Aenean at tincidunt felis. Cras rhoncus
  mauris libero, a vestibulum risus faucibus quis. Aenean malesuada vitae nibh
  ut dapibus. Pellentesque vel blandit odio.

  Maecenas massa leo, egestas id augue at, aliquam iaculis leo. Etiam ac lacus
  tempus, malesuada dolor vel, mattis leo. Duis tortor mi, accumsan vitae
  ligula eu, luctus accumsan diam. Etiam venenatis elit non magna aliquam
  eleifend. Phasellus in nunc at arcu iaculis ultrices sed sed ante. Nullam in
  velit a metus convallis vestibulum a vitae turpis. Proin fringilla dui
  tempor, ultrices metus nec, lobortis elit. Sed at posuere augue. Phasellus ac
  massa fringilla, convallis urna nec, aliquet orci. Mauris placerat tellus vel
  scelerisque tempus. Donec lacinia tincidunt mattis. Donec congue, augue sed
  ullamcorper placerat, erat nunc vestibulum tellus, vel consequat sem diam in
  magna. Vivamus ac dolor lacinia magna pharetra maximus. Nulla congue feugiat
  vehicula. Praesent luctus purus ac justo tempor eleifend.

  Nunc eu ex vel ipsum ultrices molestie. In eget sodales turpis. Donec egestas
  facilisis nulla id feugiat. Duis gravida lorem quis porttitor interdum. Sed
  turpis leo, aliquet non metus a, vulputate volutpat ante. Donec neque metus,
  volutpat quis congue non, aliquam sed nunc. Curabitur erat mauris, elementum
  id rhoncus quis, condimentum eu felis. Quisque porta gravida velit a congue.
  Nulla gravida suscipit pulvinar. Sed sed erat ut turpis consequat sagittis.
  Sed scelerisque, massa ac tincidunt rutrum, libero dolor suscipit lorem,
  interdum dignissim massa enim a purus. Aliquam porta orci non urna
  sollicitudin, sed lobortis nibh ullamcorper. Aliquam erat volutpat. Phasellus
  ac purus in massa aliquet ultricies non sit amet justo.

  Quisque placerat lobortis risus. Vestibulum ante ipsum primis in faucibus orci
  luctus et ultrices posuere cubilia Curae; Pellentesque eget odio sed lectus
  sollicitudin consectetur et ornare libero. Aliquam et ullamcorper arcu. Fusce
  mollis euismod purus, vitae auctor quam lobortis eu. Nunc mollis, velit eu
  cursus feugiat, nunc neque pellentesque arcu, a suscipit tellus nunc quis
  quam. Cras diam est, fermentum a rutrum sed, pretium eu tortor.

\end{tueSUMARIO}

  \begin{tueABSTRACT}

  Bitcoin served as the catalyst for creating a solution to secure digital
  transactions without requiring a trusted third party to be involved. To solve
  this problem, the mechanisms now associated with a Blockchain were
  conceptualized and implemented to serve as the backbone for the Bitcoin
  network. More specifically, it was used as a security tool making Bitcoin a
  more transparent and reliable form of cash, a digital cryptographic currency.
  Even tough Bitcoin ended up not fulfilling its intended purpose as a
  currency, the Blockchain technology has enabled further avenues for
  innovation and creativity.

  Blockchain has since been used as the backbone for various cryptocurrencies
  networks. Some implementations of this technology allow the execution of
  code, also known as "smart contracts". Smart contracts are executed in an
  autonomous manner, with no human intervention. These can be used to solve a
  new set of problems due to their transparent behavior, lack of human
  intervention and distributed nature. 

  Blockchain technology allows the creation of systems that introduce a number
  of benefits over traditional data handling used in today's Healthcare
  Information Systems. Costs and risks associated with these systems can be
  reduced and information can become transparent and trustworthy to all
  participants.
  
  The Hyperledger Fabric Network with true private transactions and advanced
  security mechanisms was used to serve as the basis for the system proposed in
  this dissertation. Moreover, a client application was also created that
  interacts with smart contracts to manipulate the ledger.
  
  The work discussed in this dissertation shows that a Blockchain system based
  on Hyperledger Fabric is suitable for managing patients identity, in
  Healthcare. Even tough the feature set of this Blockchain is very focused in
  privacy and security, some additional measures regarding confidentiality of
  data had to be implemented.  Regardless, a system was built successfully that
  met the requirements. The implementation of this system would provide
  transparency, immutability and additional security for patients and medical
  staff alike. 

\end{tueABSTRACT}

}
%
% ----------------------------------------------------------------
%
%	CONTEÚDO
%
%	Texto principal da tese.
%
\tueCONTEUDO  % A partir da página 1
{
	%!TEX root = main.tex
\chapter{Introdução}

Health is becoming more digital thanks to the widespread availability of
computing devices.  More and more medical records are stored on a digital
format.  For storing patient clinical data and their identity in a medical
context, the Electronic Health Record (EHR) was created.
 
While all this information should benefit both patient and health professionals
alike, it is not being handled in an effective manner due to problems caused,
in part, due to the fragmentation of the patients identity that naturally
occurs in today's Health Information Systems.

Health is an important topic, for everyone. Healthcare should strive to provide
the best service it can for everyone and everyone should have access to a
quality service. EHR are being generated at an ever increasing rate but most of
the data is not used in a way that puts the patient's privacy and trust at the
forefront.

The purpose of the work presented in this paper is to create and implement a
Blockchain based system for Identity Management in the Healthcare domain. The
patient will be able to manage his data and control its access. Such a system
would be suited to handle the patient’s identity, for example, in hospitals or
clinics and would be able to solve many problems in how data is traditionally
handled in the Information Systems (IS) available in a regular medical
environment.

Blockchain is known as the technology behind the Bitcoin Cryptocurrency,
altough nowadays it is being used for many more purposes that are explored in
the following sections, and its main design goal is to provide security and
immutability to an agreed upon list of records.

A Blockchain runs on a network of computers and the list of records is
replicated in some manner depending on the Blockchain implementation. The first
Blockchain was conceptualized as the public ledger for the Bitcoin
cryptocurrency in 2008 by Satoshi Nakamoto, a pen name of, a still unknown to
this day, individual or organization of individuals.  The network was
implemented in 2009 and many are now finding it has a much broader potential
across many fields, with some implementations even resembling a programming
platform to execute code in an autonomous manner.  \cite{Nakamoto2008}

A single universal way to identify a person in a given environment is clearly
something we should strive towards as seen in, for example, the \textit{Cartão
do Cidadão}, a portuguese identification document that replaces four other
identification documents, streamlining portuguese civilian identification.
This also allows many businesses to tailor their services to this document
making it easier on both parts and eliminating unnecessary costs and risks.

Electronic Health Records (EHR) have seen some progress made regarding the
standards that allow for interoperability between different organizations
thanks to the Health Level 7 (HL7) standard.  While this standard is growing in
use and is represented internationally, Portugal has just started the work
required to implement it.  \cite{HealthLevel7}

In an effort to make the identity of a patient more secure and transparent a
Blockchain can be used to create a system that puts at the forefront of its
design the patients, breaking conventions in traditional patient data handling.

In this article different Blockchain implementations are explored and related
work in this field is presented.  More precisely, in Section~\ref{background},
a brief introduction to Blockchain is made followed by an introduction to its
most prominent implementations. Then a number of real-world use cases of this
technology in the healthcare field are explored. In Section~\ref{HLFHealthcare}
technical details of the system will be presented.  Finally, in
Section~\ref{conclusion},  some conclusions are observed regarding the change
enabled by these advances.

\section{Motivação}

Suspendisse ac dui et urna faucibus consequat. Integer porta vulputate lorem quis condimentum. Vestibulum ut tristique elit. Lorem ipsum dolor sit amet, consectetur adipiscing elit. In in orci id dolor tristique laoreet ut at mauris. Suspendisse mollis leo nulla, nec vehicula mi ultricies sit amet. Nullam nulla purus, blandit et dui a, rhoncus vehicula diam. Aenean sagittis lorem in nunc cursus luctus. Aliquam condimentum ipsum volutpat purus mollis tincidunt. Donec non erat orci. Donec tincidunt sit amet libero ac vulputate. Pellentesque vitae ex vitae odio sollicitudin vestibulum. Ut hendrerit placerat sagittis. Phasellus posuere a felis id feugiat. Maecenas pulvinar, sapien sit amet maximus facilisis, metus arcu pretium nisi, ut consectetur eros nisl eget magna.

\begin{figure}
	\begin{center}
		\begin{tikzpicture}
			%
			\draw[step = 0.5, color = uegray!50!white] (0,0) grid (3.0,3.0); 
			\draw[->,thick] (0,0) -- (3.25,0);
			\draw[->,thick] (0,0) -- (0,3.25);
			%
			\coordinate (A) at (0.0,2.0);
			\coordinate (B) at (0.5,1.5);
			\coordinate (C) at (1.0,0.0);
			\coordinate (D) at (1.5,1.0);
			\coordinate (E) at (2.0,2.0);
			\coordinate (F) at (2.5,2.5);
			%
			\node at (A) {$\circ$};
			\node at (B) {$\circ$};
			\node at (C) {$\circ$};
			\node at (D) {$\circ$};
			\node at (E) {$\circ$};
			\node at (F) {$\circ$};
			%
			\draw[very thick, color = uered] (A) -- (B) -- (C) -- (D) -- (E) -- (F);
			%
		\end{tikzpicture}
	\end{center}
	\caption{Exemplo de uma figura.}
\end{figure}

Quisque fringilla dictum \index{tellus} sed faucibus. Vestibulum eget augue pellentesque, rutrum turpis ac, efficitur nunc. Integer molestie, erat at vehicula commodo, orci nunc consequat diam, fermentum finibus dui ante sit amet eros. Donec venenatis, erat sit amet vestibulum rutrum, neque metus ultrices enim, vel aliquam tellus tortor ut magna. Nullam vitae dolor ipsum. Aliquam erat volutpat. Nulla sagittis elit vel felis tempor feugiat. In ultrices mattis risus, in \index{ornare} lectus faucibus et. In hac habitasse platea dictumst. Praesent pretium aliquam tincidunt. Cras pellentesque lectus ipsum, id fermentum nulla lobortis eu. Fusce tristique dui a \index{diam} semper, vitae dapibus urna ullamcorper. Sed tincidunt elit cursus imperdiet ultricies.

Etiam eu leo diam. Nam \index{imperdiet} neque maximus, pharetra mauris nec, euismod odio. Pellentesque pretium vehicula elit at tincidunt. Nam posuere suscipit dapibus. Nulla aliquam venenatis mi non porta. Nam euismod quam id faucibus sodales. Aenean scelerisque mi et est condimentum eleifend. Duis finibus, lectus quis iaculis lobortis, mauris augue finibus justo, ut semper sem ligula nec risus. Pellentesque vel fermentum felis. Integer vehicula tristique enim id porta. Proin imperdiet tristique libero, a ultricies sem imperdiet a. Ut iaculis facilisis libero sed dictum. Nulla facilisi. Quisque volutpat neque sit amet arcu condimentum, ut efficitur urna consectetur. Aenean ultricies ante sapien, quis pharetra erat elementum faucibus. Nulla lobortis tristique dolor at posuere.

\subsection{Oportunidades}

Curabitur \index{auctor} ante sit amet velit placerat, et vulputate arcu cursus. Curabitur eu molestie tellus, ac euismod felis. Morbi scelerisque justo non eleifend scelerisque. Integer pellentesque massa ac tortor porta, ac commodo lorem aliquet. Donec a metus a libero accumsan ultricies. Integer semper elementum lorem vel \index{blandit}. Vivamus tristique turpis lectus, id rutrum eros interdum eu. Proin facilisis sodales eros eu porttitor. Nunc tristique vehicula risus, eu dapibus lacus suscipit ut. Nunc nec purus at orci aliquet lacinia.

\begin{table}
	\caption{Exemplo de uma tabela.}
	\begin{center}
		\begin{tabular}{lcc|lcc}
			\textbf{Astro} & \textbf{Dia} & \textbf{Ano} & \textbf{Astro} & \textbf{Dia} & \textbf{Ano}\\
			\hline
			Sol & -- & -- & Mercúrio  & 0.5 & 20 \\
			Vénus & 0.75 & 30 & Terra & 1 & 37 \\
			Marte & 1.5 & 45 & Lua & 3 & 3 
		\end{tabular}
	\end{center}
\end{table}

Ut rhoncus tellus nec aliquam iaculis. Aenean consectetur diam id nunc facilisis porta. Duis euismod est id risus feugiat, eu aliquam nisl imperdiet. Donec id enim feugiat, consectetur nisl in, iaculis felis. Ut et mattis elit. Mauris hendrerit, velit sit amet tristique bibendum, est nisi pretium felis, id maximus eros lorem eu sapien. Aliquam sollicitudin eros magna, et congue orci aliquam sed. 

	%!TEX root = main.tex
\chapter{State of the Art}
\section{Background} \label{background}

While Blockchain is not a new concept at this point, it is an evolving
technology that is being used to solve old problems with new approaches. This
section will explore the Blockchain technology origins and history, some of its
different implementations and a brief history to the identity problem is
presented.

\subsection{Blockchain Technology}

A Blockchain can be many things. It can refer to the Bitcoin Blockchain,
alternative implementations or forks of the Bitcoin Blockchain called Altchains
or even platforms that allow execution of code in an autonomous manner, exactly
as it was programmed, with no human intervention.  It is a continuously growing
list of records, written in the ledger, a structure where records are written,
that is being replicated across a network of devices in opposition to having a
single central record history, making it a good example of a distributed
database.  \cite{Wood2017}

The main design goal of the Blockchain is security and to fulfill this purpose
it uses techniques such as cryptography and digital signatures to not only
verify the authenticity of records but also read or write access to the
network.

Unlike a conventional central data storage, where only a single entity keeps a
copy of the underlying database, the ledger of the Blockchain is replicated
across any number of nodes.  Not every participant has the same ability to
interact with the ledger and in this respect a Blockchain can be permissionless
or permissioned. In a permissionless Blockchain every node of the network can
write in the Blockchain whereas in a permissioned Blockchain only a select
group of entities have access to writing in the ledger, making the permissioned
version, by default, secure if the entities themselves are secure and
considered trustworthy.

How does a permissionless Blockchain maintain security if every participant has
access to writing on it, including potentially malicious parties?

Take for example the Bitcoin Blockchain that uses a peer-to-peer network to
avoid meddling from a financial institution or a third party in a financial
transaction. Given that participating nodes in the network can belong to
different and often competing parties, there is no implied trust between them,
so the Blockchain needs a mechanism to ensure the integrity of the ledger and
prevent malicious meddling from interested parties or to avoid a central
authority.\cite{Barclay2017}

To solve this problem, consensus mechanisms are used differently, depending on
its implementation, but having, at its core, a solution to create immutable
records and ensure security.  In Bitcoin Blockchain’s case, consensus is
reached by the longest chain rule where the longest chain not only serves as
proof of the sequence of events witnessed, but as proof that it came from the
largest pool of computing power.\cite{Baars2016}

While the first Blockchain was conceptualized as the public ledger for the
Bitcoin cryptocurrency in 2008 by Satoshi Nakamoto and implemented in 2009,
many are now using it as a foundation across many application areas such as
identity management, traceability and asset management.  Thanks to the roaring
success of Bitcoin and the increasingly apparent use cases that the Blockchain
can provide, the public awareness of it is rising and it is quickly becoming a
technological foundation in our economic and social systems.
% Need References for this

\subsubsection{Ethereum}

Bitcoin is getting media coverage almost everyday and public awareness in
cryptocurrencies in general is rising.  Some people are considering
cryptocurrencies and the Blockchain, to be essentially the same technology and,
while that may have been somewhat true not so long ago, Blockchain technology
is starting to be used in a plethora of ways.

Ethereum is an open-source platform based on the Blockchain technology that
enables developers to build and deploy Decentralized Applications
(\textit{DAPPs}).  Ethereum is being developed by the Ethereum Foundation and
was first discussed by Buterin in 2013.  Ethereum intends to provide a
Blockchain with a built-in programming language that is used to create
\textit{Smart contracts}.  \cite{Wood2017}

These contracts are used to describe the logic of any system that developers
can imagine and, when created, can then be deployed to the Blockchain where
they execute as “autonomous agents”.  Thanks to these tools it is safe to say
that long gone are the days where building Blockchain applications required a
complex background in coding cryptography, mathematics as well as significant
resources.\cite{Wood2017,BlockGeeks2017}

Ethereum Blockchain is a permissionless Blockchain, and thus, it must have a
consensus mechanism to ensure the validation process of every record and, in
turn, ensure security and immutability. While other implementations of the
Blockchain have different consensus mechanics, in Ethereum’s case, all
participants have to reach consensus over the order of all transactions that
have taken place. If a definitive order cannot be established then a
double-spend might have occurred.

\subsubsection{Fabric}

Hyperledger Fabric (HLF) is part of the
\href{http://www.hyperledger.org/projects/fabric}{Hyperledger} project started
in December 2015 by the Linux Foundation, and is an open-source
developer-focused community of communities focused on the development of
enterprise-grade, open-source Blockchain-based solutions.  Fabric is an
implementation of a Distributed Ledger Platform (DLP) under the Hyperledger
umbrella.  \cite{Cachin2016}

HLF’s initial commit was contributed by IBM and written in Go language.  It is
a permissioned Blockchain and its main design goal was to surpass previous
Blockchain implementation limitations, such as, lack of true private
transactions and confidential contracts.

This is achieved thanks to assigning peers in the network three distinct roles
and by offering the ability to create channels each with its own private
ledger.  A peer can have the role of endorser, committer or consenter or
sometimes multiple roles.  HLF is intended as a foundation for developing
applications in a modular fashion, opting for a plug-and-play approach to
various components. \cite{HyperledgerFabricDocs2017}

HLF, as discussed, also allows the creation of smart contracts which can be
written in Chaincode.  As this Blockchain's key operational requirement is
privacy, true private transactions and confidential contracts can exist and are
a great asset for a business environment where sensitive information is
necessary and disclosed often.  Thanks to its modular approach consensus
protocols are no longer hard-coded and trust models can be repurposed.

\subsubsection{Burrow}

Hyperledger Burrow (HLB) is also part of the Hyperledger project and its
development started in 2014 by Monax and sponsored by Intel. It is a
permissionable smart contract machine written in Go and offers a modular
Blockchain client with a permissioned smart contract interpreter built, in
part, to the specification of the Ethereum Virtual Machine (EVM) and the client
has, essentially, three main components, the consensus engine, the permissioned
EVM and the Remote Procedure Call (RPC) gateway.
\cite{Kuhlman2017,HyperledgerBurrow2017}

HLB has its own Consensus Engine, the Byzantine fault-tolerant Tendermint
protocol.  The Tendermint protocol is an open-source effort that allows high
performance in solving the consensus problem and also has a flexible interface
for building arbitrary applications above the consensus, as well as, a suite of
tools for deployments and their management. \cite{Buchman2016}
%
%#===========================Identity in
%Healthcare===================================#%

\subsection{Identity in Healthcare} Originally records of a patient were stored
in a physical format.  Thanks to the advent of the computers more and more
records are stored on a digital format and the Electronic Health Record (EHR)
was created.  This benefits handling of information between the patient and the
medical professionals and medical institutions. But first we must discuss what
is defined as identity in this specific case.

Identity is a construct that depends on the context.  Identity can be defined
as the characteristics determining who or what a person is.  In this paper we
define identity as the set of characteristics that determine who is the patient
in the given Healthcare ecosystem they belong to, such as, the name, the age,
the cellphone number, the gender and the birth date of the patient.  Electronic
Health Records encapsulate this information in digital format, however, they
are usually represented in a format according to the Information System they
were designed to work with.

To enable interoperability, standards for EHRs were created and many failed to
bring the much needed consensus that was required for interoperability between
different Information Systems in different institutions.  Health Level 7 has
done much work to be recognized in many countries and is quickly being
implemented in many countries to allow for joint efforts between organizations.

Even with these advances in mind, the nature of many clinics and hospitals
Information Systems makes the management of their patients identity a very
cumbersome, costly and risky affair to handle.  Security in a connected age,
where internet is easily available, is lagging behind and presenting some
problems.  There is also the question of transparent use of information by the
organizations that store it.
%
%#===========================Related Work===================================#%

\subsection{Blockchain for Identity Management in Healthcare: Use Cases} Some
companies have already started developing Blockchain applications in the
Healthcare field and established some key partnerships.

Many Blockchain-based solutions are still very early on development or
deployment.  One exception is Guardtime, that has fully deployed their system
in 2008, started cooperating in 2011 and in 2016 announced a partnership with
the Estonian Government, where a million patient records are now secured by the
strategy and, until today, still proves the resilience of the Blockchain
technology, as well as, other advances in cryptography.  Now other companies
like Verizon are becoming interested in this technology for their own purposes.
\cite{GuardTime2018,EstonianGovernmentGuardTime2016}

Another company, Gem, is collaborating with Phillips Healthcare to explore
options in this area, and is opting to solve the interoperability problem with
an additional layer of abstraction they call GemOS.  Factom, another
Blockchain-based service, has also announced a partnership with a major US
medical services provider
HealthNautica.\cite{BlockchainCompHealth2017,FactomPartnership2017}

The use of the Blockchain technology in the health field is expanding. Just
recently a new platform appeared, called Medichain that allows patients to
store their own data in a secure way and give anonymized access to this data to
specialists. Giving data allows for users to gain tokens that represent value.
\cite{MediChain2018}

	%!TEX root = main.tex
\chapter{Resolução}

Lorem ipsum dolor sit amet, consectetur adipiscing elit. Vivamus vitae est vitae risus varius malesuada et eget velit. Morbi tincidunt venenatis tellus, in volutpat ante varius et. Fusce congue maximus velit ac dignissim. Integer hendrerit pharetra libero, at vehicula odio vestibulum eget. Etiam eget fringilla leo, sit amet posuere nisl. Aenean at tincidunt felis. Cras rhoncus mauris libero, a vestibulum risus faucibus quis. Aenean malesuada vitae nibh ut dapibus. Pellentesque vel blandit odio.

Maecenas massa leo, egestas id augue at, aliquam iaculis leo. Etiam ac lacus tempus, malesuada dolor vel, mattis leo. Duis tortor mi, accumsan vitae ligula eu, luctus accumsan diam. Etiam venenatis elit non magna aliquam eleifend. Phasellus in nunc at arcu iaculis ultrices sed sed ante. Nullam in velit a metus convallis vestibulum a vitae turpis. Proin fringilla dui tempor, ultrices metus nec, lobortis elit. Sed at posuere augue. Phasellus ac massa fringilla, convallis urna nec, aliquet orci. Mauris placerat tellus vel scelerisque tempus. Donec lacinia tincidunt mattis. Donec congue, augue sed ullamcorper placerat, erat nunc vestibulum tellus, vel consequat sem diam in magna. Vivamus ac dolor lacinia magna pharetra maximus. Nulla congue feugiat vehicula. Praesent luctus purus ac justo tempor eleifend.

Nunc eu ex vel ipsum ultrices molestie. In eget sodales turpis. Donec egestas \index{facilisis} nulla id feugiat. Duis gravida lorem quis porttitor interdum. Sed turpis leo, aliquet non metus a, vulputate volutpat ante. Donec neque metus, volutpat quis congue non, aliquam sed nunc. Curabitur erat mauris, elementum id rhoncus quis, condimentum eu felis. Quisque porta gravida velit a congue. Nulla gravida suscipit pulvinar. Sed sed erat ut turpis consequat sagittis. Sed scelerisque, massa ac tincidunt rutrum, libero dolor suscipit lorem, interdum dignissim massa enim a purus. Aliquam porta orci non urna sollicitudin, sed lobortis nibh ullamcorper. Aliquam erat \index{volutpat}. Phasellus ac purus in massa aliquet ultricies non sit amet justo.

Quisque placerat lobortis risus. Vestibulum ante ipsum primis in faucibus orci luctus et ultrices posuere cubilia Curae; Pellentesque eget odio sed \index{lectus} sollicitudin consectetur et ornare libero. Aliquam et ullamcorper arcu. Fusce mollis euismod purus, vitae auctor quam lobortis eu. Nunc mollis, velit eu cursus feugiat, nunc neque pellentesque arcu, a suscipit tellus nunc quis \index{quam}. Cras diam est, fermentum a rutrum sed, pretium eu tortor.

Integer imperdiet, est mattis imperdiet luctus, nunc nisl sodales justo, sit amet dapibus urna mauris sit amet diam. Donec et massa lectus. Cras nec pellentesque odio. Integer porta varius enim vel ornare. Donec nec commodo dui, a aliquet magna. Vestibulum sollicitudin nibh justo, ac mattis nibh volutpat et. Morbi eget condimentum enim, sit amet lobortis ligula. Vivamus nec mauris purus.
}
%
% ----------------------------------------------------------------
%
%	APÊNDICES
%
%	Texto complementar da tese.
%
\tueAPENDICES % Material de suporte
{
	%!TEX root = main.tex
\chapter{Bases Formais}

Lorem ipsum dolor sit amet, consectetur adipiscing elit. Vivamus vitae est vitae risus varius malesuada et eget velit. Morbi tincidunt venenatis tellus, in volutpat ante varius et. Fusce congue maximus velit ac dignissim. Integer hendrerit pharetra libero, at vehicula odio vestibulum eget. Etiam eget fringilla leo, sit amet posuere nisl. Aenean at tincidunt felis. Cras rhoncus mauris libero, a vestibulum \index{risus} faucibus quis. Aenean malesuada vitae nibh ut dapibus. Pellentesque vel blandit odio.

Maecenas massa leo, egestas id augue at, aliquam iaculis leo. Etiam ac lacus tempus, malesuada dolor vel, mattis leo. Duis tortor mi, accumsan vitae ligula eu, luctus accumsan diam. Etiam venenatis elit non magna aliquam eleifend. Phasellus in nunc at arcu iaculis ultrices sed sed ante. Nullam in velit a metus convallis vestibulum a vitae turpis. Proin fringilla dui tempor, ultrices metus nec, lobortis elit. Sed at posuere augue. Phasellus ac massa fringilla, convallis urna nec, aliquet orci. Mauris placerat tellus vel scelerisque tempus. Donec lacinia tincidunt mattis. Donec congue, augue sed ullamcorper placerat, erat nunc vestibulum tellus, vel consequat sem diam in magna. Vivamus ac dolor lacinia magna pharetra maximus. Nulla congue feugiat vehicula. Praesent luctus purus ac justo tempor eleifend.

Nunc eu ex vel ipsum ultrices molestie. In eget sodales turpis. Donec egestas facilisis nulla id feugiat. Duis \index{gravida} lorem quis porttitor interdum. Sed turpis leo, aliquet non metus a, vulputate volutpat ante. Donec neque metus, volutpat quis congue non, aliquam sed nunc. Curabitur erat mauris, elementum id rhoncus quis, condimentum eu felis. Quisque porta gravida velit a congue. Nulla gravida suscipit pulvinar. Sed sed erat ut turpis consequat sagittis. Sed scelerisque, massa ac tincidunt rutrum, libero dolor suscipit lorem, interdum dignissim massa enim a purus. Aliquam porta orci non urna sollicitudin, sed lobortis nibh ullamcorper. Aliquam erat volutpat. Phasellus ac purus in massa aliquet ultricies non sit amet justo.

Quisque placerat lobortis risus. Vestibulum ante ipsum primis in faucibus orci luctus et ultrices posuere cubilia Curae; Pellentesque eget odio sed lectus sollicitudin consectetur et ornare libero. Aliquam et ullamcorper arcu. Fusce mollis euismod purus, vitae auctor quam lobortis eu. Nunc mollis, velit eu cursus feugiat, nunc neque pellentesque arcu, a suscipit tellus nunc quis quam. Cras diam est, fermentum a rutrum sed, pretium eu tortor.

Integer imperdiet, est mattis imperdiet luctus, nunc nisl sodales justo, sit amet dapibus urna mauris sit amet diam. Donec et massa lectus. Cras nec pellentesque odio. Integer porta varius enim vel ornare. Donec nec commodo dui, a aliquet \index{magna}. Vestibulum sollicitudin nibh justo, ac mattis nibh volutpat et. Morbi eget condimentum enim, sit amet lobortis ligula. Vivamus nec mauris purus.
	%!TEX root = main.tex
\chapter{Resultados Empíricos}

Lorem ipsum dolor sit amet, consectetur adipiscing elit. Vivamus vitae est vitae risus varius malesuada et eget velit. Morbi tincidunt venenatis tellus, in volutpat ante varius et. Fusce congue maximus velit ac dignissim. Integer \index{hendrerit} pharetra libero, at vehicula odio vestibulum eget. Etiam eget fringilla leo, sit amet posuere nisl. Aenean at tincidunt felis. Cras rhoncus mauris libero, a vestibulum risus faucibus quis. Aenean malesuada vitae nibh ut dapibus. Pellentesque vel blandit odio.

Maecenas massa leo, egestas id augue at, aliquam iaculis leo. Etiam ac lacus tempus, malesuada dolor vel, mattis leo. Duis tortor mi, accumsan vitae ligula eu, luctus accumsan diam. Etiam venenatis elit non magna aliquam eleifend. Phasellus in nunc at arcu iaculis ultrices sed sed ante. Nullam in velit a metus convallis vestibulum a vitae turpis. Proin fringilla dui tempor, ultrices metus nec, lobortis elit. Sed at posuere augue. Phasellus ac massa fringilla, convallis urna nec, aliquet orci. Mauris placerat tellus vel scelerisque tempus. Donec lacinia tincidunt mattis. Donec congue, augue sed ullamcorper placerat, erat nunc vestibulum tellus, vel consequat sem diam in magna. Vivamus ac dolor lacinia magna pharetra maximus. Nulla congue feugiat vehicula. Praesent luctus purus ac justo tempor eleifend.

Nunc eu ex vel ipsum ultrices molestie. In eget sodales turpis. Donec egestas facilisis nulla id feugiat. Duis gravida lorem quis porttitor interdum. Sed turpis leo, aliquet non metus a, vulputate volutpat ante. Donec neque metus, volutpat quis congue non, aliquam sed nunc. Curabitur erat mauris, elementum id rhoncus quis, condimentum eu felis. Quisque porta gravida velit a congue. Nulla gravida suscipit pulvinar. Sed sed erat ut turpis consequat sagittis. Sed scelerisque, massa ac tincidunt rutrum, libero dolor suscipit lorem, interdum dignissim massa enim a purus. Aliquam porta orci non urna sollicitudin, sed lobortis nibh ullamcorper. Aliquam erat volutpat. Phasellus ac purus in massa aliquet ultricies non sit amet justo.

Quisque placerat lobortis risus. Vestibulum ante ipsum primis in faucibus orci luctus et ultrices posuere cubilia Curae; Pellentesque eget odio sed lectus sollicitudin consectetur et ornare libero. Aliquam et ullamcorper arcu. Fusce mollis euismod purus, vitae auctor quam lobortis eu. Nunc mollis, velit eu cursus feugiat, nunc \index{neque} pellentesque arcu, a suscipit tellus nunc quis quam. Cras diam est, fermentum a rutrum sed, pretium eu tortor.

Integer imperdiet, est mattis imperdiet luctus, nunc nisl \index{sodales} justo, sit amet dapibus urna mauris sit amet diam. Donec et massa lectus. Cras nec pellentesque odio. Integer porta varius enim vel ornare. Donec nec commodo dui, a aliquet magna. Vestibulum sollicitudin nibh justo, ac mattis nibh volutpat et. Morbi eget condimentum enim, sit amet lobortis ligula. Vivamus nec mauris purus.
}
%
% ----------------------------------------------------------------
%
%	BIBLIOGRAFIA
%
%	Por omissão...
%	- usa BibTex
%	- com o estilo "alpha"
%	- consulta o ficheiro "bibliografia.tex"
%	- lista **todas** as obras, mesmo que não referenciadas no texto da tese
%
%\tueBIBLIOGRAFIA{}
%
% ----------------------------------------------------------------
%
%	ÍNDICE REMISSIVO
%
%\tueINDICEREMISSIVO{}
%
% ----------------------------------------------------------------
%
% ================================================================
%
%	Modo ORGANIZAÇÃO DA DISSETAÇÃO COMPLETA.
%
%	Prevê que
%		- a informação sobre título, autor, orientadores, etc está definida acima e que
%		- a obra tem a seguinte estrutura:
%
%			prefácio
%			agradecimentos
%			tabela de conteúdos
%			lista de figuras
%			lista de tabelas
%			lista de acrónimos
%			sumário
%			tradução do sumário
%			------------------------------
%			CONTEÚDO (vários capítulos)
%			APÊNDICES (vários capítulos)
%			------------------------------
%			bibliografia
%			índice remissivo
%
% ================================================================
%
\tueDOCUMENTO
%
% ================================================================
%	Modo CAPA, CONTRA-CAPA e LOMBADAS.
%
%	Prevê que a informação sobre título, autor, orientadores, etc está definida acima.
%
% ================================================================
%
%\tueCAPAS

