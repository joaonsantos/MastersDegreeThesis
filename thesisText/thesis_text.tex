%!TEX program = xelatex
%
% ================================================================
%	Tipo de dissertação:
%		escolher entre "doutoramento" ou "mestrado"
%
%	Área científica:
%		escolher entre
%			- "ct" (ciências e tecnologia, final); "ctR" (ciências e tecnologia, rascunho);
%			- "csh" (ciências sociais e humanas, final); "cshR" (ciências sociais e humanas, rascunho);
%			- "artes" (artes, final); "artesR" (artes, rascunhos)
%
% ================================================================
%
\documentclass[mestrado,ctR,12pt]{teseue}
%
%
% ================================================================
%	DOCUMENTO:
%		 
%		Língua, Título, Nome do Candidato, Curso, etc
%		Estrutura
% ================================================================
%
% ----------------------------------------------------------------
%
%	LÍNGUA DA TESE
%
%	Opções atuais:
%	- PT: Português (novo acordo ortográfico)
%	- EN: Inglês
%
\tueLINGUA{EN}
%
% ----------------------------------------------------------------
%
%	TÍTULO DA TESE
%
%	Em Português e Inglês.
%
\tueTITULO
{Identity Management in Healthcare Using Blockchain Technology}
{}
%
% ----------------------------------------------------------------
%
%	SUBTÍTULO DA TESE
%
%	Em Português e Inglês.
%
\tueSUBTITULO
{}
{}
%
% ----------------------------------------------------------------
%
%	CANDIDATO
%
%	Nome completo.
%		
\tueCANDIDATO
{João Pedro Nunes dos Santos}
%
% ----------------------------------------------------------------
%
%	TÍTULO E NOME DO/A ORIENTADOR/A
%
%	Designação oficial e nome do orientador/a.
%	Em geral, "Orientador" ou "Orientadora".
%
\tueORIENTADOR
{Orientador}
{Pedro ... Salgueiro}
%
% ----------------------------------------------------------------
%
%	SEGUNDO ORIENTADOR/A (se aplicável)
%
%	Designação oficial e nome do segundo orientador/a.
%	Em geral, "Co-orientador" ou "Co-orientadora".
%
\tueSEGUNDOORIENTADOR
{Orientadora}
{Vítor ... Nogueira}
%
% ----------------------------------------------------------------
%
%	TERCEIRO ORIENTADOR/A (se aplicável)
%
%	Designação oficial e nome do terceiro orientador/a.
%	Em geral, "Co-orientador" ou "Co-orientadora".
%
%\tueTERCEIROORIENTADOR
%{Co-Orientador}
%{António Inácio Norberto}
%
% ----------------------------------------------------------------
%
%	CURSO
%
%	Nome do curso em que se enquadra esta tese.
%
\tueCURSO
{Engenharia Informática}
%
% ----------------------------------------------------------------
%
%	ESPECIALIDADE (se aplicável)
%
%	Nome da especialidade em que se enquadra esta tese.
%
%\tueESPECIALIDADE
%{Coordenação de Recursos Naturais}
%
% ----------------------------------------------------------------
%
%	DEPARTAMENTO
%
%	Departamento anfitrião do curso.
%
\tueDEPARTAMENTO
{Departamento de Informática}
%
% ----------------------------------------------------------------
%
%	ESCOLA
%
%	Escola a que pertence o departamento.
%
\tueESCOLA
{Escola de Ciências e Tecnologia}
%
% ----------------------------------------------------------------
%
%	PALAVRAS CHAVE
%
%	Data de submissão da tese.
%
\tuePALAVRASCHAVE
{Blockchain, Health, Identity, Big Data}
{Blockchain, Saúde, Identidade, Big Data}
%
% ----------------------------------------------------------------
%
%	DATA
%
%	Data de submissão da tese.
%
\tueDATA
{\today}
%
% ----------------------------------------------------------------
%
%	DEDICATÓRIA
%
\tueDEDICATORIA
{To My Family}
%
% ----------------------------------------------------------------
%
%	PREAMBULO
%
%	Comandos e definições para o LaTeX que devem estar **antes**
%	do texto do documento.
%
\tuePREAMBULOLATEX{
	\usepackage[figureright]{rotating}
}
%
% ----------------------------------------------------------------
%
%	PREAMBULO
%
%	Texto até à página 1. 
%
%	Por omissão os conteúdos estão definidos nos ficheiros
%		- prefacio.tex
%		- agradecimentos.tex
%		- acronimos.tex
%		- sumario.tex
%		- abstract.tex
%
\tuePREAMBULO {
  \chapter*{Acknowledgements}

Firstly, I want to thank my dissertation advisors, Pedro Salgueiro and Vítor
Beires Nogueira, for being patient with me, for their availability and for
their dedication in this project.

I want to thank my family who helped me finish this project, with their
unending support, words of wisdom and for always pushing me to do better.

I also want to thank my work colleagues, who always supported me, helped me
grow as a professional and person, challenged me to improve, and whom I
consider as a second family.

I need to thank my close friends for always believing in me and for always
being available when I needed the most, showing they are true friends.

Lastly, every person who supported me in some manner, I truly am grateful for
your support.

  %!TEX root = main.tex
\begin{tueACRONIMOS}
	\begin{acronym}[IEEE]
		\acro{EHR}{\emph{Electronic Health Record}}
		\acro{HL7}{\emph{Health Level 7}}
		\acro{DDOS}{\emph{Distributed Denial of Service}}
		\acro{BFT}{\emph{Byzantine Fault Tolerant}}
		\acro{GDPR}{\emph{General Data Protection Rule}}
		\acro{DLP}{\emph{Distributed Ledger Platform}}
		\acro{EVM}{\emph{Ethereum Virtual Machine}}
		\acro{SDK}{\emph{Software Development Kit}}
		\acro{MSP}{\emph{Membership Service Provider}}
		\acro{HLF}{\emph{Hyperledger Fabric}}
		\acro{CA}{\emph{Certificate Authority}}
		\acro{FHIR}{\emph{Fast Healthcare Interoperability Resources}}
		\acro{API}{\emph{Application Programming Interface}}
		\acro{JSON}{\emph{JavaScript Object Notation}}
		\acro{FHIR}{\emph{Fast Healthcare Interoperability Resources}}
		\acro{CIA}{\emph{Confidentiality, Integrity, and Availability}}
		\acro{SHA}{\emph{Secure Hash Algorithm}}
		\acro{TLS}{\emph{Transport Layer Security}}
	\end{acronym}
\end{tueACRONIMOS}

  %!TEX root = main.tex
\begin{tueSUMARIO}

  Lorem ipsum dolor sit amet, consectetur adipiscing elit. Vivamus vitae est
  vitae risus varius malesuada et eget velit. Morbi tincidunt venenatis tellus,
  in volutpat ante varius et. Fusce congue maximus velit ac dignissim. Integer
  hendrerit pharetra libero, at vehicula odio vestibulum eget. Etiam eget
  fringilla leo, sit amet posuere nisl. Aenean at tincidunt felis. Cras rhoncus
  mauris libero, a vestibulum risus faucibus quis. Aenean malesuada vitae nibh
  ut dapibus. Pellentesque vel blandit odio.

  Maecenas massa leo, egestas id augue at, aliquam iaculis leo. Etiam ac lacus
  tempus, malesuada dolor vel, mattis leo. Duis tortor mi, accumsan vitae
  ligula eu, luctus accumsan diam. Etiam venenatis elit non magna aliquam
  eleifend. Phasellus in nunc at arcu iaculis ultrices sed sed ante. Nullam in
  velit a metus convallis vestibulum a vitae turpis. Proin fringilla dui
  tempor, ultrices metus nec, lobortis elit. Sed at posuere augue. Phasellus ac
  massa fringilla, convallis urna nec, aliquet orci. Mauris placerat tellus vel
  scelerisque tempus. Donec lacinia tincidunt mattis. Donec congue, augue sed
  ullamcorper placerat, erat nunc vestibulum tellus, vel consequat sem diam in
  magna. Vivamus ac dolor lacinia magna pharetra maximus. Nulla congue feugiat
  vehicula. Praesent luctus purus ac justo tempor eleifend.

  Nunc eu ex vel ipsum ultrices molestie. In eget sodales turpis. Donec egestas
  facilisis nulla id feugiat. Duis gravida lorem quis porttitor interdum. Sed
  turpis leo, aliquet non metus a, vulputate volutpat ante. Donec neque metus,
  volutpat quis congue non, aliquam sed nunc. Curabitur erat mauris, elementum
  id rhoncus quis, condimentum eu felis. Quisque porta gravida velit a congue.
  Nulla gravida suscipit pulvinar. Sed sed erat ut turpis consequat sagittis.
  Sed scelerisque, massa ac tincidunt rutrum, libero dolor suscipit lorem,
  interdum dignissim massa enim a purus. Aliquam porta orci non urna
  sollicitudin, sed lobortis nibh ullamcorper. Aliquam erat volutpat. Phasellus
  ac purus in massa aliquet ultricies non sit amet justo.

  Quisque placerat lobortis risus. Vestibulum ante ipsum primis in faucibus orci
  luctus et ultrices posuere cubilia Curae; Pellentesque eget odio sed lectus
  sollicitudin consectetur et ornare libero. Aliquam et ullamcorper arcu. Fusce
  mollis euismod purus, vitae auctor quam lobortis eu. Nunc mollis, velit eu
  cursus feugiat, nunc neque pellentesque arcu, a suscipit tellus nunc quis
  quam. Cras diam est, fermentum a rutrum sed, pretium eu tortor.

\end{tueSUMARIO}

  \begin{tueABSTRACT}

  Bitcoin served as the catalyst for creating a solution to secure digital
  transactions without requiring a trusted third party to be involved. To solve
  this problem, the mechanisms now associated with a Blockchain were
  conceptualized and implemented to serve as the backbone for the Bitcoin
  network. More specifically, it was used as a security tool making Bitcoin a
  more transparent and reliable form of cash, a digital cryptographic currency.
  Even tough Bitcoin ended up not fulfilling its intended purpose as a
  currency, the Blockchain technology has enabled further avenues for
  innovation and creativity.

  Blockchain has since been used as the backbone for various cryptocurrencies
  networks. Some implementations of this technology allow the execution of
  code, also known as "smart contracts". Smart contracts are executed in an
  autonomous manner, with no human intervention. These can be used to solve a
  new set of problems due to their transparent behavior, lack of human
  intervention and distributed nature. 

  Blockchain technology allows the creation of systems that introduce a number
  of benefits over traditional data handling used in today's Healthcare
  Information Systems. Costs and risks associated with these systems can be
  reduced and information can become transparent and trustworthy to all
  participants.
  
  The Hyperledger Fabric Network with true private transactions and advanced
  security mechanisms was used to serve as the basis for the system proposed in
  this dissertation. Moreover, a client application was also created that
  interacts with smart contracts to manipulate the ledger.
  
  The work discussed in this dissertation shows that a Blockchain system based
  on Hyperledger Fabric is suitable for managing patients identity, in
  Healthcare. Even tough the feature set of this Blockchain is very focused in
  privacy and security, some additional measures regarding confidentiality of
  data had to be implemented.  Regardless, a system was built successfully that
  met the requirements. The implementation of this system would provide
  transparency, immutability and additional security for patients and medical
  staff alike. 

\end{tueABSTRACT}

}
%
% ----------------------------------------------------------------
%
%	CONTEÚDO
%
%	Texto principal da tese.
%
\tueCONTEUDO  % A partir da página 1
{
	%!TEX root = main.tex
\chapter{Introdução}

Lorem ipsum dolor sit amet, consectetur adipiscing elit. Vivamus vitae est vitae risus varius malesuada et eget velit. Morbi tincidunt venenatis tellus, in volutpat ante varius et. Fusce congue maximus velit ac dignissim. Integer hendrerit pharetra libero, at vehicula odio vestibulum eget. Etiam eget fringilla leo, sit amet posuere nisl. Aenean at tincidunt felis. Cras rhoncus mauris libero, a vestibulum risus faucibus quis. Aenean malesuada vitae nibh ut dapibus. Pellentesque vel blandit odio.

Maecenas massa leo, egestas id augue at, aliquam iaculis leo. Etiam ac lacus tempus, malesuada dolor vel, mattis leo. Duis tortor mi, accumsan vitae ligula eu, luctus accumsan diam. Etiam venenatis elit non magna aliquam eleifend. Phasellus in nunc at arcu iaculis ultrices sed sed ante. Nullam in velit a metus convallis vestibulum a vitae turpis. Proin fringilla dui tempor, ultrices metus nec, lobortis elit. Sed at posuere augue. Phasellus ac massa fringilla, convallis urna nec, aliquet orci. Mauris placerat tellus vel scelerisque tempus. Donec lacinia tincidunt mattis. Donec congue, augue sed ullamcorper placerat, erat nunc vestibulum tellus, vel consequat sem diam in magna. Vivamus ac dolor lacinia magna pharetra maximus. Nulla congue feugiat vehicula. Praesent luctus purus ac justo tempor eleifend.

Nunc eu ex vel ipsum ultrices molestie. In eget sodales turpis. Donec egestas facilisis nulla id feugiat. Duis gravida lorem quis porttitor interdum. Sed turpis leo, aliquet non metus a, vulputate volutpat ante. Donec neque metus, volutpat quis congue non, aliquam sed nunc. Curabitur erat mauris, elementum id rhoncus quis, condimentum eu felis. Quisque porta gravida velit a congue. Nulla gravida suscipit pulvinar. Sed sed erat ut turpis consequat sagittis. Sed scelerisque, massa ac tincidunt rutrum, libero dolor suscipit lorem, interdum dignissim massa enim a purus. Aliquam porta orci non urna sollicitudin, sed lobortis nibh ullamcorper. Aliquam erat volutpat. Phasellus ac purus in massa aliquet ultricies non sit amet justo.

Quisque placerat lobortis risus. Vestibulum ante ipsum primis in faucibus orci luctus et ultrices posuere cubilia Curae; Pellentesque eget odio sed lectus sollicitudin consectetur et ornare libero. Aliquam et ullamcorper arcu. Fusce mollis euismod purus, vitae auctor quam lobortis eu. Nunc mollis, velit eu cursus feugiat, nunc neque pellentesque arcu, a suscipit tellus nunc quis quam. Cras diam est, fermentum a rutrum sed, pretium eu tortor.

Integer imperdiet, est mattis imperdiet luctus, nunc nisl sodales justo, sit amet dapibus urna mauris sit amet diam. Donec et massa lectus. Cras nec pellentesque odio. Integer porta varius enim vel ornare. Donec nec commodo dui, a aliquet magna. Vestibulum sollicitudin nibh justo, ac mattis nibh volutpat et. Morbi eget condimentum enim, sit amet lobortis ligula. Vivamus nec mauris purus.

\section{Motivação}


Suspendisse ac dui et urna faucibus consequat. Integer porta vulputate lorem quis condimentum. Vestibulum ut tristique elit. Lorem ipsum dolor sit amet, consectetur adipiscing elit. In in orci id dolor tristique laoreet ut at mauris. Suspendisse mollis leo nulla, nec vehicula mi ultricies sit amet. Nullam nulla purus, blandit et dui a, rhoncus vehicula diam. Aenean sagittis lorem in nunc cursus luctus. Aliquam condimentum ipsum volutpat purus mollis tincidunt. Donec non erat orci. Donec tincidunt sit amet libero ac vulputate. Pellentesque vitae ex vitae odio sollicitudin vestibulum. Ut hendrerit placerat sagittis. Phasellus posuere a felis id feugiat. Maecenas pulvinar, sapien sit amet maximus facilisis, metus arcu pretium nisi, ut consectetur eros nisl eget magna.

\begin{figure}
	\begin{center}
		\begin{tikzpicture}
			%
			\draw[step = 0.5, color = uegray!50!white] (0,0) grid (3.0,3.0); 
			\draw[->,thick] (0,0) -- (3.25,0);
			\draw[->,thick] (0,0) -- (0,3.25);
			%
			\coordinate (A) at (0.0,2.0);
			\coordinate (B) at (0.5,1.5);
			\coordinate (C) at (1.0,0.0);
			\coordinate (D) at (1.5,1.0);
			\coordinate (E) at (2.0,2.0);
			\coordinate (F) at (2.5,2.5);
			%
			\node at (A) {$\circ$};
			\node at (B) {$\circ$};
			\node at (C) {$\circ$};
			\node at (D) {$\circ$};
			\node at (E) {$\circ$};
			\node at (F) {$\circ$};
			%
			\draw[very thick, color = uered] (A) -- (B) -- (C) -- (D) -- (E) -- (F);
			%
		\end{tikzpicture}
	\end{center}
	\caption{Exemplo de uma figura.}
\end{figure}

Quisque fringilla dictum \index{tellus} sed faucibus. Vestibulum eget augue pellentesque, rutrum turpis ac, efficitur nunc. Integer molestie, erat at vehicula commodo, orci nunc consequat diam, fermentum finibus dui ante sit amet eros. Donec venenatis, erat sit amet vestibulum rutrum, neque metus ultrices enim, vel aliquam tellus tortor ut magna. Nullam vitae dolor ipsum. Aliquam erat volutpat. Nulla sagittis elit vel felis tempor feugiat. In ultrices mattis risus, in \index{ornare} lectus faucibus et. In hac habitasse platea dictumst. Praesent pretium aliquam tincidunt. Cras pellentesque lectus ipsum, id fermentum nulla lobortis eu. Fusce tristique dui a \index{diam} semper, vitae dapibus urna ullamcorper. Sed tincidunt elit cursus imperdiet ultricies.

Etiam eu leo diam. Nam \index{imperdiet} neque maximus, pharetra mauris nec, euismod odio. Pellentesque pretium vehicula elit at tincidunt. Nam posuere suscipit dapibus. Nulla aliquam venenatis mi non porta. Nam euismod quam id faucibus sodales. Aenean scelerisque mi et est condimentum eleifend. Duis finibus, lectus quis iaculis lobortis, mauris augue finibus justo, ut semper sem ligula nec risus. Pellentesque vel fermentum felis. Integer vehicula tristique enim id porta. Proin imperdiet tristique libero, a ultricies sem imperdiet a. Ut iaculis facilisis libero sed dictum. Nulla facilisi. Quisque volutpat neque sit amet arcu condimentum, ut efficitur urna consectetur. Aenean ultricies ante sapien, quis pharetra erat elementum faucibus. Nulla lobortis tristique dolor at posuere.

\subsection{Oportunidades}

Curabitur \index{auctor} ante sit amet velit placerat, et vulputate arcu cursus. Curabitur eu molestie tellus, ac euismod felis. Morbi scelerisque justo non eleifend scelerisque. Integer pellentesque massa ac tortor porta, ac commodo lorem aliquet. Donec a metus a libero accumsan ultricies. Integer semper elementum lorem vel \index{blandit}. Vivamus tristique turpis lectus, id rutrum eros interdum eu. Proin facilisis sodales eros eu porttitor. Nunc tristique vehicula risus, eu dapibus lacus suscipit ut. Nunc nec purus at orci aliquet lacinia.

\begin{table}
	\caption{Exemplo de uma tabela.}
	\begin{center}
		\begin{tabular}{lcc|lcc}
			\textbf{Astro} & \textbf{Dia} & \textbf{Ano} & \textbf{Astro} & \textbf{Dia} & \textbf{Ano}\\
			\hline
			Sol & -- & -- & Mercúrio  & 0.5 & 20 \\
			Vénus & 0.75 & 30 & Terra & 1 & 37 \\
			Marte & 1.5 & 45 & Lua & 3 & 3 
		\end{tabular}
	\end{center}
\end{table}

Ut rhoncus tellus nec aliquam iaculis. Aenean consectetur diam id nunc facilisis porta. Duis euismod est id risus feugiat, eu aliquam nisl imperdiet. Donec id enim feugiat, consectetur nisl in, iaculis felis. Ut et mattis elit. Mauris hendrerit, velit sit amet tristique bibendum, est nisi pretium felis, id maximus eros lorem eu sapien. Aliquam sollicitudin eros magna, et congue orci aliquam sed. 
	%!TEX root = main.tex
\chapter{Estado da Arte}

Lorem ipsum dolor sit amet, consectetur adipiscing elit. Vivamus vitae est vitae risus varius malesuada et eget velit. Morbi tincidunt venenatis tellus, in volutpat ante varius et. Fusce congue maximus velit ac dignissim. Integer hendrerit pharetra libero, at vehicula odio vestibulum eget. Etiam eget fringilla leo, sit amet posuere nisl. Aenean at tincidunt felis. Cras rhoncus mauris libero, a vestibulum risus faucibus quis. Aenean malesuada vitae nibh ut dapibus. Pellentesque vel blandit odio \cite{barber2012bayesian}.

Maecenas massa leo, egestas id augue at, aliquam iaculis leo. Etiam ac lacus tempus, malesuada dolor vel, mattis leo. Duis tortor mi, accumsan vitae ligula eu, luctus accumsan diam. \index{Etiam} venenatis elit non magna aliquam eleifend. Phasellus in nunc at arcu iaculis ultrices sed sed ante. Nullam in velit a metus convallis vestibulum a vitae turpis. Proin fringilla dui tempor, ultrices metus nec, lobortis elit. Sed at \index{posuere} augue. Phasellus ac massa fringilla, convallis urna nec, aliquet orci. Mauris placerat tellus vel scelerisque tempus. Donec lacinia tincidunt mattis. Donec congue, augue sed ullamcorper placerat, erat nunc vestibulum tellus, vel consequat sem diam in magna. Vivamus ac dolor lacinia magna pharetra maximus. Nulla congue feugiat vehicula. Praesent luctus purus ac \index{justo} tempor eleifend.

Nunc eu ex vel ipsum \index{ultrices} molestie. In eget sodales turpis. Donec egestas facilisis nulla id feugiat. Duis gravida lorem quis porttitor interdum. Sed turpis leo, aliquet non metus a, vulputate volutpat ante. Donec neque metus, volutpat quis congue non, aliquam sed nunc. Curabitur erat mauris, elementum id rhoncus quis, condimentum eu felis. Quisque porta gravida velit a congue. Nulla gravida suscipit pulvinar. Sed sed erat ut turpis consequat sagittis. Sed scelerisque, massa ac tincidunt rutrum, libero dolor suscipit lorem, interdum dignissim massa enim a purus. Aliquam porta orci non urna sollicitudin, sed lobortis nibh ullamcorper. Aliquam erat volutpat. Phasellus ac purus in massa aliquet ultricies non sit amet justo.

Quisque placerat lobortis risus. Vestibulum ante ipsum primis in faucibus orci luctus et ultrices posuere cubilia Curae; Pellentesque eget odio sed lectus sollicitudin consectetur et ornare libero. Aliquam et ullamcorper arcu. Fusce mollis euismod purus, vitae auctor quam lobortis eu. Nunc mollis, velit eu cursus feugiat, nunc neque pellentesque arcu, a suscipit tellus nunc quis quam. Cras diam est, fermentum a rutrum sed, pretium eu tortor.

Integer imperdiet, est mattis imperdiet luctus, nunc nisl sodales justo, sit amet dapibus urna mauris sit amet diam. Donec et massa lectus. Cras nec pellentesque odio. Integer porta varius enim vel ornare. Donec nec \index{commodo} dui, a aliquet magna. Vestibulum sollicitudin nibh justo, ac mattis nibh volutpat et. Morbi eget condimentum enim, sit amet lobortis ligula. Vivamus nec mauris purus.
	%!TEX root = main.tex
\chapter{Resolução}

Lorem ipsum dolor sit amet, consectetur adipiscing elit. Vivamus vitae est vitae risus varius malesuada et eget velit. Morbi tincidunt venenatis tellus, in volutpat ante varius et. Fusce congue maximus velit ac dignissim. Integer hendrerit pharetra libero, at vehicula odio vestibulum eget. Etiam eget fringilla leo, sit amet posuere nisl. Aenean at tincidunt felis. Cras rhoncus mauris libero, a vestibulum risus faucibus quis. Aenean malesuada vitae nibh ut dapibus. Pellentesque vel blandit odio.

Maecenas massa leo, egestas id augue at, aliquam iaculis leo. Etiam ac lacus tempus, malesuada dolor vel, mattis leo. Duis tortor mi, accumsan vitae ligula eu, luctus accumsan diam. Etiam venenatis elit non magna aliquam eleifend. Phasellus in nunc at arcu iaculis ultrices sed sed ante. Nullam in velit a metus convallis vestibulum a vitae turpis. Proin fringilla dui tempor, ultrices metus nec, lobortis elit. Sed at posuere augue. Phasellus ac massa fringilla, convallis urna nec, aliquet orci. Mauris placerat tellus vel scelerisque tempus. Donec lacinia tincidunt mattis. Donec congue, augue sed ullamcorper placerat, erat nunc vestibulum tellus, vel consequat sem diam in magna. Vivamus ac dolor lacinia magna pharetra maximus. Nulla congue feugiat vehicula. Praesent luctus purus ac justo tempor eleifend.

Nunc eu ex vel ipsum ultrices molestie. In eget sodales turpis. Donec egestas \index{facilisis} nulla id feugiat. Duis gravida lorem quis porttitor interdum. Sed turpis leo, aliquet non metus a, vulputate volutpat ante. Donec neque metus, volutpat quis congue non, aliquam sed nunc. Curabitur erat mauris, elementum id rhoncus quis, condimentum eu felis. Quisque porta gravida velit a congue. Nulla gravida suscipit pulvinar. Sed sed erat ut turpis consequat sagittis. Sed scelerisque, massa ac tincidunt rutrum, libero dolor suscipit lorem, interdum dignissim massa enim a purus. Aliquam porta orci non urna sollicitudin, sed lobortis nibh ullamcorper. Aliquam erat \index{volutpat}. Phasellus ac purus in massa aliquet ultricies non sit amet justo.

Quisque placerat lobortis risus. Vestibulum ante ipsum primis in faucibus orci luctus et ultrices posuere cubilia Curae; Pellentesque eget odio sed \index{lectus} sollicitudin consectetur et ornare libero. Aliquam et ullamcorper arcu. Fusce mollis euismod purus, vitae auctor quam lobortis eu. Nunc mollis, velit eu cursus feugiat, nunc neque pellentesque arcu, a suscipit tellus nunc quis \index{quam}. Cras diam est, fermentum a rutrum sed, pretium eu tortor.

Integer imperdiet, est mattis imperdiet luctus, nunc nisl sodales justo, sit amet dapibus urna mauris sit amet diam. Donec et massa lectus. Cras nec pellentesque odio. Integer porta varius enim vel ornare. Donec nec commodo dui, a aliquet magna. Vestibulum sollicitudin nibh justo, ac mattis nibh volutpat et. Morbi eget condimentum enim, sit amet lobortis ligula. Vivamus nec mauris purus.
}
%
% ----------------------------------------------------------------
%
%	APÊNDICES
%
%	Texto complementar da tese.
%
\tueAPENDICES % Material de suporte
{
	%!TEX root = main.tex
\chapter{Bases Formais}

Lorem ipsum dolor sit amet, consectetur adipiscing elit. Vivamus vitae est vitae risus varius malesuada et eget velit. Morbi tincidunt venenatis tellus, in volutpat ante varius et. Fusce congue maximus velit ac dignissim. Integer hendrerit pharetra libero, at vehicula odio vestibulum eget. Etiam eget fringilla leo, sit amet posuere nisl. Aenean at tincidunt felis. Cras rhoncus mauris libero, a vestibulum \index{risus} faucibus quis. Aenean malesuada vitae nibh ut dapibus. Pellentesque vel blandit odio.

Maecenas massa leo, egestas id augue at, aliquam iaculis leo. Etiam ac lacus tempus, malesuada dolor vel, mattis leo. Duis tortor mi, accumsan vitae ligula eu, luctus accumsan diam. Etiam venenatis elit non magna aliquam eleifend. Phasellus in nunc at arcu iaculis ultrices sed sed ante. Nullam in velit a metus convallis vestibulum a vitae turpis. Proin fringilla dui tempor, ultrices metus nec, lobortis elit. Sed at posuere augue. Phasellus ac massa fringilla, convallis urna nec, aliquet orci. Mauris placerat tellus vel scelerisque tempus. Donec lacinia tincidunt mattis. Donec congue, augue sed ullamcorper placerat, erat nunc vestibulum tellus, vel consequat sem diam in magna. Vivamus ac dolor lacinia magna pharetra maximus. Nulla congue feugiat vehicula. Praesent luctus purus ac justo tempor eleifend.

Nunc eu ex vel ipsum ultrices molestie. In eget sodales turpis. Donec egestas facilisis nulla id feugiat. Duis \index{gravida} lorem quis porttitor interdum. Sed turpis leo, aliquet non metus a, vulputate volutpat ante. Donec neque metus, volutpat quis congue non, aliquam sed nunc. Curabitur erat mauris, elementum id rhoncus quis, condimentum eu felis. Quisque porta gravida velit a congue. Nulla gravida suscipit pulvinar. Sed sed erat ut turpis consequat sagittis. Sed scelerisque, massa ac tincidunt rutrum, libero dolor suscipit lorem, interdum dignissim massa enim a purus. Aliquam porta orci non urna sollicitudin, sed lobortis nibh ullamcorper. Aliquam erat volutpat. Phasellus ac purus in massa aliquet ultricies non sit amet justo.

Quisque placerat lobortis risus. Vestibulum ante ipsum primis in faucibus orci luctus et ultrices posuere cubilia Curae; Pellentesque eget odio sed lectus sollicitudin consectetur et ornare libero. Aliquam et ullamcorper arcu. Fusce mollis euismod purus, vitae auctor quam lobortis eu. Nunc mollis, velit eu cursus feugiat, nunc neque pellentesque arcu, a suscipit tellus nunc quis quam. Cras diam est, fermentum a rutrum sed, pretium eu tortor.

Integer imperdiet, est mattis imperdiet luctus, nunc nisl sodales justo, sit amet dapibus urna mauris sit amet diam. Donec et massa lectus. Cras nec pellentesque odio. Integer porta varius enim vel ornare. Donec nec commodo dui, a aliquet \index{magna}. Vestibulum sollicitudin nibh justo, ac mattis nibh volutpat et. Morbi eget condimentum enim, sit amet lobortis ligula. Vivamus nec mauris purus.
	%!TEX root = main.tex
\chapter{Resultados Empíricos}

Lorem ipsum dolor sit amet, consectetur adipiscing elit. Vivamus vitae est vitae risus varius malesuada et eget velit. Morbi tincidunt venenatis tellus, in volutpat ante varius et. Fusce congue maximus velit ac dignissim. Integer \index{hendrerit} pharetra libero, at vehicula odio vestibulum eget. Etiam eget fringilla leo, sit amet posuere nisl. Aenean at tincidunt felis. Cras rhoncus mauris libero, a vestibulum risus faucibus quis. Aenean malesuada vitae nibh ut dapibus. Pellentesque vel blandit odio.

Maecenas massa leo, egestas id augue at, aliquam iaculis leo. Etiam ac lacus tempus, malesuada dolor vel, mattis leo. Duis tortor mi, accumsan vitae ligula eu, luctus accumsan diam. Etiam venenatis elit non magna aliquam eleifend. Phasellus in nunc at arcu iaculis ultrices sed sed ante. Nullam in velit a metus convallis vestibulum a vitae turpis. Proin fringilla dui tempor, ultrices metus nec, lobortis elit. Sed at posuere augue. Phasellus ac massa fringilla, convallis urna nec, aliquet orci. Mauris placerat tellus vel scelerisque tempus. Donec lacinia tincidunt mattis. Donec congue, augue sed ullamcorper placerat, erat nunc vestibulum tellus, vel consequat sem diam in magna. Vivamus ac dolor lacinia magna pharetra maximus. Nulla congue feugiat vehicula. Praesent luctus purus ac justo tempor eleifend.

Nunc eu ex vel ipsum ultrices molestie. In eget sodales turpis. Donec egestas facilisis nulla id feugiat. Duis gravida lorem quis porttitor interdum. Sed turpis leo, aliquet non metus a, vulputate volutpat ante. Donec neque metus, volutpat quis congue non, aliquam sed nunc. Curabitur erat mauris, elementum id rhoncus quis, condimentum eu felis. Quisque porta gravida velit a congue. Nulla gravida suscipit pulvinar. Sed sed erat ut turpis consequat sagittis. Sed scelerisque, massa ac tincidunt rutrum, libero dolor suscipit lorem, interdum dignissim massa enim a purus. Aliquam porta orci non urna sollicitudin, sed lobortis nibh ullamcorper. Aliquam erat volutpat. Phasellus ac purus in massa aliquet ultricies non sit amet justo.

Quisque placerat lobortis risus. Vestibulum ante ipsum primis in faucibus orci luctus et ultrices posuere cubilia Curae; Pellentesque eget odio sed lectus sollicitudin consectetur et ornare libero. Aliquam et ullamcorper arcu. Fusce mollis euismod purus, vitae auctor quam lobortis eu. Nunc mollis, velit eu cursus feugiat, nunc \index{neque} pellentesque arcu, a suscipit tellus nunc quis quam. Cras diam est, fermentum a rutrum sed, pretium eu tortor.

Integer imperdiet, est mattis imperdiet luctus, nunc nisl \index{sodales} justo, sit amet dapibus urna mauris sit amet diam. Donec et massa lectus. Cras nec pellentesque odio. Integer porta varius enim vel ornare. Donec nec commodo dui, a aliquet magna. Vestibulum sollicitudin nibh justo, ac mattis nibh volutpat et. Morbi eget condimentum enim, sit amet lobortis ligula. Vivamus nec mauris purus.
}
%
% ----------------------------------------------------------------
%
%	BIBLIOGRAFIA
%
%	Por omissão...
%	- usa BibTex
%	- com o estilo "alpha"
%	- consulta o ficheiro "bibliografia.tex"
%	- lista **todas** as obras, mesmo que não referenciadas no texto da tese
%
%\tueBIBLIOGRAFIA{}
%
% ----------------------------------------------------------------
%
%	ÍNDICE REMISSIVO
%
%\tueINDICEREMISSIVO{}
%
% ----------------------------------------------------------------
%
% ================================================================
%
%	Modo ORGANIZAÇÃO DA DISSETAÇÃO COMPLETA.
%
%	Prevê que
%		- a informação sobre título, autor, orientadores, etc está definida acima e que
%		- a obra tem a seguinte estrutura:
%
%			prefácio
%			agradecimentos
%			tabela de conteúdos
%			lista de figuras
%			lista de tabelas
%			lista de acrónimos
%			sumário
%			tradução do sumário
%			------------------------------
%			CONTEÚDO (vários capítulos)
%			APÊNDICES (vários capítulos)
%			------------------------------
%			bibliografia
%			índice remissivo
%
% ================================================================
%
\tueDOCUMENTO
%
% ================================================================
%	Modo CAPA, CONTRA-CAPA e LOMBADAS.
%
%	Prevê que a informação sobre título, autor, orientadores, etc está definida acima.
%
% ================================================================
%
%\tueCAPAS

