%!TEX program = xelatex
%
% ================================================================
%	Tipo de dissertação:
%		escolher entre "doutoramento" ou "mestrado"
%
%	Área científica:
%		escolher entre
%			- "ct" (ciências e tecnologia, final); "ctR" (ciências e tecnologia, rascunho);
%			- "csh" (ciências sociais e humanas, final); "cshR" (ciências sociais e humanas, rascunho);
%			- "artes" (artes, final); "artesR" (artes, rascunhos)
%
% ================================================================
%
\documentclass[mestrado,ctR,12pt]{teseue}
%
%
% ================================================================
%	DOCUMENTO:
%		 
%		Língua, Título, Nome do Candidato, Curso, etc
%		Estrutura
% ================================================================
%
% ----------------------------------------------------------------
%
%	LÍNGUA DA TESE
%
%	Opções atuais:
%	- PT: Português (novo acordo ortográfico)
%	- EN: Inglês
%
\tueLINGUA{EN}
%
% ----------------------------------------------------------------
%
%	TÍTULO DA TESE
%
%	Em Português e Inglês.
%
\tueTITULO
{Identity Management in Healthcare Using Blockchain Technology}
{}
%
% ----------------------------------------------------------------
%
%	SUBTÍTULO DA TESE
%
%	Em Português e Inglês.
%
\tueSUBTITULO
{}
{}
%
% ----------------------------------------------------------------
%
%	CANDIDATO
%
%	Nome completo.
%		
\tueCANDIDATO
{João Pedro Nunes dos Santos}
%
% ----------------------------------------------------------------
%
%	TÍTULO E NOME DO/A ORIENTADOR/A
%
%	Designação oficial e nome do orientador/a.
%	Em geral, "Orientador" ou "Orientadora".
%
\tueORIENTADOR
{Orientador}
{Pedro Salgueiro}
%
% ----------------------------------------------------------------
%
%	SEGUNDO ORIENTADOR/A (se aplicável)
%
%	Designação oficial e nome do segundo orientador/a.
%	Em geral, "Co-orientador" ou "Co-orientadora".
%
\tueSEGUNDOORIENTADOR
{Orientadora}
{Vítor Beires Nogueira}
%
% ----------------------------------------------------------------
%
%	TERCEIRO ORIENTADOR/A (se aplicável)
%
%	Designação oficial e nome do terceiro orientador/a.
%	Em geral, "Co-orientador" ou "Co-orientadora".
%
%\tueTERCEIROORIENTADOR
%{Co-Orientador}
%{António Inácio Norberto}
%
% ----------------------------------------------------------------
%
%	CURSO
%
%	Nome do curso em que se enquadra esta tese.
%
\tueCURSO
{Engenharia Informática}
%
% ----------------------------------------------------------------
%
%	ESPECIALIDADE (se aplicável)
%
%	Nome da especialidade em que se enquadra esta tese.
%
%\tueESPECIALIDADE
%{Coordenação de Recursos Naturais}
%
% ----------------------------------------------------------------
%
%	DEPARTAMENTO
%
%	Departamento anfitrião do curso.
%
\tueDEPARTAMENTO
{Departamento de Informática}
%
% ----------------------------------------------------------------
%
%	ESCOLA
%
%	Escola a que pertence o departamento.
%
\tueESCOLA
{Escola de Ciências e Tecnologia}
%
% ----------------------------------------------------------------
%
%	PALAVRAS CHAVE
%
%	Data de submissão da tese.
%
\tuePALAVRASCHAVE
{Blockchain, Saúde, Identidade, Big Data}
{Blockchain, Health, Identity, Big Data}
%
% ----------------------------------------------------------------
%
%	DATA
%
%	Data de submissão da tese.
%
\tueDATA
{\today}
%
% ----------------------------------------------------------------
%
%	DEDICATÓRIA
%
\tueDEDICATORIA
{To My Family}
%
% ----------------------------------------------------------------
%
%	PREAMBULO
%
%	Comandos e definições para o LaTeX que devem estar **antes**
%	do texto do documento.
%
\tuePREAMBULOLATEX{
	\usepackage[figureright]{rotating}
}
%
% ----------------------------------------------------------------
%
%	PREAMBULO
%
%	Texto até à página 1. 
%
%	Por omissão os conteúdos estão definidos nos ficheiros
%		- prefacio.tex
%		- agradecimentos.tex
%		- acronimos.tex
%		- sumario.tex
%		- abstract.tex
%
%\tuePREAMBULO {}
%
% ----------------------------------------------------------------
%
%	CONTEÚDO
%
%	Texto principal da tese.
%
\tueCONTEUDO  % A partir da página 1
{
	\chapter{Introduction}
\label{introduction}

\begin{quote} 
  \emph{This Chapter introduces the main topics and technologies covered by
  this dissertation. Healthcare and its relationship with technology is
  presented. The current flaws associated with patients identity data
  management are described. The Blockchain technology is introduced as a
  potential solution to some of these problems.} 
\end{quote}

The aim of this dissertation is to create a solution for managing the identity
of patients in the Healthcare environment by using Blockchain technology, and
in turn, evaluate the use of this technology in this specific use case.  Health
is intrinsically linked with technology, as new technologies enable safer and
better treatments. Nowadays, Healthcare organizations store patients data on a
digital format. The Electronic Health Record (EHR) is an abstract concept
representing the patients digitally stored clinical data and their identity in
a medical and clinical context.

Standards are an important aspect to take into account when designing an
information system because they allow interoperability between different
organizations. The Health Level 7 (HL7) Fast Healthcare Interoperability
Resources (FHIR) standard (see Section~\ref{blockchainHealthcare}), is being
built primarily by the Health Level Seven organization. Over the last few
years, it has seen a significant growth in usage. It is also an international
standard with partnerships worldwide. HL7 Portugal is now starting its
operations and is building a community to support this standard in
Portugal~\cite{HealthLevel7}.

Blockchain is often known as the technology behind the Bitcoin cryptocurrency.
Bitcoin depends on two complementary technologies, digital tokens and a
Blockchain, that when orchestrated together facilitate trust, immutability and
resiliency~\cite{Evans2016}.

A Blockchain runs on a network of computers and has a list of records that are
replicated across the participating peers. Blockchain, as we know today, was
conceptualized as the public ledger~\footnote{A ledger is defined as an object
in which items are regularly recorded, originally business activities and money
received or paid, but in reality, it can be used to store any type of record.}
for the Bitcoin cryptocurrency in 2008 by Satoshi Nakamoto~\cite{Nakamoto2008}.
Satoshi Nakamoto is a pen name of, a still unknown to this day, individual or
organization of individuals.

Traditional Healthcare databases and architectures are increasingly vulnerable
and a target to groups of malicious actors that possess the technical expertise
to deny services with Distributed Denial of Service (DDOS) attacks~\footnote{A
Distributed Denial of Service attack is an attempt to make an online service
unavailable by overwhelming it with traffic from multiple sources.} or cause a
data breach~\footnote{A data breach is the intentional or unintentional release
of secure or private/confidential information to an untrusted
environment.}~\cite{mcCoy2018}. 

Making matters worse other problems spring to mind. The data that comprises the
identity of a patient is often fragmented across multiple Healthcare
organizations, in such a way that, to get a true overview of the patients
history and diagnosis there would be a need to merge all the pieces of
information stored in data systems that are hosted in architecturally different
Healthcare information systems. Transparency is also a concern, as a patient
does not currently possess the means to track how his medical data is being
handled.

As more information becomes available, new insights can be extracted by
Healthcare professionals that lead to an overall improvement of the patients
interaction with the Healthcare ecosystem. However, maintaining a high amount
of data secure is a costly and risky matter for every party involved. Security
and privacy are a top concern regarding sensitive data. 

This dissertation provides an insight into the design and implementation of a
Blockchain based system for managing the identity of patients in an Healthcare
setting and its subsequent evaluation. The creation of this system and its
subsequent evaluation could provide interesting conclusions to medical staff as
well as patients, regarding its potential implementation and deployment in the
field.

In this document, different Blockchain implementations are explored to get an
overview of their feature set and focus. Considering a set of defined
requirements a platform is chosen, in order to evaluate the suitability of this
technology in the Healthcare field. More precisely, in
Chapter~\ref{background}, a brief introduction to Blockchain and its most
prominent implementations is presented. The technology is further explored in
Chapter~\ref{blockchain} and a number of real world use cases of this
technology in the Healthcare field are explored.  In Chapter~\ref{development}
a Blockchain platform is chosen in order to build a prototype system to
evaluate the usability of this technology in the Healthcare field. Insight is
given into the system design, implementation. In Chapter~\ref{experiments}, the
system is tested and evaluated. Finally, in Chapter~\ref{Conclusion} some
conclusions are presented and potential future work is discussed.

	%!TEX root = main.tex
\chapter{Estado da Arte}

Lorem ipsum dolor sit amet, consectetur adipiscing elit. Vivamus vitae est vitae risus varius malesuada et eget velit. Morbi tincidunt venenatis tellus, in volutpat ante varius et. Fusce congue maximus velit ac dignissim. Integer hendrerit pharetra libero, at vehicula odio vestibulum eget. Etiam eget fringilla leo, sit amet posuere nisl. Aenean at tincidunt felis. Cras rhoncus mauris libero, a vestibulum risus faucibus quis. Aenean malesuada vitae nibh ut dapibus. Pellentesque vel blandit odio \cite{barber2012bayesian}.

Maecenas massa leo, egestas id augue at, aliquam iaculis leo. Etiam ac lacus tempus, malesuada dolor vel, mattis leo. Duis tortor mi, accumsan vitae ligula eu, luctus accumsan diam. \index{Etiam} venenatis elit non magna aliquam eleifend. Phasellus in nunc at arcu iaculis ultrices sed sed ante. Nullam in velit a metus convallis vestibulum a vitae turpis. Proin fringilla dui tempor, ultrices metus nec, lobortis elit. Sed at \index{posuere} augue. Phasellus ac massa fringilla, convallis urna nec, aliquet orci. Mauris placerat tellus vel scelerisque tempus. Donec lacinia tincidunt mattis. Donec congue, augue sed ullamcorper placerat, erat nunc vestibulum tellus, vel consequat sem diam in magna. Vivamus ac dolor lacinia magna pharetra maximus. Nulla congue feugiat vehicula. Praesent luctus purus ac \index{justo} tempor eleifend.

Nunc eu ex vel ipsum \index{ultrices} molestie. In eget sodales turpis. Donec egestas facilisis nulla id feugiat. Duis gravida lorem quis porttitor interdum. Sed turpis leo, aliquet non metus a, vulputate volutpat ante. Donec neque metus, volutpat quis congue non, aliquam sed nunc. Curabitur erat mauris, elementum id rhoncus quis, condimentum eu felis. Quisque porta gravida velit a congue. Nulla gravida suscipit pulvinar. Sed sed erat ut turpis consequat sagittis. Sed scelerisque, massa ac tincidunt rutrum, libero dolor suscipit lorem, interdum dignissim massa enim a purus. Aliquam porta orci non urna sollicitudin, sed lobortis nibh ullamcorper. Aliquam erat volutpat. Phasellus ac purus in massa aliquet ultricies non sit amet justo.

Quisque placerat lobortis risus. Vestibulum ante ipsum primis in faucibus orci luctus et ultrices posuere cubilia Curae; Pellentesque eget odio sed lectus sollicitudin consectetur et ornare libero. Aliquam et ullamcorper arcu. Fusce mollis euismod purus, vitae auctor quam lobortis eu. Nunc mollis, velit eu cursus feugiat, nunc neque pellentesque arcu, a suscipit tellus nunc quis quam. Cras diam est, fermentum a rutrum sed, pretium eu tortor.

Integer imperdiet, est mattis imperdiet luctus, nunc nisl sodales justo, sit amet dapibus urna mauris sit amet diam. Donec et massa lectus. Cras nec pellentesque odio. Integer porta varius enim vel ornare. Donec nec \index{commodo} dui, a aliquet magna. Vestibulum sollicitudin nibh justo, ac mattis nibh volutpat et. Morbi eget condimentum enim, sit amet lobortis ligula. Vivamus nec mauris purus.
	%!TEX root = main.tex
\chapter{Resolução}

Lorem ipsum dolor sit amet, consectetur adipiscing elit. Vivamus vitae est vitae risus varius malesuada et eget velit. Morbi tincidunt venenatis tellus, in volutpat ante varius et. Fusce congue maximus velit ac dignissim. Integer hendrerit pharetra libero, at vehicula odio vestibulum eget. Etiam eget fringilla leo, sit amet posuere nisl. Aenean at tincidunt felis. Cras rhoncus mauris libero, a vestibulum risus faucibus quis. Aenean malesuada vitae nibh ut dapibus. Pellentesque vel blandit odio.

Maecenas massa leo, egestas id augue at, aliquam iaculis leo. Etiam ac lacus tempus, malesuada dolor vel, mattis leo. Duis tortor mi, accumsan vitae ligula eu, luctus accumsan diam. Etiam venenatis elit non magna aliquam eleifend. Phasellus in nunc at arcu iaculis ultrices sed sed ante. Nullam in velit a metus convallis vestibulum a vitae turpis. Proin fringilla dui tempor, ultrices metus nec, lobortis elit. Sed at posuere augue. Phasellus ac massa fringilla, convallis urna nec, aliquet orci. Mauris placerat tellus vel scelerisque tempus. Donec lacinia tincidunt mattis. Donec congue, augue sed ullamcorper placerat, erat nunc vestibulum tellus, vel consequat sem diam in magna. Vivamus ac dolor lacinia magna pharetra maximus. Nulla congue feugiat vehicula. Praesent luctus purus ac justo tempor eleifend.

Nunc eu ex vel ipsum ultrices molestie. In eget sodales turpis. Donec egestas \index{facilisis} nulla id feugiat. Duis gravida lorem quis porttitor interdum. Sed turpis leo, aliquet non metus a, vulputate volutpat ante. Donec neque metus, volutpat quis congue non, aliquam sed nunc. Curabitur erat mauris, elementum id rhoncus quis, condimentum eu felis. Quisque porta gravida velit a congue. Nulla gravida suscipit pulvinar. Sed sed erat ut turpis consequat sagittis. Sed scelerisque, massa ac tincidunt rutrum, libero dolor suscipit lorem, interdum dignissim massa enim a purus. Aliquam porta orci non urna sollicitudin, sed lobortis nibh ullamcorper. Aliquam erat \index{volutpat}. Phasellus ac purus in massa aliquet ultricies non sit amet justo.

Quisque placerat lobortis risus. Vestibulum ante ipsum primis in faucibus orci luctus et ultrices posuere cubilia Curae; Pellentesque eget odio sed \index{lectus} sollicitudin consectetur et ornare libero. Aliquam et ullamcorper arcu. Fusce mollis euismod purus, vitae auctor quam lobortis eu. Nunc mollis, velit eu cursus feugiat, nunc neque pellentesque arcu, a suscipit tellus nunc quis \index{quam}. Cras diam est, fermentum a rutrum sed, pretium eu tortor.

Integer imperdiet, est mattis imperdiet luctus, nunc nisl sodales justo, sit amet dapibus urna mauris sit amet diam. Donec et massa lectus. Cras nec pellentesque odio. Integer porta varius enim vel ornare. Donec nec commodo dui, a aliquet magna. Vestibulum sollicitudin nibh justo, ac mattis nibh volutpat et. Morbi eget condimentum enim, sit amet lobortis ligula. Vivamus nec mauris purus.
}
%
% ----------------------------------------------------------------
%
%	APÊNDICES
%
%	Texto complementar da tese.
%
\tueAPENDICES % Material de suporte
{
	%!TEX root = main.tex
\chapter{Bases Formais}

Lorem ipsum dolor sit amet, consectetur adipiscing elit. Vivamus vitae est vitae risus varius malesuada et eget velit. Morbi tincidunt venenatis tellus, in volutpat ante varius et. Fusce congue maximus velit ac dignissim. Integer hendrerit pharetra libero, at vehicula odio vestibulum eget. Etiam eget fringilla leo, sit amet posuere nisl. Aenean at tincidunt felis. Cras rhoncus mauris libero, a vestibulum \index{risus} faucibus quis. Aenean malesuada vitae nibh ut dapibus. Pellentesque vel blandit odio.

Maecenas massa leo, egestas id augue at, aliquam iaculis leo. Etiam ac lacus tempus, malesuada dolor vel, mattis leo. Duis tortor mi, accumsan vitae ligula eu, luctus accumsan diam. Etiam venenatis elit non magna aliquam eleifend. Phasellus in nunc at arcu iaculis ultrices sed sed ante. Nullam in velit a metus convallis vestibulum a vitae turpis. Proin fringilla dui tempor, ultrices metus nec, lobortis elit. Sed at posuere augue. Phasellus ac massa fringilla, convallis urna nec, aliquet orci. Mauris placerat tellus vel scelerisque tempus. Donec lacinia tincidunt mattis. Donec congue, augue sed ullamcorper placerat, erat nunc vestibulum tellus, vel consequat sem diam in magna. Vivamus ac dolor lacinia magna pharetra maximus. Nulla congue feugiat vehicula. Praesent luctus purus ac justo tempor eleifend.

Nunc eu ex vel ipsum ultrices molestie. In eget sodales turpis. Donec egestas facilisis nulla id feugiat. Duis \index{gravida} lorem quis porttitor interdum. Sed turpis leo, aliquet non metus a, vulputate volutpat ante. Donec neque metus, volutpat quis congue non, aliquam sed nunc. Curabitur erat mauris, elementum id rhoncus quis, condimentum eu felis. Quisque porta gravida velit a congue. Nulla gravida suscipit pulvinar. Sed sed erat ut turpis consequat sagittis. Sed scelerisque, massa ac tincidunt rutrum, libero dolor suscipit lorem, interdum dignissim massa enim a purus. Aliquam porta orci non urna sollicitudin, sed lobortis nibh ullamcorper. Aliquam erat volutpat. Phasellus ac purus in massa aliquet ultricies non sit amet justo.

Quisque placerat lobortis risus. Vestibulum ante ipsum primis in faucibus orci luctus et ultrices posuere cubilia Curae; Pellentesque eget odio sed lectus sollicitudin consectetur et ornare libero. Aliquam et ullamcorper arcu. Fusce mollis euismod purus, vitae auctor quam lobortis eu. Nunc mollis, velit eu cursus feugiat, nunc neque pellentesque arcu, a suscipit tellus nunc quis quam. Cras diam est, fermentum a rutrum sed, pretium eu tortor.

Integer imperdiet, est mattis imperdiet luctus, nunc nisl sodales justo, sit amet dapibus urna mauris sit amet diam. Donec et massa lectus. Cras nec pellentesque odio. Integer porta varius enim vel ornare. Donec nec commodo dui, a aliquet \index{magna}. Vestibulum sollicitudin nibh justo, ac mattis nibh volutpat et. Morbi eget condimentum enim, sit amet lobortis ligula. Vivamus nec mauris purus.
	%!TEX root = main.tex
\chapter{Resultados Empíricos}

Lorem ipsum dolor sit amet, consectetur adipiscing elit. Vivamus vitae est vitae risus varius malesuada et eget velit. Morbi tincidunt venenatis tellus, in volutpat ante varius et. Fusce congue maximus velit ac dignissim. Integer \index{hendrerit} pharetra libero, at vehicula odio vestibulum eget. Etiam eget fringilla leo, sit amet posuere nisl. Aenean at tincidunt felis. Cras rhoncus mauris libero, a vestibulum risus faucibus quis. Aenean malesuada vitae nibh ut dapibus. Pellentesque vel blandit odio.

Maecenas massa leo, egestas id augue at, aliquam iaculis leo. Etiam ac lacus tempus, malesuada dolor vel, mattis leo. Duis tortor mi, accumsan vitae ligula eu, luctus accumsan diam. Etiam venenatis elit non magna aliquam eleifend. Phasellus in nunc at arcu iaculis ultrices sed sed ante. Nullam in velit a metus convallis vestibulum a vitae turpis. Proin fringilla dui tempor, ultrices metus nec, lobortis elit. Sed at posuere augue. Phasellus ac massa fringilla, convallis urna nec, aliquet orci. Mauris placerat tellus vel scelerisque tempus. Donec lacinia tincidunt mattis. Donec congue, augue sed ullamcorper placerat, erat nunc vestibulum tellus, vel consequat sem diam in magna. Vivamus ac dolor lacinia magna pharetra maximus. Nulla congue feugiat vehicula. Praesent luctus purus ac justo tempor eleifend.

Nunc eu ex vel ipsum ultrices molestie. In eget sodales turpis. Donec egestas facilisis nulla id feugiat. Duis gravida lorem quis porttitor interdum. Sed turpis leo, aliquet non metus a, vulputate volutpat ante. Donec neque metus, volutpat quis congue non, aliquam sed nunc. Curabitur erat mauris, elementum id rhoncus quis, condimentum eu felis. Quisque porta gravida velit a congue. Nulla gravida suscipit pulvinar. Sed sed erat ut turpis consequat sagittis. Sed scelerisque, massa ac tincidunt rutrum, libero dolor suscipit lorem, interdum dignissim massa enim a purus. Aliquam porta orci non urna sollicitudin, sed lobortis nibh ullamcorper. Aliquam erat volutpat. Phasellus ac purus in massa aliquet ultricies non sit amet justo.

Quisque placerat lobortis risus. Vestibulum ante ipsum primis in faucibus orci luctus et ultrices posuere cubilia Curae; Pellentesque eget odio sed lectus sollicitudin consectetur et ornare libero. Aliquam et ullamcorper arcu. Fusce mollis euismod purus, vitae auctor quam lobortis eu. Nunc mollis, velit eu cursus feugiat, nunc \index{neque} pellentesque arcu, a suscipit tellus nunc quis quam. Cras diam est, fermentum a rutrum sed, pretium eu tortor.

Integer imperdiet, est mattis imperdiet luctus, nunc nisl \index{sodales} justo, sit amet dapibus urna mauris sit amet diam. Donec et massa lectus. Cras nec pellentesque odio. Integer porta varius enim vel ornare. Donec nec commodo dui, a aliquet magna. Vestibulum sollicitudin nibh justo, ac mattis nibh volutpat et. Morbi eget condimentum enim, sit amet lobortis ligula. Vivamus nec mauris purus.
}
%
% ----------------------------------------------------------------
%
%	BIBLIOGRAFIA
%
%	Por omissão...
%	- usa BibTex
%	- com o estilo "alpha"
%	- consulta o ficheiro "bibliografia.tex"
%	- lista **todas** as obras, mesmo que não referenciadas no texto da tese
%
%\tueBIBLIOGRAFIA{}
%
% ----------------------------------------------------------------
%
%	ÍNDICE REMISSIVO
%
%\tueINDICEREMISSIVO{}
%
% ----------------------------------------------------------------
%
% ================================================================
%
%	Modo ORGANIZAÇÃO DA DISSETAÇÃO COMPLETA.
%
%	Prevê que
%		- a informação sobre título, autor, orientadores, etc está definida acima e que
%		- a obra tem a seguinte estrutura:
%
%			prefácio
%			agradecimentos
%			tabela de conteúdos
%			lista de figuras
%			lista de tabelas
%			lista de acrónimos
%			sumário
%			tradução do sumário
%			------------------------------
%			CONTEÚDO (vários capítulos)
%			APÊNDICES (vários capítulos)
%			------------------------------
%			bibliografia
%			índice remissivo
%
% ================================================================
%
\tueDOCUMENTO
%
% ================================================================
%	Modo CAPA, CONTRA-CAPA e LOMBADAS.
%
%	Prevê que a informação sobre título, autor, orientadores, etc está definida acima.
%
% ================================================================
%
%\tueCAPAS

