%!TEX program = xelatex
%
% ================================================================
%	Tipo de dissertação:
%		escolher entre "doutoramento" ou "mestrado"
%
%	Área científica:
%		escolher entre
%			- "ct" (ciências e tecnologia, final); "ctR" (ciências e tecnologia, rascunho);
%			- "csh" (ciências sociais e humanas, final); "cshR" (ciências sociais e humanas, rascunho);
%			- "artes" (artes, final); "artesR" (artes, rascunhos)
%
% ================================================================
%
\documentclass[mestrado,ctR,12pt]{teseue}
%
%
% ================================================================
%	DOCUMENTO:
%		 
%		Língua, Título, Nome do Candidato, Curso, etc
%		Estrutura
% ================================================================
%
% ----------------------------------------------------------------
%
%	LÍNGUA DA TESE
%
%	Opções atuais:
%	- PT: Português (novo acordo ortográfico)
%	- EN: Inglês
%
\tueLINGUA{EN}
%
% ----------------------------------------------------------------
%
%	TÍTULO DA TESE
%
%	Em Português e Inglês.
%
\tueTITULO
{Identity Management in Healthcare Using Blockchain Technology}
{}
%
% ----------------------------------------------------------------
%
%	SUBTÍTULO DA TESE
%
%	Em Português e Inglês.
%
\tueSUBTITULO
{}
{}
%
% ----------------------------------------------------------------
%
%	CANDIDATO
%
%	Nome completo.
%		
\tueCANDIDATO
{João Pedro Nunes dos Santos}
%
% ----------------------------------------------------------------
%
%	TÍTULO E NOME DO/A ORIENTADOR/A
%
%	Designação oficial e nome do orientador/a.
%	Em geral, "Orientador" ou "Orientadora".
%
\tueORIENTADOR
{Orientador}
{Pedro ... Salgueiro}
%
% ----------------------------------------------------------------
%
%	SEGUNDO ORIENTADOR/A (se aplicável)
%
%	Designação oficial e nome do segundo orientador/a.
%	Em geral, "Co-orientador" ou "Co-orientadora".
%
\tueSEGUNDOORIENTADOR
{Orientadora}
{Vítor ... Nogueira}
%
% ----------------------------------------------------------------
%
%	TERCEIRO ORIENTADOR/A (se aplicável)
%
%	Designação oficial e nome do terceiro orientador/a.
%	Em geral, "Co-orientador" ou "Co-orientadora".
%
%\tueTERCEIROORIENTADOR
%{Co-Orientador}
%{António Inácio Norberto}
%
% ----------------------------------------------------------------
%
%	CURSO
%
%	Nome do curso em que se enquadra esta tese.
%
\tueCURSO
{Engenharia Informática}
%
% ----------------------------------------------------------------
%
%	ESPECIALIDADE (se aplicável)
%
%	Nome da especialidade em que se enquadra esta tese.
%
%\tueESPECIALIDADE
%{Coordenação de Recursos Naturais}
%
% ----------------------------------------------------------------
%
%	DEPARTAMENTO
%
%	Departamento anfitrião do curso.
%
\tueDEPARTAMENTO
{Departamento de Informática}
%
% ----------------------------------------------------------------
%
%	ESCOLA
%
%	Escola a que pertence o departamento.
%
\tueESCOLA
{Escola de Ciências e Tecnologia}
%
% ----------------------------------------------------------------
%
%	PALAVRAS CHAVE
%
%	Data de submissão da tese.
%
\tuePALAVRASCHAVE
{Blockchain, Health, Identity, Big Data}
{Blockchain, Saúde, Identidade, Big Data}
%
% ----------------------------------------------------------------
%
%	DATA
%
%	Data de submissão da tese.
%
\tueDATA
{\today}
%
% ----------------------------------------------------------------
%
%	DEDICATÓRIA
%
\tueDEDICATORIA
{To My Family}
%
% ----------------------------------------------------------------
%
%	PREAMBULO
%
%	Comandos e definições para o LaTeX que devem estar **antes**
%	do texto do documento.
%
\tuePREAMBULOLATEX{
	\usepackage[figureright]{rotating}
}
%
% ----------------------------------------------------------------
%
%	PREAMBULO
%
%	Texto até à página 1. 
%
%	Por omissão os conteúdos estão definidos nos ficheiros
%		- prefacio.tex
%		- agradecimentos.tex
%		- acronimos.tex
%		- sumario.tex
%		- abstract.tex
%
\tuePREAMBULO {
  %!TEX root = main.tex
\chapter*{Acknowledgements}

Lorem ipsum dolor sit amet, consectetur adipiscing elit. Vivamus vitae est vitae risus varius malesuada et eget velit. Morbi tincidunt venenatis tellus, in volutpat ante varius et. Fusce congue maximus velit ac dignissim. Integer hendrerit pharetra libero, at vehicula odio vestibulum eget. Etiam eget fringilla leo, sit amet posuere nisl. Aenean at tincidunt felis. Cras rhoncus mauris libero, a vestibulum risus faucibus quis. Aenean malesuada vitae nibh ut dapibus. Pellentesque vel blandit odio.

Maecenas massa leo, egestas id augue at, aliquam iaculis leo. Etiam ac lacus tempus, malesuada dolor vel, mattis leo. Duis tortor mi, accumsan vitae ligula eu, luctus accumsan diam. Etiam venenatis elit non magna aliquam eleifend. Phasellus in nunc at arcu iaculis ultrices sed sed ante. Nullam in velit a metus convallis vestibulum a vitae turpis. Proin fringilla dui tempor, ultrices metus nec, lobortis elit. Sed at posuere augue. Phasellus ac massa fringilla, convallis urna nec, aliquet orci. Mauris placerat tellus vel scelerisque tempus. Donec lacinia tincidunt mattis. Donec congue, augue sed ullamcorper placerat, erat nunc vestibulum tellus, vel consequat sem diam in magna. Vivamus ac dolor lacinia magna pharetra maximus. Nulla congue feugiat vehicula. Praesent luctus purus ac justo tempor eleifend.
  %!TEX root = main.tex
\begin{tueACRONIMOS}
	\begin{acronym}[IEEE]
		\acro{EHR}{\emph{Electronic Health Record}}
		\acro{HL7}{\emph{Health Level 7}}
		\acro{DDOS}{\emph{Distributed Denial of Service}}
		\acro{EU}{\emph{European Union}}
		\acro{BFT}{\emph{Byzantine Fault Tolerant}}
		\acro{GDPR}{\emph{General Data Protection Regulation}}
		\acro{DLP}{\emph{Distributed Ledger Platform}}
		\acro{EVM}{\emph{Ethereum Virtual Machine}}
		\acro{SDK}{\emph{Software Development Kit}}
		\acro{MSP}{\emph{Membership Service Provider}}
		\acro{HLF}{\emph{Hyperledger Fabric}}
		\acro{CA}{\emph{Certificate Authority}}
		\acro{FHIR}{\emph{Fast Healthcare Interoperability Resources}}
		\acro{API}{\emph{Application Programming Interface}}
		\acro{JSON}{\emph{JavaScript Object Notation}}
		\acro{FHIR}{\emph{Fast Healthcare Interoperability Resources}}
		\acro{CIA}{\emph{Confidentiality, Integrity, and Availability}}
		\acro{SHA}{\emph{Secure Hash Algorithm}}
		\acro{TLS}{\emph{Transport Layer Security}}
	\end{acronym}
\end{tueACRONIMOS}

  \begin{tueSUMARIO}

  A criptomoeda Bitcoin foi essencial para criar uma solução para transacções
  digitais seguras, sem requerer a participação de um terceiro interveniente
  fidedigno para ambas as partes.  Para resolver este problema, os mecanismos
  que hoje são associados com a tecnologia Blockchain, foram concebidos e
  implementados para servir como base para a rede da Bitcoin. Mais
  especificamente, esta foi utilizada como um mecanismo de segurança, de forma
  a tornar a Bitcoin uma forma de dinheiro mais transparente e estável, uma
  moeda criptográfica. Mesmo que a Bitcoin não tenha conseguido cumprir o seu
  propósito original, a tecnologia Blockchain despoletou novas inovações e
  permitiu maior criatividade.

  A Blockchain tem sido, desde então, a base tecnológica de várias
  criptomoedas. Algumas implementações desta tecnologia permitem a execução de
  código de uma forma autónoma exactamente como foi programado, sem intervenção
  humana.  Habitualmente chamados \textit{smart contracts}, estes podem ser
  usados para resolver um novo conjunto de problemas devido ao seu
  comportamento transparente, ausência de intervenção humana e devido à sua
  natureza distribuida. 

  A Blockchain é uma tecnologia que permite a criação de sistemas que
  introduzem um conjunto de beneficios em relação aos sistemas tradicionais de
  armazenamento de dados utilizados nos serviços de saúde. Custos e riscos
  associados a estes sistemas podem ser reduzidos e a informação pode ser mais
  transparente e fidedigna para todos os participantes.

  A rede Hyperledger Fabric com transacções privadas e mecanismos avançados de
  segurança foi usada como base para a criação do sistema proposto nesta
  dissertação. Adicionalmente, uma aplicação foi criada que usa \textit{smart
  contracts} para manipular o \textit{ledger} da Blockchain.

  O trabalho apresentado nesta dissertação mostra que um sistema baseado em
  Blockchain, neste caso em Hyperledger Fabric, é adequado a gerir a identidade
  de utentes,  em organizações prestadoras de cuidados de saúde. Apesar das
  funcionalidades apresentadas por esta plataforma serem focadas em privacidade
  e segurança, algumas medidas adicionais em torno da confidencialidade dos
  dados tiveram de ser implementadas. Independentemente disso, o sistema foi
  construido com sucesso e conseguiu cumprir os requerimentos que foram
  definidos. A implementação deste sistema em serviços de saúde traria
  tranparência, imutabilidade e segurança adicional para utentes e
  profissionais de saúde.

\end{tueSUMARIO}

  \begin{tueABSTRACT}

  Bitcoin served as the catalyst for creating a solution to secure digital
  transactions without requiring a trusted third party to be involved. To solve
  this problem, the mechanisms now associated with a Blockchain were
  conceptualized and implemented to serve as the backbone for the Bitcoin
  network. More specifically it was used as a security tool to try and make
  Bitcoin a more transparent and reliable form of cash, a digital currency.
  Even though today, it is clear that it currently, cannot fulfill its intended
  original purpose, it nonetheless, enabled further avenues for innovation and
  creativity used to solve both new sets of problems as well as old problems in
  a new way.

  Blockchain was initially used as the backbone for various cryptocurrencies
  networks and was eventually extended to provide a platform that allows the
  execution of code in an autonomous manner exactly as it was programmed, with
  no human intervention. These smart contracts can be used to solve yet another
  set of problems due to their transparent behaviour, lack of human
  intervention and distributed nature. 

  Blockchain technology allows the creation of systems that introduce a number
  of benefits over traditional data handling used in today's Healthcare
  Information Systems. Costs and risks associated with these systems can be
  reduced and information can become transparent and trustworthy to all
  participants. In this article the technological foundations that enable this
  change are explored and analysed. The Hyperledger Fabric Network with true
  private transactions and advanced security mechanisms was used to serve as
  the basis for this system. An application was created that uses smart
  contracts to manipulate the ledger. In this paper we present this system and
  its impact in Healthcare.

\end{tueABSTRACT}

}
%
% ----------------------------------------------------------------
%
%	CONTEÚDO
%
%	Texto principal da tese.
%
\tueCONTEUDO  % A partir da página 1
{
	%!TEX root = main.tex
\chapter{Introdução}

Health is becoming more digital thanks to the widespread availability of
computing devices.  More and more medical records are stored on a digital
format.  For storing patient clinical data and their identity in a medical
context, the Electronic Health Record (EHR) was created.
 
While all this information should benefit both patient and health professionals
alike, it is not being handled in an effective manner due to problems caused,
in part, due to the fragmentation of the patients identity that naturally
occurs in today's Health Information Systems.

Health is an important topic, for everyone. Healthcare should strive to provide
the best service it can for everyone and everyone should have access to a
quality service. EHR are being generated at an ever increasing rate but most of
the data is not used in a way that puts the patient's privacy and trust at the
forefront.

The purpose of the work presented in this paper is to create and implement a
Blockchain based system for Identity Management in the Healthcare domain. The
patient will be able to manage his data and control its access. Such a system
would be suited to handle the patient’s identity, for example, in hospitals or
clinics and would be able to solve many problems in how data is traditionally
handled in the Information Systems (IS) available in a regular medical
environment.

Blockchain is known as the technology behind the Bitcoin Cryptocurrency,
altough nowadays it is being used for many more purposes that are explored in
the following sections, and its main design goal is to provide security and
immutability to an agreed upon list of records.

A Blockchain runs on a network of computers and the list of records is
replicated in some manner depending on the Blockchain implementation. The first
Blockchain was conceptualized as the public ledger for the Bitcoin
cryptocurrency in 2008 by Satoshi Nakamoto, a pen name of, a still unknown to
this day, individual or organization of individuals.  The network was
implemented in 2009 and many are now finding it has a much broader potential
across many fields, with some implementations even resembling a programming
platform to execute code in an autonomous manner.  \cite{Nakamoto2008}

A single universal way to identify a person in a given environment is clearly
something we should strive towards as seen in, for example, the \textit{Cartão
do Cidadão}, a portuguese identification document that replaces four other
identification documents, streamlining portuguese civilian identification.
This also allows many businesses to tailor their services to this document
making it easier on both parts and eliminating unnecessary costs and risks.

Electronic Health Records (EHR) have seen some progress made regarding the
standards that allow for interoperability between different organizations
thanks to the Health Level 7 (HL7) standard.  While this standard is growing in
use and is represented internationally, Portugal has just started the work
required to implement it.  \cite{HealthLevel7}

In an effort to make the identity of a patient more secure and transparent a
Blockchain can be used to create a system that puts at the forefront of its
design the patients, breaking conventions in traditional patient data handling.

In this article different Blockchain implementations are explored and related
work in this field is presented.  More precisely, in Section~\ref{background},
a brief introduction to Blockchain is made followed by an introduction to its
most prominent implementations. Then a number of real-world use cases of this
technology in the healthcare field are explored. In Section~\ref{HLFHealthcare}
technical details of the system will be presented.  Finally, in
Section~\ref{conclusion},  some conclusions are observed regarding the change
enabled by these advances.

\section{Motivação}

Suspendisse ac dui et urna faucibus consequat. Integer porta vulputate lorem quis condimentum. Vestibulum ut tristique elit. Lorem ipsum dolor sit amet, consectetur adipiscing elit. In in orci id dolor tristique laoreet ut at mauris. Suspendisse mollis leo nulla, nec vehicula mi ultricies sit amet. Nullam nulla purus, blandit et dui a, rhoncus vehicula diam. Aenean sagittis lorem in nunc cursus luctus. Aliquam condimentum ipsum volutpat purus mollis tincidunt. Donec non erat orci. Donec tincidunt sit amet libero ac vulputate. Pellentesque vitae ex vitae odio sollicitudin vestibulum. Ut hendrerit placerat sagittis. Phasellus posuere a felis id feugiat. Maecenas pulvinar, sapien sit amet maximus facilisis, metus arcu pretium nisi, ut consectetur eros nisl eget magna.

\begin{figure}
	\begin{center}
		\begin{tikzpicture}
			%
			\draw[step = 0.5, color = uegray!50!white] (0,0) grid (3.0,3.0); 
			\draw[->,thick] (0,0) -- (3.25,0);
			\draw[->,thick] (0,0) -- (0,3.25);
			%
			\coordinate (A) at (0.0,2.0);
			\coordinate (B) at (0.5,1.5);
			\coordinate (C) at (1.0,0.0);
			\coordinate (D) at (1.5,1.0);
			\coordinate (E) at (2.0,2.0);
			\coordinate (F) at (2.5,2.5);
			%
			\node at (A) {$\circ$};
			\node at (B) {$\circ$};
			\node at (C) {$\circ$};
			\node at (D) {$\circ$};
			\node at (E) {$\circ$};
			\node at (F) {$\circ$};
			%
			\draw[very thick, color = uered] (A) -- (B) -- (C) -- (D) -- (E) -- (F);
			%
		\end{tikzpicture}
	\end{center}
	\caption{Exemplo de uma figura.}
\end{figure}

Quisque fringilla dictum \index{tellus} sed faucibus. Vestibulum eget augue pellentesque, rutrum turpis ac, efficitur nunc. Integer molestie, erat at vehicula commodo, orci nunc consequat diam, fermentum finibus dui ante sit amet eros. Donec venenatis, erat sit amet vestibulum rutrum, neque metus ultrices enim, vel aliquam tellus tortor ut magna. Nullam vitae dolor ipsum. Aliquam erat volutpat. Nulla sagittis elit vel felis tempor feugiat. In ultrices mattis risus, in \index{ornare} lectus faucibus et. In hac habitasse platea dictumst. Praesent pretium aliquam tincidunt. Cras pellentesque lectus ipsum, id fermentum nulla lobortis eu. Fusce tristique dui a \index{diam} semper, vitae dapibus urna ullamcorper. Sed tincidunt elit cursus imperdiet ultricies.

Etiam eu leo diam. Nam \index{imperdiet} neque maximus, pharetra mauris nec, euismod odio. Pellentesque pretium vehicula elit at tincidunt. Nam posuere suscipit dapibus. Nulla aliquam venenatis mi non porta. Nam euismod quam id faucibus sodales. Aenean scelerisque mi et est condimentum eleifend. Duis finibus, lectus quis iaculis lobortis, mauris augue finibus justo, ut semper sem ligula nec risus. Pellentesque vel fermentum felis. Integer vehicula tristique enim id porta. Proin imperdiet tristique libero, a ultricies sem imperdiet a. Ut iaculis facilisis libero sed dictum. Nulla facilisi. Quisque volutpat neque sit amet arcu condimentum, ut efficitur urna consectetur. Aenean ultricies ante sapien, quis pharetra erat elementum faucibus. Nulla lobortis tristique dolor at posuere.

\subsection{Oportunidades}

Curabitur \index{auctor} ante sit amet velit placerat, et vulputate arcu cursus. Curabitur eu molestie tellus, ac euismod felis. Morbi scelerisque justo non eleifend scelerisque. Integer pellentesque massa ac tortor porta, ac commodo lorem aliquet. Donec a metus a libero accumsan ultricies. Integer semper elementum lorem vel \index{blandit}. Vivamus tristique turpis lectus, id rutrum eros interdum eu. Proin facilisis sodales eros eu porttitor. Nunc tristique vehicula risus, eu dapibus lacus suscipit ut. Nunc nec purus at orci aliquet lacinia.

\begin{table}
	\caption{Exemplo de uma tabela.}
	\begin{center}
		\begin{tabular}{lcc|lcc}
			\textbf{Astro} & \textbf{Dia} & \textbf{Ano} & \textbf{Astro} & \textbf{Dia} & \textbf{Ano}\\
			\hline
			Sol & -- & -- & Mercúrio  & 0.5 & 20 \\
			Vénus & 0.75 & 30 & Terra & 1 & 37 \\
			Marte & 1.5 & 45 & Lua & 3 & 3 
		\end{tabular}
	\end{center}
\end{table}

Ut rhoncus tellus nec aliquam iaculis. Aenean consectetur diam id nunc facilisis porta. Duis euismod est id risus feugiat, eu aliquam nisl imperdiet. Donec id enim feugiat, consectetur nisl in, iaculis felis. Ut et mattis elit. Mauris hendrerit, velit sit amet tristique bibendum, est nisi pretium felis, id maximus eros lorem eu sapien. Aliquam sollicitudin eros magna, et congue orci aliquam sed. 

	\include{cap2}
	\include{demo-cap-3}
}
%
% ----------------------------------------------------------------
%
%	APÊNDICES
%
%	Texto complementar da tese.
%
\tueAPENDICES % Material de suporte
{
	\include{demo-apend-1}
	\include{demo-apend-2}
}
%
% ----------------------------------------------------------------
%
%	BIBLIOGRAFIA
%
%	Por omissão...
%	- usa BibTex
%	- com o estilo "alpha"
%	- consulta o ficheiro "bibliografia.tex"
%	- lista **todas** as obras, mesmo que não referenciadas no texto da tese
%
%\tueBIBLIOGRAFIA{}
%
% ----------------------------------------------------------------
%
%	ÍNDICE REMISSIVO
%
%\tueINDICEREMISSIVO{}
%
% ----------------------------------------------------------------
%
% ================================================================
%
%	Modo ORGANIZAÇÃO DA DISSETAÇÃO COMPLETA.
%
%	Prevê que
%		- a informação sobre título, autor, orientadores, etc está definida acima e que
%		- a obra tem a seguinte estrutura:
%
%			prefácio
%			agradecimentos
%			tabela de conteúdos
%			lista de figuras
%			lista de tabelas
%			lista de acrónimos
%			sumário
%			tradução do sumário
%			------------------------------
%			CONTEÚDO (vários capítulos)
%			APÊNDICES (vários capítulos)
%			------------------------------
%			bibliografia
%			índice remissivo
%
% ================================================================
%
\tueDOCUMENTO
%
% ================================================================
%	Modo CAPA, CONTRA-CAPA e LOMBADAS.
%
%	Prevê que a informação sobre título, autor, orientadores, etc está definida acima.
%
% ================================================================
%
%\tueCAPAS

