%!TEX program = xelatex
%
% ================================================================
%	Tipo de dissertação:
%		escolher entre "doutoramento" ou "mestrado"
%
%	Área científica:
%		escolher entre
%			- "ct" (ciências e tecnologia, final); "ctR" (ciências e tecnologia, rascunho);
%			- "csh" (ciências sociais e humanas, final); "cshR" (ciências sociais e humanas, rascunho);
%			- "artes" (artes, final); "artesR" (artes, rascunhos)
%
% ================================================================
%
\documentclass[mestrado,ctR,12pt]{teseue}
%
%
% ================================================================
%	DOCUMENTO:
%		 
%		Língua, Título, Nome do Candidato, Curso, etc
%		Estrutura
% ================================================================
%
% ----------------------------------------------------------------
%
%	LÍNGUA DA TESE
%
%	Opções atuais:
%	- PT: Português (novo acordo ortográfico)
%	- EN: Inglês
%
\tueLINGUA{EN}
%
% ----------------------------------------------------------------
%
%	TÍTULO DA TESE
%
%	Em Português e Inglês.
%
\tueTITULO
{Identity Management in Healthcare Using Blockchain Technology}
{}
%
% ----------------------------------------------------------------
%
%	SUBTÍTULO DA TESE
%
%	Em Português e Inglês.
%
\tueSUBTITULO
{}
{}
%
% ----------------------------------------------------------------
%
%	CANDIDATO
%
%	Nome completo.
%		
\tueCANDIDATO
{João Pedro Nunes dos Santos}
%
% ----------------------------------------------------------------
%
%	TÍTULO E NOME DO/A ORIENTADOR/A
%
%	Designação oficial e nome do orientador/a.
%	Em geral, "Orientador" ou "Orientadora".
%
\tueORIENTADOR
{Orientador}
{Pedro Salgueiro}
%
% ----------------------------------------------------------------
%
%	SEGUNDO ORIENTADOR/A (se aplicável)
%
%	Designação oficial e nome do segundo orientador/a.
%	Em geral, "Co-orientador" ou "Co-orientadora".
%
\tueSEGUNDOORIENTADOR
{Orientadora}
{Vítor Beires Nogueira}
%
% ----------------------------------------------------------------
%
%	TERCEIRO ORIENTADOR/A (se aplicável)
%
%	Designação oficial e nome do terceiro orientador/a.
%	Em geral, "Co-orientador" ou "Co-orientadora".
%
%\tueTERCEIROORIENTADOR
%{Co-Orientador}
%{António Inácio Norberto}
%
% ----------------------------------------------------------------
%
%	CURSO
%
%	Nome do curso em que se enquadra esta tese.
%
\tueCURSO
{Engenharia Informática}
%
% ----------------------------------------------------------------
%
%	ESPECIALIDADE (se aplicável)
%
%	Nome da especialidade em que se enquadra esta tese.
%
%\tueESPECIALIDADE
%{Coordenação de Recursos Naturais}
%
% ----------------------------------------------------------------
%
%	DEPARTAMENTO
%
%	Departamento anfitrião do curso.
%
\tueDEPARTAMENTO
{Departamento de Informática}
%
% ----------------------------------------------------------------
%
%	ESCOLA
%
%	Escola a que pertence o departamento.
%
\tueESCOLA
{Escola de Ciências e Tecnologia}
%
% ----------------------------------------------------------------
%
%	PALAVRAS CHAVE
%
%	Data de submissão da tese.
%
\tuePALAVRASCHAVE
{Blockchain, Saúde, Identidade, Big Data}
{Blockchain, Health, Identity, Big Data}
%
% ----------------------------------------------------------------
%
%	DATA
%
%	Data de submissão da tese.
%
\tueDATA
{\today}
%
% ----------------------------------------------------------------
%
%	DEDICATÓRIA
%
\tueDEDICATORIA
{To My Family}
%
% ----------------------------------------------------------------
%
%	PREAMBULO
%
%	Comandos e definições para o LaTeX que devem estar **antes**
%	do texto do documento.
%
\tuePREAMBULOLATEX{
	\usepackage[figureright]{rotating}
}
%
% ----------------------------------------------------------------
%
%	PREAMBULO
%
%	Texto até à página 1. 
%
%	Por omissão os conteúdos estão definidos nos ficheiros
%		- prefacio.tex
%		- agradecimentos.tex
%		- acronimos.tex
%		- sumario.tex
%		- abstract.tex
%
\tuePREAMBULO {
  \chapter*{Acknowledgements}

Firstly, I want to thank my dissertation advisors, Pedro Salgueiro and Vítor
Beires Nogueira, for being patient with me, for their availability and for
their dedication in this project.

I want to thank my family who helped me finish this project, with their
unending support, words of wisdom and for always pushing me to do better.

I also want to thank my work colleagues, who always supported me, helped me
grow as a professional and person, challenged me to improve, and whom I
consider as a second family.

I need to thank my close friends for always believing in me and for always
being available when I needed the most, showing they are true friends.

Lastly, every person who supported me in some manner, I truly am grateful for
your support.

  %!TEX root = main.tex
\begin{tueACRONIMOS}
	\begin{acronym}[IEEE]
		\acro{EHR}{\emph{Electronic Health Record}}
		\acro{HL7}{\emph{Health Level 7}}
		\acro{DDOS}{\emph{Distributed Denial of Service}}
		\acro{BFT}{\emph{Byzantine Fault Tolerant}}
		\acro{GDPR}{\emph{General Data Protection Rule}}
		\acro{DLP}{\emph{Distributed Ledger Platform}}
		\acro{EVM}{\emph{Ethereum Virtual Machine}}
		\acro{SDK}{\emph{Software Development Kit}}
		\acro{MSP}{\emph{Membership Service Provider}}
		\acro{HLF}{\emph{Hyperledger Fabric}}
		\acro{CA}{\emph{Certificate Authority}}
		\acro{FHIR}{\emph{Fast Healthcare Interoperability Resources}}
		\acro{API}{\emph{Application Programming Interface}}
		\acro{JSON}{\emph{JavaScript Object Notation}}
		\acro{FHIR}{\emph{Fast Healthcare Interoperability Resources}}
		\acro{CIA}{\emph{Confidentiality, Integrity, and Availability}}
		\acro{SHA}{\emph{Secure Hash Algorithm}}
		\acro{TLS}{\emph{Transport Layer Security}}
	\end{acronym}
\end{tueACRONIMOS}

  \begin{tueABSTRACT}

  Bitcoin served as the catalyst for creating a solution to secure digital
  transactions without requiring a trusted third party to be involved. To solve
  this problem, the mechanisms now associated with a Blockchain were
  conceptualized and implemented to serve as the backbone for the Bitcoin
  network. More specifically, it was used as a security tool making Bitcoin a
  more transparent and reliable form of cash, a digital cryptographic currency.
  Even tough Bitcoin ended up not fulfilling its intended purpose as a
  currency, the Blockchain technology has enabled further avenues for
  innovation and creativity.

  Blockchain has since been used as the backbone for various cryptocurrencies
  networks. Some implementations of this technology allow the execution of
  code, also known as "smart contracts". Smart contracts are executed in an
  autonomous manner, with no human intervention. These can be used to solve a
  new set of problems due to their transparent behavior, lack of human
  intervention and distributed nature. 

  Blockchain technology allows the creation of systems that introduce a number
  of benefits over traditional data handling used in today's Healthcare
  Information Systems. Costs and risks associated with these systems can be
  reduced and information can become transparent and trustworthy to all
  participants.
  
  The Hyperledger Fabric Network with true private transactions and advanced
  security mechanisms was used to serve as the basis for the system proposed in
  this dissertation. Moreover, a client application was also created that
  interacts with smart contracts to manipulate the ledger.
  
  The work discussed in this dissertation shows that a Blockchain system based
  on Hyperledger Fabric is suitable for managing patients identity, in
  Healthcare. Even tough the feature set of this Blockchain is very focused in
  privacy and security, some additional measures regarding confidentiality of
  data had to be implemented.  Regardless, a system was built successfully that
  met the requirements. The implementation of this system would provide
  transparency, immutability and additional security for patients and medical
  staff alike. 

\end{tueABSTRACT}

  %!TEX root = main.tex
\begin{tueSUMARIO}

  Lorem ipsum dolor sit amet, consectetur adipiscing elit. Vivamus vitae est
  vitae risus varius malesuada et eget velit. Morbi tincidunt venenatis tellus,
  in volutpat ante varius et. Fusce congue maximus velit ac dignissim. Integer
  hendrerit pharetra libero, at vehicula odio vestibulum eget. Etiam eget
  fringilla leo, sit amet posuere nisl. Aenean at tincidunt felis. Cras rhoncus
  mauris libero, a vestibulum risus faucibus quis. Aenean malesuada vitae nibh
  ut dapibus. Pellentesque vel blandit odio.

  Maecenas massa leo, egestas id augue at, aliquam iaculis leo. Etiam ac lacus
  tempus, malesuada dolor vel, mattis leo. Duis tortor mi, accumsan vitae
  ligula eu, luctus accumsan diam. Etiam venenatis elit non magna aliquam
  eleifend. Phasellus in nunc at arcu iaculis ultrices sed sed ante. Nullam in
  velit a metus convallis vestibulum a vitae turpis. Proin fringilla dui
  tempor, ultrices metus nec, lobortis elit. Sed at posuere augue. Phasellus ac
  massa fringilla, convallis urna nec, aliquet orci. Mauris placerat tellus vel
  scelerisque tempus. Donec lacinia tincidunt mattis. Donec congue, augue sed
  ullamcorper placerat, erat nunc vestibulum tellus, vel consequat sem diam in
  magna. Vivamus ac dolor lacinia magna pharetra maximus. Nulla congue feugiat
  vehicula. Praesent luctus purus ac justo tempor eleifend.

  Nunc eu ex vel ipsum ultrices molestie. In eget sodales turpis. Donec egestas
  facilisis nulla id feugiat. Duis gravida lorem quis porttitor interdum. Sed
  turpis leo, aliquet non metus a, vulputate volutpat ante. Donec neque metus,
  volutpat quis congue non, aliquam sed nunc. Curabitur erat mauris, elementum
  id rhoncus quis, condimentum eu felis. Quisque porta gravida velit a congue.
  Nulla gravida suscipit pulvinar. Sed sed erat ut turpis consequat sagittis.
  Sed scelerisque, massa ac tincidunt rutrum, libero dolor suscipit lorem,
  interdum dignissim massa enim a purus. Aliquam porta orci non urna
  sollicitudin, sed lobortis nibh ullamcorper. Aliquam erat volutpat. Phasellus
  ac purus in massa aliquet ultricies non sit amet justo.

  Quisque placerat lobortis risus. Vestibulum ante ipsum primis in faucibus orci
  luctus et ultrices posuere cubilia Curae; Pellentesque eget odio sed lectus
  sollicitudin consectetur et ornare libero. Aliquam et ullamcorper arcu. Fusce
  mollis euismod purus, vitae auctor quam lobortis eu. Nunc mollis, velit eu
  cursus feugiat, nunc neque pellentesque arcu, a suscipit tellus nunc quis
  quam. Cras diam est, fermentum a rutrum sed, pretium eu tortor.

\end{tueSUMARIO}

}
%
% ----------------------------------------------------------------
%
%	CONTEÚDO
%
%	Texto principal da tese.
%
\tueCONTEUDO  % A partir da página 1
{
	\chapter{Introduction}
\label{introduction}

\begin{quote} 
  \emph{This Chapter introduces the main topics and technologies covered by
  this dissertation. Healthcare and its relationship with technology is
  presented. The current flaws associated with patients identity data
  management are described. The Blockchain technology is introduced as a
  potential solution to some of these problems.} 
\end{quote}

The aim of this dissertation is to create a solution for managing the identity
of patients in the Healthcare environment by using Blockchain technology, and
in turn, evaluate the use of this technology in this specific use case.  Health
is intrinsically linked with technology, as new technologies enable safer and
better treatments. Nowadays, Healthcare organizations store patients data on a
digital format. The Electronic Health Record (EHR) is an abstract concept
representing the patients digitally stored clinical data and their identity in
a medical and clinical context.

Standards are an important aspect to take into account when designing an
information system because they allow interoperability between different
organizations. The Health Level 7 (HL7) Fast Healthcare Interoperability
Resources (FHIR) standard (see Section~\ref{blockchainHealthcare}), is being
built primarily by the Health Level Seven organization. Over the last few
years, it has seen a significant growth in usage. It is also an international
standard with partnerships worldwide. HL7 Portugal is now starting its
operations and is building a community to support this standard in
Portugal~\cite{HealthLevel7}.

Blockchain is often known as the technology behind the Bitcoin cryptocurrency.
Bitcoin depends on two complementary technologies, digital tokens and a
Blockchain, that when orchestrated together facilitate trust, immutability and
resiliency~\cite{Evans2016}.

A Blockchain runs on a network of computers and has a list of records that are
replicated across the participating peers. Blockchain, as we know today, was
conceptualized as the public ledger~\footnote{A ledger is defined as an object
in which items are regularly recorded, originally business activities and money
received or paid, but in reality, it can be used to store any type of record.}
for the Bitcoin cryptocurrency in 2008 by Satoshi Nakamoto~\cite{Nakamoto2008}.
Satoshi Nakamoto is a pen name of, a still unknown to this day, individual or
organization of individuals.

Traditional Healthcare databases and architectures are increasingly vulnerable
and a target to groups of malicious actors that possess the technical expertise
to deny services with Distributed Denial of Service (DDOS) attacks~\footnote{A
Distributed Denial of Service attack is an attempt to make an online service
unavailable by overwhelming it with traffic from multiple sources.} or cause a
data breach~\footnote{A data breach is the intentional or unintentional release
of secure or private/confidential information to an untrusted
environment.}~\cite{mcCoy2018}. 

Making matters worse other problems spring to mind. The data that comprises the
identity of a patient is often fragmented across multiple Healthcare
organizations, in such a way that, to get a true overview of the patients
history and diagnosis there would be a need to merge all the pieces of
information stored in data systems that are hosted in architecturally different
Healthcare information systems. Transparency is also a concern, as a patient
does not currently possess the means to track how his medical data is being
handled.

As more information becomes available, new insights can be extracted by
Healthcare professionals that lead to an overall improvement of the patients
interaction with the Healthcare ecosystem. However, maintaining a high amount
of data secure is a costly and risky matter for every party involved. Security
and privacy are a top concern regarding sensitive data. 

This dissertation provides an insight into the design and implementation of a
Blockchain based system for managing the identity of patients in an Healthcare
setting and its subsequent evaluation. The creation of this system and its
subsequent evaluation could provide interesting conclusions to medical staff as
well as patients, regarding its potential implementation and deployment in the
field.

In this document, different Blockchain implementations are explored to get an
overview of their feature set and focus. Considering a set of defined
requirements a platform is chosen, in order to evaluate the suitability of this
technology in the Healthcare field. More precisely, in
Chapter~\ref{background}, a brief introduction to Blockchain and its most
prominent implementations is presented. The technology is further explored in
Chapter~\ref{blockchain} and a number of real world use cases of this
technology in the Healthcare field are explored.  In Chapter~\ref{development}
a Blockchain platform is chosen in order to build a prototype system to
evaluate the usability of this technology in the Healthcare field. Insight is
given into the system design, implementation. In Chapter~\ref{experiments}, the
system is tested and evaluated. Finally, in Chapter~\ref{Conclusion} some
conclusions are presented and potential future work is discussed.

	\chapter{Background}\label{background}


\begin{quote} \emph{"This project aims to build a Blockchain based system to
  manage the identity of patients and investigating the suitability of creating
  such a system in the Healthcare environment according to objective criteria.
  While Blockchain is not a new concept at this point, it is an evolving
  technology that is being used to solve old problems with new approaches while
  at the same time creating new application fields and challenging old
  conventions and methodologies. This Chapter will provide an overview of this
  technology and some of its most prominent implementations. Finally, some
  context is given to how technology has been helping the Healthcare industry
  to enable better management of their patients identity and how the current
  information systems of Healthcare establishments handle this task."}
\end{quote}

\section{Blockchain Technology}

The concept of Blockchain is abstract. It is a collection of technologies
orchestrated to work together. In this sense the concept can be used to refer
to the Bitcoin's Blockchain, alternative implementations or even forks of the
Bitcoin Blockchain called Altchains~\cite{Lewis2015} that share many
characteristics but may have different features and purposes. It can even
refer to platforms that allow execution of code in an autonomous manner,
exactly as it was programmed, with no human intervention.  A Blockchain is,
generally speaking, a continuously growing list of records being written in
the ledger, a structure where all records are written and stored, that is
constantly being replicated across a network of peers, in opposition to
having a single central record history, making it a good example of a
distributed database, thus avoiding having a single central point of failure
that can be easily targetable~\cite{Barclay2017}.

The purpose of a Blockchain is to maintain integrity in a network of
distributed systems~\cite{Drescher2017}. To fulfill this purpose it uses
cryptographic techniques and digital signatures to not only verify the
authenticity of records but also as a way to manage read or write access to
the network and as proof that a record was written in the ledger and was
never tampered with, creating an immutable history of records, that benefits
various use cases as discussed later in this document.

Unlike a conventional database system running in a server, where only a
single entity keeps a copy of the underlying database, making it centralized
by design, the ledger of the Blockchain is constantly replicated across any
number of participating nodes in the network~\cite{Lewis2015} in a regular
cadence defined at the genesis of the network. In some implementations, not
every participant has the same ability to interact with the ledger and in
this respect a Blockchain can be permissionless or permissioned. Generally
speaking, in a permissionless Blockchain every node of the network can write
in the ledger whereas in a permissioned Blockchain only a select group of
entities have access to writing in the ledger, making the permissioned
version, by default secure, if the entities themselves who manage the network
are considered secure and trustworthy by the participants in the
network~\cite{Lewis2015,Valenta2017}.

But then, how does a permissionless Blockchain maintain security if every
participant in the network has access to writing on it, including potentially
malicious parties?

Given that participating nodes in a public network can belong to different
and often competing parties, there is no implied trust between them, so the
Blockchain needs a mechanism to ensure the integrity of the ledger and
prevent malicious meddling from interested parties and avoiding the need for
a central authority~\cite{Barclay2017}.  Take for example the Bitcoin
Blockchain that uses a peer-to-peer network to avoid the requirement of a
third party being involved in a financial transaction such as a financial
institution or a middle man, which must be trusted with the details of a
transaction to see it through~\cite{Nakamoto2008}.

Consensus is a mechanism employed by the Blockchain to solve this problem.
Even though consensus mechanisms can behave vastly different, depending on
its implementation and purpose, they are at the core a solution to create
immutability and ensure resiliency by ensuring the majority of the network
agrees upon the sequence of events.  For example, in the Bitcoin's Blockchain
case, consensus is reached by the longest chain rule where the longest chain
of blocks not only serves as proof of the sequence of events witnessed, but
as proof that it came from the largest pool of computing power, as it uses a
proof of work (\textbf{pow}) algorithm that relies on brute force to solve a
complex mathematical puzzle, making the longest chain of blocks the one with
the most computing power behind it and therefore agreed upon by the majority
of the network~\cite{Baars2016,Wood2017} making it the most likely to be the
one that represents the sequence of events witnessed.

While the Blockchain, we now know today, was conceptualized as the public
ledger for the Bitcoin cryptocurrency in 2008 by Satoshi Nakamoto and
implemented in 2009, many are now using it as a foundation across many
application areas such as traceability and asset management~\cite{MIT2016}.
Thanks to the roaring success of Bitcoin and the increasingly apparent use
cases that the Blockchain can provide, the public and the various industries
interest in this technological advance is rising and it is quickly becoming a
technological foundation in our economic and social systems~\cite{Zago2018,
Marr2018,Long2018}.

\subsection{Ethereum}

Due to Bitcoin getting extensive media coverage, the average public awareness
in cryptocurrencies is shown to be rising~\cite{BitAwareness2017}. While
Blockchain is used as a means to increase the resiliency of the Bitcoin
cryptocurrency from malicious parties, a token is used to represent the coin. 

Just like a Dollar it has no value by itself, it has value only because we
agree to trade goods and services in exchange for a higher amount of the
currency under our control and we believe others will do the same
\cite{aliessi2016}. Through the years Blockchain has evolved to be capable of
being an independent development platform using the token as a means to
reward those who maintain the consensus by spending electricity and
computation power in the network. In some networks like Ethereum one can
build upon the network to create Decentralized Applications (\textbf{Ðapps})
that allow logic to be executed in an autonomous manner~\cite{Wood2017}. 

In the same manner that the Bitcoin Blockchain can be seen as an adding
machine, the Ethereum Blockchain can be seen as a computer able to execute
programs designed for it~\cite{Wood2015}.

Ethereum is an open-source platform based on the Blockchain technology that
enables developers to build and deploy \textbf{Ðapps}. Ethereum is being
developed by the Ethereum Foundation and was first discussed by Buterin in
2013.  Ethereum intends to provide a Blockchain with a built-in programming
language that is used to create \textit{Smart contracts}~\cite{Wood2017}.

These are used to describe the logic of any system that developers can
imagine and, when created, can be deployed to the Blockchain where they
execute as “autonomous agents”.  Thanks to these tools it is safe to say that
long gone are the days where building Blockchain applications required a
complex background in coding cryptography, mathematics as well as significant
resources~\cite{Wood2017,BlockGeeks2017}.

The Ethereum Blockchain is a permissionless Blockchain, and thus, it must
have a consensus mechanism to ensure the validation process of every record
and, in turn, ensure resiliency and immutability. While other implementations
of the Blockchain have different consensus mechanics, in Ethereum’s case, all
participants have to reach consensus over the order of all transactions that
have taken place. If a definitive order cannot be established then a
double-spend~\footnote{Double-spending is a potential flaw in a digital cash
scheme in which the same single digital token can be spent more than once.
This is possible because a digital token consists of a digital file that can
be duplicated or falsified. As with counterfeit money, such double-spending
leads to inflation by creating a new amount of fraudulent currency that did
not previously exist. This devalues the currency relative to other monetary
units, and diminishes user trust as well as the circulation and retention of
the currency.} might have occurred and the transaction is
rejected~\cite{Wood2017}.

\subsection{Fabric}

Hyperledger Fabric (\textbf{hlf}) is part of the Hyperledger project started
in December 2015 by the Linux Foundation, and is an open-source
developer-focused community of communities focused on the development of
enterprise-grade, open-source Blockchain-based solutions.  Fabric is an
implementation of a Distributed Ledger Platform (\textbf{dlp}) under the
Hyperledger umbrella~\cite{Cachin2016}.

Hyperledger Fabric’s initial commit was contributed by IBM and written in the
Go programming language.  It is a permissioned Blockchain and its main design
goal was to surpass previous Blockchain implementation limitations, such as,
lack of true private transactions and confidential contracts.

These goals are achieved thanks to assigning peers in the network three
distinct roles and by offering the ability to create channels each with its
own private ledger.  A peer can have the role of endorser, committer or
consenter or sometimes multiple roles.  Hyperledger Fabric is intended as a
foundation for developing applications in a modular fashion, opting for a
plug-and-play approach to its various components as well as its consensus
mechanism~\cite{HyperledgerFabricDocs2017}.

Hyperledger Fabric, as discussed, also allows the creation of smart contracts
which can be written in Chaincode.  Given that this Blockchain's key
operational requirement is privacy, featuring true private transactions and
confidential contracts, it makes this technology a great asset for a business
environment where sensitive information must be handled with care and
disclosed on a case by case basis.  Thanks to its modular approach consensus
protocols are no longer hard-coded and trust models can be repurposed.

\subsection{Burrow}

Hyperledger Burrow (\textbf{hlb}) is also part of the Hyperledger project and
its development started in 2014 by Monax and sponsored by Intel. It is a
permissionable smart contract machine written in Go and offers a modular
Blockchain client with a permissioned smart contract interpreter built, in
part, to the specification of the Ethereum Virtual Machine (\textbf{evm})
with the client having, essentially, three main components, the consensus
engine, the permissioned \textbf{evm} and the Remote Procedure Call
(\textbf{rpc}) gateway~\cite{Kuhlman2017,HyperledgerBurrow2017}.

Hyperledger Burrow has its own Consensus Engine, the Byzantine fault-tolerant
Tendermint protocol.  The Tendermint protocol is an open-source effort that
allows high performance in solving the consensus problem and also has a
flexible interface for building arbitrary applications above the consensus,
as well as, a suite of tools for deployments and their
management~\cite{Buchman2016}.

\section{Identity in Healthcare}

Originally records of a patient were stored in paper, a physical format.
Thanks to the advent of the computers more and more records are stored on a
digital format and the Electronic Health Record (EHR) was created.
\cite{Marquez2017}  This benefits handling of information between the patient
and the medical professionals and medical
institutions.\cite{ONCoordinator2017} But first we must discuss what is
defined as identity in this specific case.

Identity is a construct that depends on the context.  Identity can be defined
as the characteristics determining who or what a person is.  In this paper we
define identity as the set of characteristics that determine who is the
patient in the given Healthcare ecosystem they belong to, such as, the name,
the age, the cellphone number, the gender and the birth date of the patient.
Electronic Health Records encapsulate this information in digital format,
however, they are usually represented in a format according to the
Information System they were designed to work with.

To enable interoperability, standards for EHRs were created and many failed
to bring the much needed consensus that was required for interoperability
between different Information Systems in different institutions.
\cite{Eichelberg2006} Health Level 7 has done much work to be recognized in
many countries and is quickly being implemented in many countries to allow
for joint efforts between organizations. \cite{HL7Anual2016}

Even with these advances in mind, the nature of many clinics and hospitals
Information Systems makes the management of their patients identity a very
cumbersome, costly and risky affair to handle.  Security in a connected age,
where internet is easily available, is lagging behind and presenting some
problems.  There is also the question of transparent use of information by
the organizations that store it.

	\section{Identity in Healthcare}

Originally records of a patient were stored in paper, a physical format.
Thanks to the advent of the computers more and more records are stored on a
digital format and the Electronic Health Record (EHR) was created.
\cite{Marquez2017}  This benefits handling of information between the patient
and the medical professionals and medical institutions.\cite{ONCoordinator2017}
But first we must discuss what is defined as identity in this specific case.

Identity is a construct that depends on the context.  Identity can be defined
as the characteristics determining who or what a person is.  In this paper we
define identity as the set of characteristics that determine who is the patient
in the given Healthcare ecosystem they belong to, such as, the name, the age,
the cellphone number, the gender and the birth date of the patient.  Electronic
Health Records encapsulate this information in digital format, however, they
are usually represented in a format according to the Information System they
were designed to work with.

To enable interoperability, standards for EHRs were created and many failed to
bring the much needed consensus that was required for interoperability between
different Information Systems in different institutions. \cite{Eichelberg2006}
Health Level 7 has done much work to be recognized in many countries and is
quickly being implemented in many countries to allow for joint efforts between
organizations. \cite{HL7Anual2016}

Even with these advances in mind, the nature of many clinics and hospitals
Information Systems makes the management of their patients identity a very
cumbersome, costly and risky affair to handle.  Security in a connected age,
where internet is easily available, is lagging behind and presenting some
problems.  There is also the question of transparent use of information by the
organizations that store it.

	\section{Blockchain for Identity Management in Healthcare: Use Cases}

Some companies have already started developing Blockchain applications in the
Healthcare field and established some key partnerships.

Many Blockchain-based solutions are still very early on development or
deployment.  One exception is Guardtime, that has fully deployed their system
in 2008, started cooperating in 2011 and in 2016 announced a partnership with
the Estonian Government, where a million patient records are now secured by the
strategy and, until today, still proves the resilience of the Blockchain
technology, as well as, other advances in cryptography.  Now other companies
like Verizon are becoming interested in this technology for their own purposes.
\cite{GuardTime2018,EstonianGovernmentGuardTime2016}

Another company, Gem, is collaborating with Phillips Healthcare to explore
options in this area, and is opting to solve the interoperability problem with
an additional layer of abstraction they call GemOS.  Factom, another
Blockchain-based service, has also announced a partnership with a major US
medical services provider
HealthNautica.\cite{BlockchainCompHealth2017,FactomPartnership2017}

The use of the Blockchain technology in the health field is expanding. Just
recently a new platform appeared, called Medichain that allows patients to
store their own data in a secure way and give anonymized access to this data to
specialists. Giving data allows for users to gain tokens that represent value.
\cite{MediChain2018}

}
%
% ----------------------------------------------------------------
%
%	APÊNDICES
%
%	Texto complementar da tese.
%
\tueAPENDICES % Material de suporte
{
	%!TEX root = main.tex
\chapter{Bases Formais}

Lorem ipsum dolor sit amet, consectetur adipiscing elit. Vivamus vitae est vitae risus varius malesuada et eget velit. Morbi tincidunt venenatis tellus, in volutpat ante varius et. Fusce congue maximus velit ac dignissim. Integer hendrerit pharetra libero, at vehicula odio vestibulum eget. Etiam eget fringilla leo, sit amet posuere nisl. Aenean at tincidunt felis. Cras rhoncus mauris libero, a vestibulum \index{risus} faucibus quis. Aenean malesuada vitae nibh ut dapibus. Pellentesque vel blandit odio.

Maecenas massa leo, egestas id augue at, aliquam iaculis leo. Etiam ac lacus tempus, malesuada dolor vel, mattis leo. Duis tortor mi, accumsan vitae ligula eu, luctus accumsan diam. Etiam venenatis elit non magna aliquam eleifend. Phasellus in nunc at arcu iaculis ultrices sed sed ante. Nullam in velit a metus convallis vestibulum a vitae turpis. Proin fringilla dui tempor, ultrices metus nec, lobortis elit. Sed at posuere augue. Phasellus ac massa fringilla, convallis urna nec, aliquet orci. Mauris placerat tellus vel scelerisque tempus. Donec lacinia tincidunt mattis. Donec congue, augue sed ullamcorper placerat, erat nunc vestibulum tellus, vel consequat sem diam in magna. Vivamus ac dolor lacinia magna pharetra maximus. Nulla congue feugiat vehicula. Praesent luctus purus ac justo tempor eleifend.

Nunc eu ex vel ipsum ultrices molestie. In eget sodales turpis. Donec egestas facilisis nulla id feugiat. Duis \index{gravida} lorem quis porttitor interdum. Sed turpis leo, aliquet non metus a, vulputate volutpat ante. Donec neque metus, volutpat quis congue non, aliquam sed nunc. Curabitur erat mauris, elementum id rhoncus quis, condimentum eu felis. Quisque porta gravida velit a congue. Nulla gravida suscipit pulvinar. Sed sed erat ut turpis consequat sagittis. Sed scelerisque, massa ac tincidunt rutrum, libero dolor suscipit lorem, interdum dignissim massa enim a purus. Aliquam porta orci non urna sollicitudin, sed lobortis nibh ullamcorper. Aliquam erat volutpat. Phasellus ac purus in massa aliquet ultricies non sit amet justo.

Quisque placerat lobortis risus. Vestibulum ante ipsum primis in faucibus orci luctus et ultrices posuere cubilia Curae; Pellentesque eget odio sed lectus sollicitudin consectetur et ornare libero. Aliquam et ullamcorper arcu. Fusce mollis euismod purus, vitae auctor quam lobortis eu. Nunc mollis, velit eu cursus feugiat, nunc neque pellentesque arcu, a suscipit tellus nunc quis quam. Cras diam est, fermentum a rutrum sed, pretium eu tortor.

Integer imperdiet, est mattis imperdiet luctus, nunc nisl sodales justo, sit amet dapibus urna mauris sit amet diam. Donec et massa lectus. Cras nec pellentesque odio. Integer porta varius enim vel ornare. Donec nec commodo dui, a aliquet \index{magna}. Vestibulum sollicitudin nibh justo, ac mattis nibh volutpat et. Morbi eget condimentum enim, sit amet lobortis ligula. Vivamus nec mauris purus.
	%!TEX root = main.tex
\chapter{Resultados Empíricos}

Lorem ipsum dolor sit amet, consectetur adipiscing elit. Vivamus vitae est vitae risus varius malesuada et eget velit. Morbi tincidunt venenatis tellus, in volutpat ante varius et. Fusce congue maximus velit ac dignissim. Integer \index{hendrerit} pharetra libero, at vehicula odio vestibulum eget. Etiam eget fringilla leo, sit amet posuere nisl. Aenean at tincidunt felis. Cras rhoncus mauris libero, a vestibulum risus faucibus quis. Aenean malesuada vitae nibh ut dapibus. Pellentesque vel blandit odio.

Maecenas massa leo, egestas id augue at, aliquam iaculis leo. Etiam ac lacus tempus, malesuada dolor vel, mattis leo. Duis tortor mi, accumsan vitae ligula eu, luctus accumsan diam. Etiam venenatis elit non magna aliquam eleifend. Phasellus in nunc at arcu iaculis ultrices sed sed ante. Nullam in velit a metus convallis vestibulum a vitae turpis. Proin fringilla dui tempor, ultrices metus nec, lobortis elit. Sed at posuere augue. Phasellus ac massa fringilla, convallis urna nec, aliquet orci. Mauris placerat tellus vel scelerisque tempus. Donec lacinia tincidunt mattis. Donec congue, augue sed ullamcorper placerat, erat nunc vestibulum tellus, vel consequat sem diam in magna. Vivamus ac dolor lacinia magna pharetra maximus. Nulla congue feugiat vehicula. Praesent luctus purus ac justo tempor eleifend.

Nunc eu ex vel ipsum ultrices molestie. In eget sodales turpis. Donec egestas facilisis nulla id feugiat. Duis gravida lorem quis porttitor interdum. Sed turpis leo, aliquet non metus a, vulputate volutpat ante. Donec neque metus, volutpat quis congue non, aliquam sed nunc. Curabitur erat mauris, elementum id rhoncus quis, condimentum eu felis. Quisque porta gravida velit a congue. Nulla gravida suscipit pulvinar. Sed sed erat ut turpis consequat sagittis. Sed scelerisque, massa ac tincidunt rutrum, libero dolor suscipit lorem, interdum dignissim massa enim a purus. Aliquam porta orci non urna sollicitudin, sed lobortis nibh ullamcorper. Aliquam erat volutpat. Phasellus ac purus in massa aliquet ultricies non sit amet justo.

Quisque placerat lobortis risus. Vestibulum ante ipsum primis in faucibus orci luctus et ultrices posuere cubilia Curae; Pellentesque eget odio sed lectus sollicitudin consectetur et ornare libero. Aliquam et ullamcorper arcu. Fusce mollis euismod purus, vitae auctor quam lobortis eu. Nunc mollis, velit eu cursus feugiat, nunc \index{neque} pellentesque arcu, a suscipit tellus nunc quis quam. Cras diam est, fermentum a rutrum sed, pretium eu tortor.

Integer imperdiet, est mattis imperdiet luctus, nunc nisl \index{sodales} justo, sit amet dapibus urna mauris sit amet diam. Donec et massa lectus. Cras nec pellentesque odio. Integer porta varius enim vel ornare. Donec nec commodo dui, a aliquet magna. Vestibulum sollicitudin nibh justo, ac mattis nibh volutpat et. Morbi eget condimentum enim, sit amet lobortis ligula. Vivamus nec mauris purus.
}
%
% ----------------------------------------------------------------
%
%	BIBLIOGRAFIA
%
%	Por omissão...
%	- usa BibTex
%	- com o estilo "alpha"
%	- consulta o ficheiro "bibliografia.tex"
%	- lista **todas** as obras, mesmo que não referenciadas no texto da tese
%
%\tueBIBLIOGRAFIA{}
%
% ----------------------------------------------------------------
%
%	ÍNDICE REMISSIVO
%
%\tueINDICEREMISSIVO{}
%
% ----------------------------------------------------------------
%
% ================================================================
%
%	Modo ORGANIZAÇÃO DA DISSETAÇÃO COMPLETA.
%
%	Prevê que
%		- a informação sobre título, autor, orientadores, etc está definida acima e que
%		- a obra tem a seguinte estrutura:
%
%			prefácio
%			agradecimentos
%			tabela de conteúdos
%			lista de figuras
%			lista de tabelas
%			lista de acrónimos
%			sumário
%			tradução do sumário
%			------------------------------
%			CONTEÚDO (vários capítulos)
%			APÊNDICES (vários capítulos)
%			------------------------------
%			bibliografia
%			índice remissivo
%
% ================================================================
%
\tueDOCUMENTO
%
% ================================================================
%	Modo CAPA, CONTRA-CAPA e LOMBADAS.
%
%	Prevê que a informação sobre título, autor, orientadores, etc está definida acima.
%
% ================================================================
%
%\tueCAPAS

