\chapter{Conclusions and Future Work}
\label{Conclusion}

\begin{quote}
\emph{This Chapter presents some remarks about the conclusions observed during
the course of the work described in this dissertation. Also some possible
future work is presented based on the findings shown on this document.}
\end{quote}

In the context of this dissertation a system was built that is capable of
managing patients identity using Blockchain technology. As this is a relatively
new technology, there was the need to create simple system in order to reach
some conclusions. The prototype system leverages this technology in the
Healthcare field. The platform chosen was the Hyperledger Fabric Distributed
Ledger Platform. 

While some difficulties were encountered developing a system in this
technology, ultimately, a system using Hyperledger Fabric was successfully
designed and built and some conclusions were taken into account. The platform
was evaluated against the Confidentiality, Integrity and Availability triad,
the international standard for information security.

\section{Conclusions}

Hyperledger Fabric was one of the first projects that were built under the
Hyperledger umbrella, and it is a general purpose Distributed Ledger Platform.

Using Hyperledger Fabric a system was built that leverages the features of this
platform to manage the identity of patients in a Healthcare environment. This
system was designed and implemented successfully. The system was evaluated
against the CIA triad model and was able to meet the desired requirements.

It was shown that it is possible to create a system for the purpose mentioned
with this technology and that it provides several advantages in relation to
more traditional systems such as data transparency to the patient, immutability
of data and decentralization.

The patient benefits from transparency because he will be able to see his data
everytime he wishes to do so, creating a greater degree of trust with the
Healthcare organizations than what is possible nowadays. Immutability is
important to maintain this trust because if he discovers his data was altered
in any way, then the record of that tampering would be forever recorded and
could be traced back to the malicious actor due to this platform supporting
auditability with the Membership Service Provider that issues client identity
certificates through the built-in CA service.  Decentralization provides
additional resiliency as it avoids having a single point of failure that could
be targeted.

This work described in this dissertation proves that this technology has many
applications in this field, and that it can be used more often as the platform
becomes more mature and complete.

This research is especially important because security must be a key focus of
the Healthcare industry for the next few years, as expertise will increase in
the digital space providing more opportunities for malicious parties to use
potential flaws in the information systems deployed in the Healthcare
organizations.

In the Healthcare field, patients data must be treated with the utmost care
because health information is a sensitive and personal matter for each patient.
The privacy rights of the patients must be respected and, as Healthcare becomes
a more digital industry, technology will need to provide additional means to
help the medical professionals ensure this.

\section{Future Work} \label{futureWork}

There are many approaches that can be taken in future work. Blockchain
technology is certainly interesting and other platforms can be explored to
evaluate their suitability in Healthcare. As this platform matures and new
features are added future work could be built upon the new features added or
new Hyperledger projects such as Hyperledger Indy. 

Hyperledger Indy was a platform that was not available when this dissertation
was initially discussed. The platform is interesting choice because of its
focus on identity. Some research on this platform could be made to map
potential similarities to Fabric in ways that could show their common points
and differences. It could be shown which platform would be more suited for this
task or the different use cases they excel in could be noted.

The prototype project could be expanded with a graphical interface, instead of
the current command line interface, leading to a more intuitive usage of this
system and additional platforms to be targeted naturally. Further optimization
efforts could be made regarding channels, encryption methods, data segregation
and scalability. To improve security the key generated when an entity registers
in the network should refresh periodically, to avoid potential identity theft.

Fabric roles could be explored to distinguish between a doctor and a nurse in
the same Healthcare organization.  This way to access information could be
created using the Role Based Access Control added to Fabric.  Hyperledger
Burrow could be used to improve compatibility between Ethereum and Hyperledger
Fabric and target two platforms with similar smart contract code.  Finally this
concept could be expanded further and Blockchain could serve as a secure access
key store that enables secure transmission of access keys to external protected
servers that store a big amount of data that the Blockchain would not be suited
for.
