\chapter{Conclusions and Future Work}
\label{Conclusion}

\emph{In this chapter a step back is taken and some reflections about the
conclusions observed during the course of this work are noted. A simple working
system was successfully built, meeting requirements set in order to keep a good
compromise between the benefits the Blockchain technology provides and the
regulatory reality that organizations must adhere to. It was proven that this
technology can be applied in this field, providing several advantages in
relation to more traditional systems, such as transparency, immutability and
decentralization. As this platform matures and new features are added future
work could be built upon the new features added or new Hyperledger projects
such as Hyperledger Indy. Finally this concept could be expanded further and
Blockchain could serve as a secure access key store that enables secure
transmission of access keys to external secure servers that store high amounts
of data, a task that Blockchain would not be suited for.}

In this thesis a system was built that is capable of managing patients identity
using Blockchain technology. As this is a relatively new technology some
research was done and some conclusions were reached.

In this work it was shown an overview of the Blockchain technology and its
origins. The evolution of this technology as first introduced by Nakamoto and
used in Bitcoin to solve the need for a middle man and enable trust between
unknown parties to a development platform and wide array of applications it
currently is used was then shown. It was shown that there are two different
implementations of the Blockchain each with its own purpose, with these being
the permissionless and permissioned implementations.

In order to gain some working knowledge a prototype system was deemed necessary
to be built that leverages this technology in the Healthcare field. The
platform chosen was the Hyperledger Fabric Distributed Ledger Platform. After
some difficulties a system was successfully designed and built and some
conclusions were taken into account. The platform was evaluated against the CIA
triad, the international standard for information security.

\section{Conclusions}

Blockchain was initially conceptualized as the public ledger for the Bitcoin
cryptocurrency. The technology was developed to avoid centralization and to
solve a digital currency problem known as double-spend. Blockchain employs a
mechanism called consensus to ensure resiliency and immutability even as a
decentralized structure.

Ethereum introduced the Blockchain as a development platform, by enabling the
execution of logic in the network, in the form of smart contracts that execute
autonomously exactly as they were designed. Various other Blockchain platform
embraced this idea and many projects created their own Blockchain development
platform.

The Linux Foundation and IBM saw potential in this technology space and created
the Hyperledger developer-focused community. The focus of this community was
creating Blockchain based platforms that were aimed at the enterprise and the
reality of regulations and the need for auditability while still maintaining
the key characteristics of Blockchain such as immutability of data. Hyperledger
Fabric was one of the first projects that were built and is a general purpose
Distributed Ledger Platform.

Using Hyperledger Fabric a system was built that leverages the features of this
platform to manage the identity of patients in a Healthcare environment. This
system was designed and implemented successfully. The system was evaluated
against the CIA triad model and was able to meet the desired requirements.

It was shown that it is possible to create a system for the purpose mentioned
with this technology and that it provides several advantages in relation to
more traditional systems such as data transparency to the patient, immutability
of data and decentralization. The patient benefits from transparency because he
will be able to see his data everytime he wishes to do so, creating a greater
degree of trust with the Healthcare organizations than what is possible
nowadays. Immutability is important to maintain this trust because if he
discovers his data was altered in any way, then the record of that tampering
would be forever recorded and could be traced back to the malicious actor due
to this platform supporting auditability with the Membership Service Provider
that issues client identity certificates through the built-in CA service.
Decentralization provides additional resiliency as it avoids having a single
point of failure that could be targeted. Security must be a key focus of the
Healthcare industry for the next few years because expertise will increase in
the digital space providing more opportunities for malicious parties and data
privacy of patients must be treated with the utmost care.

This proves that this technology has many applications in this field, and that
it can be used more often as the platform becomes more mature and complete.

\section{Future Work}

There are many approaches that can be taken in future work. Blockchain
technology is certainly interesting and other platforms can be explored to
evaluate their suitability in Healthcare. Hyperledger Indy was a platform that
was not available when this thesis was initially discussed but due to its focus
on identity some investigation on this platform could be made and potentially
map similarities to Fabric in ways that could show their similarities and
differences to show which one is more suited for this task or the different use
cases they excel in.

Also the prototype project could be expanded with a graphical interface leading
to a more intuitive usage of this system and additional platforms to be
targeted. Additional optimization could be made regarding channels, encryption
methods, data segregation and scalability. Fabric roles could be explored to
distinguish between a doctor and a nurse in the same Healthcare organization.
This way access to information would be made through Role Based Access Control.
Hyperledger Burrow could be used to improve compatibility between Ethereum and
Hyperledger Fabric and target two platforms with similar smart contract code.
Finally this concept could be expanded further and Blockchain could serve as a
secure access key store that enable secure transmission of access keys to
external protected servers that store a big amount of data that the Blockchain
would not be suited for.
