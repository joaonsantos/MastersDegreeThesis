\chapter{Conclusions and Future Work}\label{Conclusion}

In this thesis a system was built that is capable of managing patients identity
using Blockchain technology. As this is a relatively new technology some
research was done and some conclusions of this work were laid out. 

In this work it was shown an overview of the Blockchain technology and its
origins. The evolution of this technology as first introduced by Nakamoto and
used in Bitcoin to solve the need for a middle man and enable trust between
unknown parties to a develpment platform and wide array of applications it
currently is used was then shown. It was shown that there are two different
implementations of the Blockchain each with its own purpose, with these being
the permissionless and permissioned implementations.

In order to gain some working knowledge a prototype system was deemed necessary
to be built that leverages this technology in the Healthcare field. The
platform chosen was the Hyperledger Fabric DLT. After some difficulties a
system was successfully designed and built and some conclusions were taken into
account. The platform was evaluated against the CIA triad, the international
standard for information security.

\section{Conclusions}

Blockchain was initially conceptualized as the public ledger for the Bitcoin
cryptocurrency. The technology was developed to avoid centralization and to
solve a digital currency problem known as double-spend. Blockchain employs a
mechanism called consensus to ensure resiliency and immutability even as a
decentralized structure.

Ethereum introduced the Blockchain as a development platform, by enabling the
execution of logic in the network, in the form of smart contracts that execute
autonomously exactly as they were designed. Various other Blockchain platform
embraced this ideia and many projects created their own Blockchain development
platform.

The Linux Foundation and IBM saw potential in this technology space and created
a Blockchain platform that was aimed at the enterprise and the reality of
regulations and the need for auditability while still maintaining the key
characteristcs of Blockchain such as immutability of data.

Using Hyperledger Fabric a system was built that leverages the features of this
platform to manage the identity of patients in a Healthcare environment. This
system was designed and implemented successfuly. The system was evaluated
against the CIA triad model and was able to meet the desired requirements. It
was shown that it is possible to create a system for the purpose mentioned with
this technology and that it provides several advantages in relation to more
traditional systems.

This proves that this technology has many applications in this field, and that
it can be used more often as the platform becomes more mature and complete.

\section{Future Work}

There are many approaches that can be taken in future work. Blockchain
technology is certainly interesting and other platforms can be explored to
evaluate their suitability in Healthcare. Hyperledger Indy was a platform that
was not available when this thesis was initially discussed but due to its focus
on identity some investigation on this platform could be made and potentially
map similarities to Fabric in ways that could show their similarities and
differences and show which one is more suited for this task.

Also the prototype project could be expanded with a graphical interface leading
to a more intuitive usage of this system and additional platforms to be
targeted. Additional optimization could be made regarding channels, encryption
methods, data segreggation and scalability. Hyperledger Burrow could be used to
improve compatibility between Ethereum and Hyperledger Fabric and target two
platforms with similar smart contract code. Finally this concepted could be
expanded further and Blockchain could serve as a secure access key store that
enable secure transmission of access keys to external protected servers that
store a big amount of data that the Blockchain would not be suited for.
