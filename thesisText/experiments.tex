\chapter{Experiments and System Evaluation} 
\label{experiments}

\emph{In this Chapter the solution created for managing patients identity data
in a Healthcare context is tested using some specific use cases. Then the
solution is evaluated against a security model and goals set by this
dissertation are evaluated.}

To properly evaluate this solution a number of experiments were conducted, as
follows. First some data would be present on the channel when the user
interacts that represents his identity in a certain clinic. The patient would
query the Blockchain for his data and receive his data if everything worked
accordingly. The second experiment was making the patient share his data with
the doctor in the channel. The last experiment consisted in the patient trying
to query data of another patient that was inserted at the genesis of the
network and seeing if the data was encrypted or was easily readable. The
outcomes of these experiments can shape the development of the solution as it
could take these results into consideration and highlight possible problems.

\section{Testing the Built Solution}

With the network in place and the peers set up and registered the experiments
proposed have now their requirements fulfilled.

The patient used the function provided by the application to query the network
for his information. He searched for his patient number and was shown his
information successfully. This shows that the information was recorded with
success when the chaincode was deployed. The simple way to query personal
information with an assigned patient number also proved successful and shows
that this system can be used to store patient's identity data and retrieve it.

Then the patient had to share his information with the doctor. To do this it
was necessary to assume that the patient had given his patient number to the
doctor so that he could use the application built to query for that patient
number. The doctor queried the network for the patient's information and was
able to access it successfully. This proves that this platform allows
surprisingly, to very easily share information between a patient and a doctor
using a smart contract in a simple way.

Finally the patient tried to access another patient's data. It was necessary to
assume that he was given the patient number by the respective patient. When he
queried the network for that patient's data it became clear what already had
arose suspicions in the previous experiments. He was actually able to access
that data without a problem. This would be okay if the number was willingly
given to him. However if the number was obtained unwillingly it could prove a
problem. This meant that the solution currently, did not meet the requirement
of the information being confidential that was defined previously, even tough
it is transparent and has high availability since the information was spread
through multiple peers and could be on multiple channels. It became clear that
some additional data security measures was needed.

\section{Evaluation of the Built Solution}

It was determined that to evaluate the effectiveness of this system in regards
to security, a standard for these types of solution was needed. The
international standard for information security known as the Confidentiality,
Integrity, and Availability (CIA) triad model was used and the solution was
evaluated against this standard, in order to draw further conclusions and
evaluate how secure the built system is, in regards to data security, which is
a critical concern in this particular field. The three pillars that form this
standard are the preservation of confidentiality, information availability and
ensuring information integrity.

\subsection{Confidentiality}

Confidentiality of the information stored in the network was considered a key
requirement when the requirements were presented. The Hyperledger Fabric was a
prime candidate for building the solution upon due to its focus on privacy and
a more enterprise approach to Blockchain development. While it offers many
features such as channels that truly do segregate information in a way that
many equivalent platforms cannot do at the moment it is also true that by
default data will be stored in plain text. 

To solve this problem it was necessary to implement data encryption on top of
the network using chaincode. This way, even if someone was able to access the
underlying database or if someone used a tool like Hyperledger Explorer to
explore the network, all it would see is encrypted data that would require a
key to decrypt and become human readable. With these considerations in mind,
even tough there are a number of aspects that define confidentiality it can be
said that the built system provides a confidential data storage.

\subsection{Integrity}

One of the key aspects of a Blockchain system is the immutability of data. This
means that once information is written, it cannot be changed or erased. The
transaction logs assure that the specific version of that asset is recorded
permanently in the network. In order to comply with privacy regulations some
data can become only visible as an hash but it still remains there. Therefore
the integrity of data on this Blockchain platform and solution is also
preserved.

\subsection{Availability}

Even tough Fabric is a permissioned Distributed Ledger Platform and as such it
is administrated by an administrator it is also distributed and therefore
avoids having a single point of attack. By default, it is more available than a
simple informational system that is centralized. In this aspect, it can be said
the more the network scales, the more robust it becomes and therefore more
availability it provides as information redundancy also increases.

\subsection{Thesis Goals}

The goal of this thesis was to evaluate if a system could be built based on
Blockchain technology to manage the patients identity in the Healthcare
environment. Using Hyperledger Fabric a system was built that successfuly can
create, manage and disable patients data. Information can be shared in a secure
manner and interoperability eases organization into adopting this system. This
system provides benefits to the medical staff as well as the patients due to
transparency in how data is handled and secured.

However the costs of deploying this system in a production ready environment
would be higher compared to a more traditional approach. Since this system is
built upon a Permissioned Platform, machines to host the central services need
to be acquired and an administrator of the platform is necessary for the
necessary maintenance. As the network grows it would become more resilient and
additional servers could be used to expand the core availability of Blockchain
components.

There is also the question of scalability. Even tough a Permissioned Blockchain
is always faster in relation to a Permissionless variant it still is far from
matching the scalability and performance of the Electronic Payment Management
System created by SIBS~\footnote{The Sociedade Interbancária de Serviços is a
company that manages all the debit card payment system in Portugal and that
operates with all banks. The company is responsible for the Multibanco network.
The network is comprised by the store payment machines and the automated
banking machines that offer money withdrawal and payment services, for example.
As of December 2014, the network had an average of more than 75 million
operations every month.} for banking transactions, for example. If this system
was intended for global use then additional approaches would need to be taken
regarding this matter.

With this said the pace of development has been relatively fast with new
releases on a quarterly basis that focus on the issues of scalability and
privacy, two important features pertaining to the system this Thesis proposed.
