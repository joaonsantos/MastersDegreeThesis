\chapter{Background}\label{background}

\begin{quote} 
  \emph{This Chapter presents an overview of the Blockchain technology.  Some
  Blockchain implementations are introduced and categorized. Consensus is
  introduced as a key aspect of this technology. Finally, developments in how
  the Healthcare industry has handled patients identity data management
  throughout the years is shown.}
\end{quote}


While Blockchain is not a new concept at this point, it is an evolving
technology that is being used to solve old problems with new approaches, while
at the same time creating new application fields and challenging old
conventions and methodologies.  Blockchain technology is having an
environmental and economic impact, as discussed further in this Chapter.

\section{Brief Introduction to Blockchain Technology}

Blockchain can be defined as a collection of cryptographic and network
technologies orchestrated to work together. The concept can also be used to
refer to the Bitcoin's Blockchain or refer to forks~\footnote{In this context,
a fork is a condition whereby the state of the Blockchain diverges into one or
more valid paths forward, where a part of the network has a different
perspective than a different part of the network. The fork can be either with
regards to a network's transaction history or a new rule in deciding what makes
a transaction valid. Since a fork can create a chain with different rules a new
Blockchain variation can be created.} of the Bitcoin's Blockchain called
Altchains~\cite{Lewis2015} that share some characteristics but may have
different features and purposes. Some forks even improved upon the original
premise of the concept, resulting in platforms that allow execution of code in
an autonomous manner, exactly as it was programmed, with no human intervention.

A Blockchain is, generally speaking, a continuously growing list of records
being written in the ledger. The ledger is a structure where all records are
written and stored. This structure is constantly replicated across a network of
peers, in opposition to having a single central record history, making it a
good example of a distributed database~\cite{Barclay2017}.

The purpose of a Blockchain is to estabilish trust between different
participating parties in a network of distributed systems without the need for
a trusted middle man~\cite{Drescher2017}. To fulfill this purpose it uses
cryptographic techniques and digital signatures to, not only verify the
authenticity of records, but also as a way to manage read or write access to
the network. These, are also used to create proof that a record was written in
the ledger and was never tampered with, creating an immutable history of
records.

Unlike a conventional database system running in a server, where only a single
entity keeps a copy of the underlying database, the ledger of the Blockchain is
constantly replicated across any number of participating nodes in the network,
making it a distributed system by design~\cite{Lewis2015}. Depending on the
Blockchain implementation, not every participant has the same ability to
interact with the ledger and in this respect a Blockchain can be permissionless
or permissioned.  Generally speaking, in a permissionless Blockchain every node
of the network can write to the ledger, whereas in a permissioned Blockchain
only a select group of entities have writing access to the ledger. The
permissioned alternative is secure by default if the entities who participate
the network are considered secure and trustworthy, for example, through a chain
of trust similar to Domain Name Systems digital certification
schema~\cite{Lewis2015,Valenta2017}.

But then, how does a permissionless Blockchain maintain security if every
participant in the network has access to writing on it, including potentially
malicious parties? Given that participating nodes in a public network can
belong to different and often competing parties, there is no implied trust
between them. Blockchain provides a mechanism to ensure the integrity of the
ledger and prevent malicious meddling from interested parties, while at the
same time, avoiding the need for a central authority~\cite{Barclay2017}. For
example, the Bitcoin Blockchain uses a peer-to-peer network and manages to
avoid the requirement of a third party being involved in a financial
transaction such as a financial institution or a middle man, in order to see it
through~\cite{Nakamoto2008}.

The mechanism employed by the Blockchain to solve this problem is called
consensus.  Even though consensus mechanisms can behave vastly different,
depending on its implementation and purpose, they are at the core, a solution
to create immutability and ensure resiliency and transparency by ensuring the
majority of the network agrees upon the sequence of events. For example, in the
Bitcoin's Blockchain case, consensus is reached by the longest chain rule where
the longest chain of blocks not only serves as proof of the sequence of events
witnessed, but as proof that it came from the largest pool of computing power.
This is due to the fact that the Bitcoin's Blockchain uses a proof of work
algorithm that relies on brute force to solve a complex mathematical puzzle,
making the longest chain of blocks the one with the most computing power behind
it and therefore agreed upon by the majority of the
network~\cite{Baars2016,Wood2017}, making that chain the most likely to be the
one that represents the sequence of events witnessed.

There have been however, environmental and economic reasons to replace
Proof-of-Work consensus algorithms. Nowadays the cryptocurrency mining, forms a
billion United States Dollar industry with an estimated consumption of 288
megawatts in 2017 and in 2016, 70\% of the Bitcoin's computational power was
located in China~\cite{BitcoinMining2017}.  Unfortunately, the vast majority of
electricity in the country is produced by burning coal, resulting in one of the
biggest carbon footprints in the world. In response to this environmental and
eventual economic concern over the sustainability of the mining incentives,
there have been a few alternative algorithms that have eventually appeared.
Proof-of-Stake, for example, is a consensus algorithm that was
\href{https://bitcointalk.org/index.php?topic=27787.0}{first suggested on the
Bitcointalk forum on July 11, 2011}. It is currently in use in various
currencies as their consensus algorithm and the first digital currency to use
this method was Peercoin in 2012. 

Rather than requiring the peers to perform a certain amount of computational
work, a Proof-of-Stake system requires the validators to show ownership of a
certain amount of money. Any participating peer in the network can become a
validator by sending a special type of transaction that locks their money in a
deposit, defined as the stake. The creator of a new block is chosen in a
deterministic way, depending on its wealth and other factors, determined by the
specific Proof-of-Stake implementation. Validators then participate in the
process of creating and agreeing to new blocks. Proof-of-Stake based consensus
algorithms have proven to be difficult to implement~\cite{EthereumSlasher2014},
leading some Blockchain platforms to consider implementing a mix of
Proof-of-Work and Proof-of-Stake algorithms.

In the case of a permissioned Blockchain implementation, interactions are made
among a set of known, identified participants who have a common goal, but do
not fully trust each other. By relying on the identities of peers, a
permissioned Blockchain can use a more traditional
\href{https://en.wikipedia.org/wiki/Byzantine_fault_tolerance}{Byzantine Fault
Tolerant} (BFT) consensus algorithm~\cite{Sousa2018}.

While the Blockchain, we now know today, was conceptualized as the public
ledger for the Bitcoin cryptocurrency in 2008 by Satoshi Nakamoto and
implemented in 2009, many are now using it as a foundation across a variety of
application areas such as traceability, asset management and
Healthcare~\cite{MIT2016}.

\section{Blockchain as a Platform} \label{blockchainasaPlatform}

Due to Bitcoin getting extensive media coverage, the average public awareness
in cryptocurrencies is shown to be rising~\cite{BitAwareness2017}. While
Blockchain is used as a means to increase the resiliency of the Bitcoin
cryptocurrency network from malicious parties, a token is used to represent the
coin. 

Just like a Euro, it has no value by itself, it only has value because we agree
to trade goods and services in exchange for a higher amount of the currency
under our control and we believe others will do the same \cite{aliessi2016}.
Through the years Blockchain has evolved to be capable of being an independent
development platform using the token as a means to reward those who maintain
the consensus by spending electricity and computation power in the network. In
some networks, Ethereum and Hyperledger Fabric for example, one can build upon
the network to create Decentralized Applications\footnote{In this thesis
context, Decentralized Applications are applications that run on a peer to peer
network of computers rather than a single computer. They are a type of software
program designed to exist on a network or multiple networks in a way that is
not controlled by any single entity. Decentralized applications consist of the
whole package, from backend to frontend. The smart contract is only the backend
of these type of applications.} (Ðapps) that allow logic to be executed in an
autonomous manner~\cite{Wood2017}. 

In the same manner that the Bitcoin Blockchain can be seen as an adding
machine, the Ethereum and Hyperledger Fabric Blockchain (see
Section~\ref{enterpriseBlockchain}) can be seen as computers able to execute
programs designed for it~\cite{Wood2015}.

Ethereum is an open-source platform based on the Blockchain technology that
enables developers to build and deploy Ðapps. Ethereum is being developed by
the Ethereum Foundation and was first discussed by Buterin in 2013. Ethereum
intends to provide a Blockchain with a built-in programming language that is
used to create smart contracts~\cite{Wood2017}, defined in the following
paragraphs.  Many Blockchain implementations nowadays use this concept.

A Blockchain that supports Bitcoin style transactions enables asset transfers
between parties that do not trust each other. A Blockchain that supports smart
contracts however, takes this further and allows for multi-step interactions to
occur between mutually distrustful parties. Nick Szabo introduced this concept
in 1994~\cite{Christidis2016} and defined a smart contract as "a computerized
transaction protocol that executes the terms of a contract". 

Smart contracts can translate contractual clauses into a piece of code,
embedding it into property hardware, or software that can self-enforce these.
Smart contracts are designed in order to minimize the need for trusted
intermediaries between transacting parties, as well as, the occurrence of
malicious or accidental exceptions.

 In a Blockchain, smart contracts are scripts that describe the logical backed
 of a Decentralized application and are stored on the Blockchain where they
 execute as “autonomous agents” and where they can be instantiated and invoked
 as needed after achieving consensus.  Since they reside on the network, they
 have a unique address. A smart contract is invoked by addressing a transaction
 to it.  It then executes independently and automatically in a prescribed
 manner, according to the data that was included in the invoking transaction.
 Smart contracts allow general purpose computations on the chain.  Smart
 contracts offer an abstract layer of interaction with the ledger, doing away
 with a required background in coding cryptography and mathematics, in order to
 program Blockchain applications~\cite{Wood2017,BlockGeeks2017}.

The Ethereum Blockchain is a permissionless Blockchain, and thus, it must have
a consensus mechanism to ensure the validation process of every record and, in
turn, ensure resiliency and immutability. While other implementations of the
Blockchain have different consensus mechanics, in Ethereum’s case, all
participants have to reach consensus over the order of all transactions that
have taken place. If a definitive order cannot be established then a
double-spend~\footnote{Double-spending is a potential flaw in a digital cash
scheme in which the same single digital token can be spent more than once.
This is possible because a digital token consists of a digital file that can be
duplicated or falsified. As with counterfeit money, such double-spending leads
to inflation by creating a new amount of fraudulent currency that did not
previously exist. This devalues the currency relative to other monetary units,
and diminishes user trust as well as the circulation and retention of the
currency.} might have occurred and the transaction is rejected~\cite{Wood2017}.

\section{Blockchain for Enterprise} \label{enterpriseBlockchain}

Hyperledger Fabric (HLF) is part of the Hyperledger project started in December
2015 by the Linux Foundation. It is an open-source developer-focused community
with the common goal of advancing the development of enterprise-grade,
open-source Blockchain-based solutions.  Fabric is an implementation of a
Distributed Ledger Platform (DLP) under the Hyperledger
umbrella~\cite{Cachin2016}.

Hyperledger Fabric’s initial commit was contributed by IBM and written in the
Go programming language.  It is a permissioned Blockchain and its main design
goal was to surpass previous Blockchain implementation limitations, such as,
lack of true private transactions and confidential contracts.

These goals are achieved thanks to assigning peers in the network three
distinct roles and by offering the ability to create channels each with its own
private ledger.  A peer has have the role of endorser, committer or consenter
or multiple roles.  Hyperledger Fabric is intended as a foundation for
developing applications in a modular fashion, opting for a plug-and-play
approach to its various components as well as its consensus
mechanism~\cite{HyperledgerFabricDocs2017}.

Hyperledger Fabric, as discussed, also allows the creation of smart contracts.
Fabric's key operational requirement is privacy, featuring true private
transactions and confidential contracts. As such it fits in a business
environment where sensitive information must be handled with care and disclosed
on a case by case basis. Thanks to its modular approach, consensus protocols
are no longer hard-coded and trust models can be repurposed, for example using
Hyperledger Burrow.

Hyperledger Burrow is also part of the Hyperledger project and its development
started in 2014 by Monax and sponsored by Intel. It is a permissionable smart
contract machine written in Go and offers a modular Blockchain client with a
permissioned smart contract interpreter built, in part, to the specification of
the Ethereum Virtual Machine (EVM) with the client having, essentially, three
main components, the consensus engine, the permissioned EVM and the Remote
Procedure Call gateway~\cite{Kuhlman2017,HyperledgerBurrow2017}.

Hyperledger Burrow has its own Consensus Engine, the Byzantine fault-tolerant
Tendermint protocol.  The Tendermint protocol is an open-source effort that
allows high performance in solving the consensus problem and also has a
flexible interface for building arbitrary applications above the consensus, as
well as, a suite of tools for deployments and their
management~\cite{Buchman2016}.

Hyperledger Indy is an open-source distributed ledger, purpose-built for
decentralized identity. Indy uses a modified version of Redundant Byzantine
Fault Tolerance called Plenum. Indy provides tools, libraries, and reusable
components for creating and using independent digital identities distributed
ledgers. Indy provides a software ecosystem where the users are in charge of
decisions about their own privacy and disclosure of such information.  Indy can
be used to define connection contracts, revocation and curated reputation, for
example. Hyperledger Indy was not used in this thesis as it was at the time in
incubation phase. Documentation was lacking and the platform was not feature
complete. With this said, Hyperledger Indy is mentioned could be used as future
work, as explained on Section~\ref{futureWork}.

The first network built on Indy was deployed on July 31, 2017, running version
1.0 of Indy. The Indy Software Development Kit (SDK) was released in August of
the same year. The SDK supports common programming languages like Python, Java,
Go, Node.js and Rust for interacting with the Indy ledger, running as Sovrin.
iOS support for Indy is mature, and Android support is planned. Institutions
currently have several incentives to adopt a solution similar to Indy, one
being regulation. GDPR, and other legal requirements are forcing companies to
adopt some measures in how they handle data pertaining to their clients and
employees. Privacy and user data control standards are being demanded by
governments and institutional organizations worldwide.

\section{Developments in Healthcare} \label{blockchainHealthcare}

Records of a patient were originally stored in paper, a physical format.
Thanks to the advent of the computers more and more records are stored on a
digital format and the Electronic Health Record (EHR) was
created~\cite{Marquez2017}. The digitalization of this data benefits handling
of information between the patient and the medical professionals and medical
institutions\cite{ONCoordinator2017}.

But what is defined as identity? Identity is a construct that depends on the
context. Identity is often defined as the characteristics determining who or
what something is. In this thesis context, identity is defined as the set of
characteristics that determine who a patient is, in the given Healthcare
ecosystem they belong to, such as the name, the age, the cellphone number, the
gender and the birth date of the patient.  

Electronic Health Records encapsulate this information in digital format.
Unfortunately, they are usually represented in a format according to the
Information System they were designed to work with, meaning that they are not
created according to any established standard. The increasing limitations of
paper-based records, the potential benefits of Electronic Health Records and
the acknowledged challenges of delivering these in practice have stimulated a
considerable investment in research and development of this solution.  Between
1991 and 1998 the European Union provided considerable direct funding support
to related research projects~\cite{Kalra2006}.

To enable interoperability, standards for EHR were sought after and
considerable research has been undertaken since 1996 to develop architecture
formalisms to capture Healthcare data comprehensively, however, many of them
have had no significant adoption, failing to bring the much needed
consensus~\cite{Eichelberg2006} that was required for enabling interoperability
between different Information Systems in Healthcare institutions. 

Healthcare professionals increasingly require access to detailed and complete
health records in order to manage a safe and effective delivery of their care
service, as well as, being able to share this information with their team in an
efficient way. Patients nowadays should also be able to access their own
information to an extent that allows them to play an active role in the
management of their health related data. These requirements are becoming
increasingly urgent as the focus of Healthcare delivery shifts progressively
from specialist centres to community settings and to the patient’s personal
environment.

Health Level 7 organization has done much work to be recognized internationally
and their standard for Healthcare data interoperability is being implemented in
many countries to allow for joint efforts between medical and clinical
institutions.  As of 2017, HL7 has an official presence in 34 countries such as
United Kingdom, Spain, France, Germany, Russia and China~\cite{HL7Anual2016}.
The organization has been keen on expanding their standard internationally as
work continues on official adoption in other countries. Adding to this, HL7 has
been providing several events and discussion meetings, as well as, developing
its learning ecosystem in the form of certifications and web streams, for
example. In 2017, HL7 has partnered with the
\href{https://www.hspconsortium.org/}{Health Services Platform Consortium},
composed of more than a dozen clinical professional societies, committing
resources for creating and testing clinical data models and thus beginning to
lay a semantic foundation for achieving interoperability.

There is a greater need for digital security solutions, as the amount of data
grows in a connected age, where access to the world wide web is easily
available and every device is part of the
\href{https://en.wikipedia.org/wiki/Internet_of_things}{Internet of Things}.
Various industries and the general public are becoming interested in solutions
to solve problems in this field. Blockchain started as a security solution and
is now laying the foundation for a change in the underlying flow of our
economic and social systems~\cite{Zago2018,Marr2018,Long2018}. Some companies
have already started developing Blockchain applications in the Healthcare field
and established some key partnerships.

Guardtime has fully deployed their system in 2008, started cooperating in 2011
and in 2016 announced a partnership with the Estonian Government, where a
million patient records are now secured by the system. Guardtime uses a system
that shares the same background as the Bitcoin but is not based on Bitcoin.
\href{https://guardtime.com/technology}{The Keyless Signature Infrastructure
system} proves the resilience of Blockchain related concepts, as well as, other
advances in cryptography. Large companies like Verizon are becoming interested
in Blockchain technology for their own
purposes~\cite{GuardTime2018,EstonianGovernmentGuardTime2016}.

Another company, Gem, is collaborating with Phillips Healthcare to explore
options in this area\cite{philips2016}, and is opting to solve the
interoperability problem with an additional layer of abstraction they call
GemOS~\cite{gemOS2018}. Factom, another Blockchain-based service, has also
announced a partnership with a major US medical services provider
HealthNautica~\cite{BlockchainCompHealth2017,FactomPartnership2017}.

Recently, a platform called Medichain was introduced. This platform is also
based on Blockchain technology and it allows patients to store their own data
in a secure way and give anonymized access to this data to specialists. Giving
data rewards users with tokens that represent value~\cite{MediChain2018},
effectively allowing patients to knowingly monetize their Healthcare data.

Even tough many Blockchain based solutions are still very early on development
or deployment and many projects do not actually materialize, the disruption
potential of this technology is clear. All over the world many efforts are
being made to regulate the high amount of data that is being generated by
digital services.

For example, in the European Union (EU), the EU General Data Protection
Regulation (GDPR) is officially in effect as of May 25, 2018. The aim of this
regulation is to protect all EU citizens from privacy and data breaches in an
increasingly data-driven world that is vastly different from when the preceding
1995 directive was established. GDPR defines a set of points that organizations
and enterprises must comply regarding the handling of personal data. Since GDPR
is a regulation, not a directive, it does not require national governments to
pass any enabling legislation and is directly binding and applicable. A
violator of this regulation may be fined up to \textbf{20 million € or up to
4\% of the annual worldwide turnover} of the preceding financial year in case
of an enterprise, whichever is greater. Although the key principles of data
privacy still hold true to the previous 1995 directive, terms of usage as well
as consent to use data has become more clear to the user of a data-driven
service.  Therefore, any solution proposed in the Healthcare field must also
take this legal landscape into consideration which may be a problem. One of the
key features of Blockchain is immutability which means that data once written
it is impossible to delete or tamper with. This causes a practical clash
between this technology and the regulation, even tough they both are aligned in
ideological concepts as both seek to return control of data usage and handling
to their rightful owners. With this in mind it becomes clear that to comply
with this regulation some additional steps need to be taken by this technology
or eventual amendments need to be added to the regulation.
