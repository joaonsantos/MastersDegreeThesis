\chapter{Background}\label{background}


\begin{quote} \emph{"This project aims to build a Blockchain based system to
  manage the identity of patients and investigating the suitability of creating
  such a system in the Healthcare environment according to objective criteria.
  While Blockchain is not a new concept at this point, it is an evolving
  technology that is being used to solve old problems with new approaches while
  at the same time creating new application fields and challenging old
  conventions and methodologies. This Chapter will provide an overview of this
  technology and some of its most prominent implementations. Finally, some
  context is given to how technology has been helping the Healthcare industry
  to enable better management of their patients identity and how the current
  information systems of Healthcare establishments handle this task."}
\end{quote}

\section{Blockchain Technology}

  The concept of Blockchain is abstract. It is a collection of technologies
  orchestrated to work together. In this sense the concept can be used to refer
  to the Bitcoin's Blockchain, alternative implementations or even forks of the
  Bitcoin Blockchain called Altchains~\cite{Lewis2015} that share many
  characteristics but may have different features and purposes. It can even
  refer to platforms that allow execution of code in an autonomous manner,
  exactly as it was programmed, with no human intervention.  A Blockchain is,
  generally speaking, a continuously growing list of records being written in
  the ledger, a structure where all records are written and stored, that is
  constantly being replicated across a network of peers, in opposition to
  having a single central record history, making it a good example of a
  distributed database, thus avoiding having a single central point of failure
  that can be easily targetable~\cite{Barclay2017}.

  The purpose of a Blockchain is to maintain integrity in a network of
  distributed systems~\cite{Drescher2017}. To fulfill this purpose it uses
  cryptographic techniques and digital signatures to not only verify the
  authenticity of records but also as a way to manage read or write access to
  the network and as proof that a record was written in the ledger and was
  never tampered with, creating an immutable history of records, that benefits
  various use cases as discussed later in this document.

  Unlike a conventional database system running in a server, where only a
  single entity keeps a copy of the underlying database, making it centralized
  by design, the ledger of the Blockchain is constantly replicated across any
  number of participating nodes in the network~\cite{Lewis2015} in a regular
  cadence defined at the genesis of the network. In some implementations, not
  every participant has the same ability to interact with the ledger and in
  this respect a Blockchain can be permissionless or permissioned. Generally
  speaking, in a permissionless Blockchain every node of the network can write
  in the ledger whereas in a permissioned Blockchain only a select group of
  entities have access to writing in the ledger, making the permissioned
  version, by default secure, if the entities themselves who manage the network
  are considered secure and trustworthy by the participants in the
  network~\cite{Lewis2015,Valenta2017}.

  But then, how does a permissionless Blockchain maintain security if every
  participant in the network has access to writing on it, including potentially
  malicious parties?

  Given that participating nodes in a public network can belong to different
  and often competing parties, there is no implied trust between them, so the
  Blockchain needs a mechanism to ensure the integrity of the ledger and
  prevent malicious meddling from interested parties and avoiding the need for
  a central authority~\cite{Barclay2017}.  Take for example the Bitcoin
  Blockchain that uses a peer-to-peer network to avoid the requirement of a
  third party being involved in a financial transaction such as a financial
  institution or a middle man, which must be trusted with the details of a
  transaction to see it through~\cite{Nakamoto2008}.

  Consensus is a mechanism employed by the Blockchain to solve this problem.
  Even though consensus mechanisms can behave vastly different, depending on
  its implementation and purpose, they are at the core a solution to create
  immutability and ensure resiliency by ensuring the majority of the network
  agrees upon the sequence of events.  For example, in the Bitcoin's Blockchain
  case, consensus is reached by the longest chain rule where the longest chain
  of blocks not only serves as proof of the sequence of events witnessed, but
  as proof that it came from the largest pool of computing power, as it uses a
  proof of work (\textbf{pow}) algorithm that relies on brute force to solve a
  complex mathematical puzzle, making the longest chain of blocks the one with
  the most computing power behind it and therefore agreed upon by the majority
  of the network~\cite{Baars2016,Wood2017} making it the most likely to be the
  one that represents the sequence of events witnessed.

  While the Blockchain, we now know today, was conceptualized as the public
  ledger for the Bitcoin cryptocurrency in 2008 by Satoshi Nakamoto and
  implemented in 2009, many are now using it as a foundation across many
  application areas such as traceability and asset management~\cite{MIT2016}.
  Thanks to the roaring success of Bitcoin and the increasingly apparent use
  cases that the Blockchain can provide, the public and the various industries
  interest in this technological advance is rising and it is quickly becoming a
  technological foundation in our economic and social systems~\cite{Zago2018,
  Marr2018,Long2018}.

  \subsection{Ethereum}

  Due to Bitcoin getting extensive media coverage, the average public awareness
  in cryptocurrencies is shown to be rising~\cite{BitAwareness2017}. While
  Blockchain is used as a means to increase the resiliency of the Bitcoin
  cryptocurrency from malicious parties, a token is used to represent the coin. 
  
  Just like a Dollar it has no value by itself, it has value only because we
  agree to trade goods and services in exchange for a higher amount of the
  currency under our control and we believe others will do the same
  \cite{aliessi2016}. Through the years Blockchain has evolved to be capable of
  being an independent development platform using the token as a means to
  reward those who maintain the consensus by spending electricity and
  computation power in the network. In some networks like Ethereum one can
  build upon the network to create Decentralized Applications (\textbf{Ðapps})
  that allow logic to be executed in an autonomous manner~\cite{Wood2017}. 
  
  In the same manner that the Bitcoin Blockchain can be seen as an adding
  machine, the Ethereum Blockchain can be seen as a computer able to execute
  programs designed for it~\cite{Wood2015}.

  Ethereum is an open-source platform based on the Blockchain technology that
  enables developers to build and deploy \textbf{Ðapps}. Ethereum is being
  developed by the Ethereum Foundation and was first discussed by Buterin in
  2013.  Ethereum intends to provide a Blockchain with a built-in programming
  language that is used to create \textit{Smart contracts}~\cite{Wood2017}.

  These are used to describe the logic of any system that developers can
  imagine and, when created, can be deployed to the Blockchain where they
  execute as “autonomous agents”.  Thanks to these tools it is safe to say that
  long gone are the days where building Blockchain applications required a
  complex background in coding cryptography, mathematics as well as significant
  resources~\cite{Wood2017,BlockGeeks2017}.

  The Ethereum Blockchain is a permissionless Blockchain, and thus, it must
  have a consensus mechanism to ensure the validation process of every record
  and, in turn, ensure resiliency and immutability. While other implementations
  of the Blockchain have different consensus mechanics, in Ethereum’s case, all
  participants have to reach consensus over the order of all transactions that
  have taken place. If a definitive order cannot be established then a
  double-spend~\footnote{Double-spending is a potential flaw in a digital cash
  scheme in which the same single digital token can be spent more than once.
  This is possible because a digital token consists of a digital file that can
  be duplicated or falsified. As with counterfeit money, such double-spending
  leads to inflation by creating a new amount of fraudulent currency that did
  not previously exist. This devalues the currency relative to other monetary
  units, and diminishes user trust as well as the circulation and retention of
  the currency.} might have occurred and the transaction is
  rejected~\cite{Wood2017}.

  \subsection{Fabric}

  Hyperledger Fabric (\textbf{hlf}) is part of the Hyperledger project started
  in December 2015 by the Linux Foundation, and is an open-source
  developer-focused community of communities focused on the development of
  enterprise-grade, open-source Blockchain-based solutions.  Fabric is an
  implementation of a Distributed Ledger Platform (\textbf{dlp}) under the
  Hyperledger umbrella~\cite{Cachin2016}.

  Hyperledger Fabric’s initial commit was contributed by IBM and written in the
  Go programming language.  It is a permissioned Blockchain and its main design
  goal was to surpass previous Blockchain implementation limitations, such as,
  lack of true private transactions and confidential contracts.

  These goals are achieved thanks to assigning peers in the network three
  distinct roles and by offering the ability to create channels each with its
  own private ledger.  A peer can have the role of endorser, committer or
  consenter or sometimes multiple roles.  Hyperledger Fabric is intended as a
  foundation for developing applications in a modular fashion, opting for a
  plug-and-play approach to its various components as well as its consensus
  mechanism~\cite{HyperledgerFabricDocs2017}.

  Hyperledger Fabric, as discussed, also allows the creation of smart contracts
  which can be written in Chaincode.  Given that this Blockchain's key
  operational requirement is privacy, featuring true private transactions and
  confidential contracts, it makes this technology a great asset for a business
  environment where sensitive information must be handled with care and
  disclosed on a case by case basis.  Thanks to its modular approach consensus
  protocols are no longer hard-coded and trust models can be repurposed.

  \subsection{Burrow}

  Hyperledger Burrow (\textbf{hlb}) is also part of the Hyperledger project and
  its development started in 2014 by Monax and sponsored by Intel. It is a
  permissionable smart contract machine written in Go and offers a modular
  Blockchain client with a permissioned smart contract interpreter built, in
  part, to the specification of the Ethereum Virtual Machine (\textbf{evm})
  with the client having, essentially, three main components, the consensus
  engine, the permissioned \textbf{evm} and the Remote Procedure Call
  (\textbf{rpc}) gateway~\cite{Kuhlman2017,HyperledgerBurrow2017}.

  Hyperledger Burrow has its own Consensus Engine, the Byzantine fault-tolerant
  Tendermint protocol.  The Tendermint protocol is an open-source effort that
  allows high performance in solving the consensus problem and also has a
  flexible interface for building arbitrary applications above the consensus,
  as well as, a suite of tools for deployments and their
  management~\cite{Buchman2016}.
  
  \section{Identity in Healthcare}
  
  Originally records of a patient were stored in paper, a physical format.
  Thanks to the advent of the computers more and more records are stored on a
  digital format and the Electronic Health Record (EHR) was created.
  \cite{Marquez2017}  This benefits handling of information between the patient
  and the medical professionals and medical
  institutions.\cite{ONCoordinator2017} But first we must discuss what is
  defined as identity in this specific case.
  
  Identity is a construct that depends on the context.  Identity can be defined
  as the characteristics determining who or what a person is.  In this paper we
  define identity as the set of characteristics that determine who is the
  patient in the given Healthcare ecosystem they belong to, such as, the name,
  the age, the cellphone number, the gender and the birth date of the patient.
  Electronic Health Records encapsulate this information in digital format,
  however, they are usually represented in a format according to the
  Information System they were designed to work with.
  
  To enable interoperability, standards for EHRs were created and many failed
  to bring the much needed consensus that was required for interoperability
  between different Information Systems in different institutions.
  \cite{Eichelberg2006} Health Level 7 has done much work to be recognized in
  many countries and is quickly being implemented in many countries to allow
  for joint efforts between organizations. \cite{HL7Anual2016}
  
  Even with these advances in mind, the nature of many clinics and hospitals
  Information Systems makes the management of their patients identity a very
  cumbersome, costly and risky affair to handle.  Security in a connected age,
  where internet is easily available, is lagging behind and presenting some
  problems.  There is also the question of transparent use of information by
  the organizations that store it.
