\chapter{Background}\label{background}

This project aims to design a Blockchain based system to provide a more
transparent and secure system to identify a patient in Healthcare. While
Blockchain is not a new concept at this point, it is an evolving technology
that is being used to solve old problems with new approaches. This chapter will
briefly explore some implementations of this technology, as well as its origins
and history.

\section{Blockchain Technology}

  A Blockchain can be many things. It can refer to the Bitcoin Blockchain,
  alternative implementations or forks of the Bitcoin Blockchain called
  Altchains~\cite{Lewis2015}. It can even refer to platforms that allow
  execution of code in an autonomous manner, exactly as it was programmed, with
  no human intervention.  A Blockchain is a continuously growing list of
  records, written in the ledger, a structure where records are written, that
  is constantly being replicated across a network of peers, in opposition to
  having a single central record history, making it a good example of a
  distributed database, thus avoiding having a single central point of failure
  that can be easily targetable~\cite{Barclay2017}.

  The purpose of the Blockchain is to maintain integrity in distributed
  systems~\cite{Drescher2017}. To fulfill this purpose it uses cryptographic
  techniques and digital signatures to not only verify the authenticity of
  records but also as a way to manage read or write access to the network and
  as proof that a record was written in the ledger.

  Unlike a conventional data storage, where only a single entity keeps a copy
  of the underlying database, making it centralized by design, the ledger of
  the Blockchain is replicated across any number of participating nodes in the
  network~\cite{Lewis2015}. In some implementations, not every participant has
  the same ability to interact with the ledger and in this respect a Blockchain
  can be permissionless or permissioned. In a permissionless Blockchain every
  node of the network can write in the ledger whereas in a permissioned
  Blockchain only a select group of entities have access to writing in the
  ledger, making the permissioned version, by default, secure if the entities
  themselves are secure and considered
  trustworthy~\cite{Lewis2015,Valenta2017}.

  But then, how does a permissionless Blockchain maintain security if every
  participant has access to writing on it, including potentially malicious
  parties?

  Given that participating nodes in the network can belong to different and
  often competing parties, there is no implied trust between them, so the
  Blockchain needs a mechanism to ensure the integrity of the ledger and
  prevent malicious meddling from interested parties and avoid having the need
  for a central authority~\cite{Barclay2017}.  Take for example the Bitcoin
  Blockchain that uses a peer-to-peer network to avoid the requirement of a
  third party being involved in a financial transaction such as a financial
  institution or a middle man, which must be trusted with the details of a
  transaction and its value to complete a transaction~\cite{Nakamoto2008}.

  Consensus is a mechanism employed by the Blockchain to solve this problem.
  Even though consensus mechanisms can behave vastly different, depending on
  its implementation and purpose, they are at core, a solution to create
  immutable records and ensure resiliency.  For example, in the Bitcoin's
  Blockchain case, consensus is reached by the longest chain rule where the
  longest chain not only serves as proof of the sequence of events witnessed,
  but as proof that it came from the largest pool of computing power as it uses
  a proof of work algorithm that relies on brute force to solve a complex
  mathematical puzzle~\cite{Baars2016,Wood2017}.

  While the Blockchain, we now know today, was conceptualized as the public
  ledger for the Bitcoin cryptocurrency in 2008 by Satoshi Nakamoto and
  implemented in 2009, many are now using it as a foundation across many
  application areas such as identity management, traceability and asset
  management~\cite{MIT2016}. Thanks to the roaring success of Bitcoin and the
  increasingly apparent use cases that the Blockchain can provide, the public
  and the various industries interest in this technological advance is rising
  and it is quickly becoming a technological foundation in our economic and
  social systems~\cite{Zago2018, Marr2018,Long2018}.

  \subsection{Ethereum}

  Due to Bitcoin getting extensive media coverage, the average public awareness
  in cryptocurrencies is rising~\cite{BitAwareness2017}. While there are some
  people considering cryptocurrencies and the Blockchain to be essentially the
  same technology, Blockchain has evolved to be capable of being a platform
  that can be developed on to create Decentralized Applications
  (\textbf{Ðapps})

  Ethereum is an open-source platform based on the Blockchain technology that
  enables developers to build and deploy \textbf{Ðapps}. Ethereum is being
  developed by the Ethereum Foundation and was first discussed by Buterin in
  2013.  Ethereum intends to provide a Blockchain with a built-in programming
  language that is used to create \textit{Smart contracts}~\cite{Wood2017}.

  These are used to describe the logic of any system that developers can
  imagine and, when created, can be deployed to the Blockchain where they
  execute as “autonomous agents”.  Thanks to these tools it is safe to say that
  long gone are the days where building Blockchain applications required a
  complex background in coding cryptography, mathematics as well as significant
  resources~\cite{Wood2017,BlockGeeks2017}.

  The Ethereum Blockchain is a permissionless Blockchain, and thus, it must
  have a consensus mechanism to ensure the validation process of every record
  and, in turn, ensure resiliency and immutability. While other implementations
  of the Blockchain have different consensus mechanics, in Ethereum’s case, all
  participants have to reach consensus over the order of all transactions that
  have taken place. If a definitive order cannot be established then a
  double-spend~\footnote{Double-spending is a potential flaw in a digital cash
  scheme in which the same single digital token can be spent more than once.
  This is possible because a digital token consists of a digital file that can
  be duplicated or falsified. As with counterfeit money, such double-spending
  leads to inflation by creating a new amount of fraudulent currency that did
  not previously exist. This devalues the currency relative to other monetary
  units, and diminishes user trust as well as the circulation and retention of
  the currency.} might have occurred and the transaction is
  rejected~\cite{Wood2017}.

  \subsection{Fabric}

  Hyperledger Fabric (\textbf{hlf}) is part of the Hyperledger project started
  in December 2015 by the Linux Foundation, and is an open-source
  developer-focused community of communities focused on the development of
  enterprise-grade, open-source Blockchain-based solutions.  Fabric is an
  implementation of a Distributed Ledger Platform (\textbf{dlp}) under the
  Hyperledger umbrella~\cite{Cachin2016}.

  Hyperledger Fabric’s initial commit was contributed by IBM and written in the
  Go programming language.  It is a permissioned Blockchain and its main design
  goal was to surpass previous Blockchain implementation limitations, such as,
  lack of true private transactions and confidential contracts.

  These goals are achieved thanks to assigning peers in the network three
  distinct roles and by offering the ability to create channels each with its
  own private ledger.  A peer can have the role of endorser, committer or
  consenter or sometimes multiple roles.  Hyperledger Fabric is intended as a
  foundation for developing applications in a modular fashion, opting for a
  plug-and-play approach to its various components as well as its consensus
  mechanism~\cite{HyperledgerFabricDocs2017}.

  Hyperledger Fabric, as discussed, also allows the creation of smart contracts
  which can be written in Chaincode.  Given that this Blockchain's key
  operational requirement is privacy, featuring true private transactions and
  confidential contracts, it makes this technology a great asset for a business
  environment where sensitive information must be handled with care and
  disclosed on a case by case basis.  Thanks to its modular approach consensus
  protocols are no longer hard-coded and trust models can be repurposed.

  \subsection{Burrow}

  Hyperledger Burrow (\textbf{hlb}) is also part of the Hyperledger project and
  its development started in 2014 by Monax and sponsored by Intel. It is a
  permissionable smart contract machine written in Go and offers a modular
  Blockchain client with a permissioned smart contract interpreter built, in
  part, to the specification of the Ethereum Virtual Machine (\textbf{evm})
  with the client having, essentially, three main components, the consensus
  engine, the permissioned \textbf{evm} and the Remote Procedure Call
  (\textbf{rpc}) gateway~\cite{Kuhlman2017,HyperledgerBurrow2017}.

  Hyperledger Burrow has its own Consensus Engine, the Byzantine fault-tolerant
  Tendermint protocol.  The Tendermint protocol is an open-source effort that
  allows high performance in solving the consensus problem and also has a
  flexible interface for building arbitrary applications above the consensus,
  as well as, a suite of tools for deployments and their
  management~\cite{Buchman2016}.
