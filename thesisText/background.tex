\chapter{Background}

While Blockchain is not a new concept at this point, it is an evolving
technology that is being used to solve old problems with new approaches. This
section will explore the Blockchain technology origins and history, some of its
different implementations and a brief history to the identity problem is
presented.

\section{Blockchain Technology}

  A Blockchain can be many things. It can refer to the Bitcoin Blockchain,
  alternative implementations or forks of the Bitcoin Blockchain called Altchains
  or even platforms that allow execution of code in an autonomous manner, exactly
  as it was programmed, with no human intervention.  It is a continuously growing
  list of records, written in the ledger, a structure where records are written,
  that is being replicated across a network of devices in opposition to having a
  single central record history, making it a good example of a distributed
  database.  \cite{Wood2017}

  The main design goal of the Blockchain is security and to fulfill this purpose
  it uses techniques such as cryptography and digital signatures to not only
  verify the authenticity of records but also read or write access to the
  network.

  Unlike a conventional central data storage, where only a single entity keeps a
  copy of the underlying database, the ledger of the Blockchain is replicated
  across any number of nodes.  Not every participant has the same ability to
  interact with the ledger and in this respect a Blockchain can be permissionless
  or permissioned. In a permissionless Blockchain every node of the network can
  write in the Blockchain whereas in a permissioned Blockchain only a select
  group of entities have access to writing in the ledger, making the permissioned
  version, by default, secure if the entities themselves are secure and
  considered trustworthy.

  How does a permissionless Blockchain maintain security if every participant has
  access to writing on it, including potentially malicious parties?

  Take for example the Bitcoin Blockchain that uses a peer-to-peer network to
  avoid meddling from a financial institution or a third party in a financial
  transaction. Given that participating nodes in the network can belong to
  different and often competing parties, there is no implied trust between them,
  so the Blockchain needs a mechanism to ensure the integrity of the ledger and
  prevent malicious meddling from interested parties or to avoid a central
  authority.\cite{Barclay2017}

  To solve this problem, consensus mechanisms are used differently, depending on
  its implementation, but having, at its core, a solution to create immutable
  records and ensure security.  In Bitcoin Blockchain’s case, consensus is
  reached by the longest chain rule where the longest chain not only serves as
  proof of the sequence of events witnessed, but as proof that it came from the
  largest pool of computing power.\cite{Baars2016}

  While the first Blockchain was conceptualized as the public ledger for the
  Bitcoin cryptocurrency in 2008 by Satoshi Nakamoto and implemented in 2009,
  many are now using it as a foundation across many application areas such as
  identity management, traceability and asset management.  Thanks to the roaring
  success of Bitcoin and the increasingly apparent use cases that the Blockchain
  can provide, the public awareness of it is rising and it is quickly becoming a
  technological foundation in our economic and social systems.
  % Need References for this

  \subsection{Ethereum}

  Bitcoin is getting media coverage almost everyday and public awareness in
  cryptocurrencies in general is rising.  Some people are considering
  cryptocurrencies and the Blockchain, to be essentially the same technology and,
  while that may have been somewhat true not so long ago, Blockchain technology
  is starting to be used in a plethora of ways.

  Ethereum is an open-source platform based on the Blockchain technology that
  enables developers to build and deploy Decentralized Applications
  (\textit{DAPPs}).  Ethereum is being developed by the Ethereum Foundation and
  was first discussed by Buterin in 2013.  Ethereum intends to provide a
  Blockchain with a built-in programming language that is used to create
  \textit{Smart contracts}.  \cite{Wood2017}

  These contracts are used to describe the logic of any system that developers
  can imagine and, when created, can then be deployed to the Blockchain where
  they execute as “autonomous agents”.  Thanks to these tools it is safe to say
  that long gone are the days where building Blockchain applications required a
  complex background in coding cryptography, mathematics as well as significant
  resources.\cite{Wood2017,BlockGeeks2017}

  Ethereum Blockchain is a permissionless Blockchain, and thus, it must have a
  consensus mechanism to ensure the validation process of every record and, in
  turn, ensure security and immutability. While other implementations of the
  Blockchain have different consensus mechanics, in Ethereum’s case, all
  participants have to reach consensus over the order of all transactions that
  have taken place. If a definitive order cannot be established then a
  double-spend might have occurred.

  \subsection{Fabric}

  Hyperledger Fabric (HLF) is part of the
  Hyperledger project started
  in December 2015 by the Linux Foundation, and is an open-source
  developer-focused community of communities focused on the development of
  enterprise-grade, open-source Blockchain-based solutions.  Fabric is an
  implementation of a Distributed Ledger Platform (DLP) under the Hyperledger
  umbrella.  \cite{Cachin2016}

  HLF’s initial commit was contributed by IBM and written in Go language.  It is
  a permissioned Blockchain and its main design goal was to surpass previous
  Blockchain implementation limitations, such as, lack of true private
  transactions and confidential contracts.

  This is achieved thanks to assigning peers in the network three distinct roles
  and by offering the ability to create channels each with its own private
  ledger.  A peer can have the role of endorser, committer or consenter or
  sometimes multiple roles.  HLF is intended as a foundation for developing
  applications in a modular fashion, opting for a plug-and-play approach to
  various components. \cite{HyperledgerFabricDocs2017}

  HLF, as discussed, also allows the creation of smart contracts which can be
  written in Chaincode.  As this Blockchain's key operational requirement is
  privacy, true private transactions and confidential contracts can exist and are
  a great asset for a business environment where sensitive information is
  necessary and disclosed often.  Thanks to its modular approach consensus
  protocols are no longer hard-coded and trust models can be repurposed.

  \subsection{Burrow}

  Hyperledger Burrow (HLB) is also part of the Hyperledger project and its
  development started in 2014 by Monax and sponsored by Intel. It is a
  permissionable smart contract machine written in Go and offers a modular
  Blockchain client with a permissioned smart contract interpreter built, in
  part, to the specification of the Ethereum Virtual Machine (EVM) and the client
  has, essentially, three main components, the consensus engine, the permissioned
  EVM and the Remote Procedure Call (RPC) gateway.
  \cite{Kuhlman2017,HyperledgerBurrow2017}

  HLB has its own Consensus Engine, the Byzantine fault-tolerant Tendermint
  protocol.  The Tendermint protocol is an open-source effort that allows high
  performance in solving the consensus problem and also has a flexible interface
  for building arbitrary applications above the consensus, as well as, a suite of
  tools for deployments and their management. \cite{Buchman2016}
