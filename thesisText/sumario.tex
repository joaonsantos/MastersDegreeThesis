\begin{tueSUMARIO}
  {\Large Gestão de Identidade nos Serviços de Saúde Utilizando Tecnologia
  Blockchain}
	\bigskip

  A criptomoeda Bitcoin foi essencial para criar uma solução para transacções
  digitais seguras, sem requerer a participação de um terceiro interveniente
  fidedigno para ambas as partes.  Para resolver este problema, os mecanismos
  que hoje são associados com a tecnologia Blockchain foram concebidos e
  implementados para servir como base para a rede da Bitcoin. Mais
  especificamente, esta foi utilizada como um mecanismo de segurança, de forma
  a tornar a Bitcoin uma forma de dinheiro mais transparente e estável, uma
  moeda criptográfica. Mesmo que a Bitcoin não tenha conseguido cumprir o seu
  propósito original, a tecnologia Blockchain despoletou novas inovações e
  permitiu maior criatividade.

  A Blockchain tem sido, desde então, a base tecnológica de várias
  criptomoedas. Algumas implementações desta tecnologia permitem a execução de
  código de uma forma autónoma exactamente como foi programado, sem intervenção
  humana.  Habitualmente chamados \textit{smart contracts}, estes podem ser
  usados para resolver um novo conjunto de problemas devido ao seu
  comportamento transparente, ausência de intervenção humana e devido à sua
  natureza distribuida. 

  A Blockchain é uma tecnologia que permite a criação de sistemas que
  introduzem um conjunto de beneficios em relação aos sistemas tradicionais de
  armazenamento de dados utilizados nos serviços de saúde. Custos e riscos
  associados a estes sistemas podem ser reduzidos e a informação pode ser mais
  transparente e fidedigna para todos os participantes.

  A rede Hyperledger Fabric com transacções privadas e mecanismos avançados de
  segurança foi usada como base para a criação do sistema proposto nesta
  dissertação. Adicionalmente, uma aplicação foi criada que usa \textit{smart
  contracts} para manipular o \textit{ledger} da Blockchain.

  O trabalho apresentado nesta dissertação mostra que um sistema baseado em
  Blockchain, neste caso em Hyperledger Fabric, é adequado a gerir a identidade
  de utentes,  em organizações prestadoras de cuidados de saúde. Apesar das
  funcionalidades apresentadas por esta plataforma serem focadas em privacidade
  e segurança, algumas medidas adicionais em torno da confidencialidade dos
  dados tiveram de ser implementadas. Independentemente disso, o sistema foi
  construido com sucesso e conseguiu cumprir os requerimentos que foram
  definidos. A implementação deste sistema em serviços de saúde traria
  tranparência, imutabilidade e segurança adicional para utentes e
  profissionais de saúde.

\end{tueSUMARIO}
