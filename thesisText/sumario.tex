%!TEX root = thesis_text.tex
\begin{tueSUMARIO}

  A criptomoeda Bitcoin foi essencial para criar uma solução para transacções
  digitais seguras, sem necessidade um terceiro interveniente. Para resolver
  este problema, os mecanismos que hoje são associados com a Blockchain foram
  concebidos e implementados para servir como base para a rede da Bitcoin. De
  uma forma mais especifica, esta foi utilizada como um mecanismo de segurança
  para tornar a Bitcoin uma forma de dinheiro mais transparente e estável, uma
  moeda criptográfica. No entanto, é hoje claro que não conseguiu atingir o seu
  propósito original. De qualquer forma conseguiu desenvolver novos caminhos.
  Trazendo com eles inovação e criatividade utilizada para resolver tanto
  problemas novos como velhos problemas com novas abordagens.

  Enquanto a Blockchain foi inicialmente usada como a base de várias
  criptomoedas, como já referido, esta foi eventualmente reaproveitada para
  disponibilizar uma plataforma que permite a execução de código de uma forma
  autónoma exactamente como foi programado, sem intervenção humana.
  Habitualmente chamados "smart contracts", estes podem ser usados para
  resolver outros conjuntos de problemas devido ao seu comportamento
  transparente, falta de intervenção humana e a sua natureza distribuida. 

  De uma forma prática, a Blockchain é uma tecnologia que permite a criação de
  sistemas que introduzem um conjunto de beneficios em relação aos sistemas
  tradicionais de armazenamento de dados utilizados nos sistemas das
  organizações prestadoras de cuidados de saúde. Custos e riscos associados a
  estes sistemas podem ser reduzidos e a informação pode ser mais transparente
  e fidedigna para todos os participantes. Neste documento são exploradas as
  fundações tecnológicas que permitem esta mudança assim como uma análise das
  mesmas.

  A rede Hyperledger Fabric com transacções privadas e mecanismos avançados de
  segurança foi usada como base para a criação de um sistema de gestão da
  identidade dos utentes. Uma aplicação foi criada que usa smart contracts para
  manipular o ledger da Blockchain. Este sistema será apresentado ao longo
  deste documento e será feita uma discussão sobre a adequação de um sistema
  deste género, neste contexto.

  Concluindo, foi no final, observado que o sistema de Blockchain baseado em
  Hyperledger Fabric é adequado ao proposito que foi definido. Apesar das
  funcionalidades apresentadas por esta plataforma Blockchain serem muito
  focadas em privacidade e segurança, algumas medidas adicionais em torno da
  confidencialidade dos dados tiveram de ser implementadas. Independentemente
  disso, o sistema foi construido com sucesso e conseguiu cumprir com
  requerimentos que foram definidos. A potencial implementação deste sistema
  traria tranparencia, imutabilidade e segurança adicional para utentes e
  profissionais de sáude.

\end{tueSUMARIO}
