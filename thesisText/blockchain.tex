\chapter{Blockchain: Overview and Use Cases}

This chapter will provide a more in-depth exploration of the Blockchain
technology purpose and explore its emerging ecosystem as a growing platform to
solve everyday problems. These Blockchain based systems will be presented,
including some examples of how it is being used in the Healthcare field. This
chapter also discusses why, after analyzing the different implementations and
approaches mentioned in Chapter \ref{background}, the Hyperledger Fabric
Distributed Ledger Platform (\textbf{dlp}) was chosen to achieve the purpose
mentioned in Chapter \ref{introduction}.

\section{Trust in the Network}

Banks used to keep track of their financial transactions writing on a book
called the ledger. This ledger maintains all the transaction information
between the bank and other entities. Nowadays the ledger is not a book but some
form of database that holds the records and has the same function. 

Imagine the following, Joe is on vacation and needs to borrow money from Jane,
his wife. Joe calls Jane to ask for some money and Jane tells him it will send
the money right away. Jane then proceeds to call her account manager in the
bank to transfer money to Joe. Finally Jane calls Joe to tell him the transfer
went through. Joe and Jane need to trust the bank to complete this transaction
and if the bank was ever to be unavailable or the database was corrupted then
all transactions would fail. 

To establish trust between two entities a middle-man with a neutral stake must
be present. The ledger is also the basis of the Blockchain, however this
technology aims to solve the dependency placed upon third parties and aims to
make two different entities trust each other through the process of consensus.
Consensus is a mechanism that appeared as a direct response to this problem,
estabilishing a set of rules that define if a record on the ledger can be
trusted or not. To properly define consensus Blockchain network implementations
must be categorized into two distinct categories.

\section{Permissionless and Permissioned Networks}
%TODO

\section{Dealing with Identity Using Blockchain}
%TODO

\section{Blockchain Applied to Healthcare}

Some companies have already started developing Blockchain applications in the
Healthcare field and established some key partnerships.

Many Blockchain-based solutions are still very early on development or
deployment.  One exception is Guardtime, that has fully deployed their system
in 2008, started cooperating in 2011 and in 2016 announced a partnership with
the Estonian Government, where a million patient records are now secured by the
strategy and, until today, still proves the resilience of the Blockchain
technology, as well as, other advances in cryptography.  Now other companies
like Verizon are becoming interested in this technology for their own purposes.
\cite{GuardTime2018,EstonianGovernmentGuardTime2016}

Another company, Gem, is collaborating with Phillips Healthcare to explore
options in this area, and is opting to solve the interoperability problem with
an additional layer of abstraction they call GemOS.  Factom, another
Blockchain-based service, has also announced a partnership with a major US
medical services provider
HealthNautica.\cite{BlockchainCompHealth2017,FactomPartnership2017}

The use of the Blockchain technology in the health field is expanding. Just
recently a new platform appeared, called Medichain that allows patients to
store their own data in a secure way and give anonymized access to this data to
specialists. Giving data allows for users to gain tokens that represent value.
\cite{MediChain2018}

\section{Hyperledger Fabric}
%TODO
