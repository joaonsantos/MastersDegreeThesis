\chapter{Blockchain: A Practical Overview and Use Cases}

\begin{quote} \emph{"The Blockchain, initially used to solve problems in
  centralized financial systems, has inherent characteristics that make it
  suitable for a variety of use cases. This Chapter will provide a more
  practical exploration of the Blockchain technology showing some of the
  considerations that were taken into account, after analyzing the alternative
  implementations previously mentioned in Chapter~\ref{background}. After
  defining the project requirements the Hyperledger Fabric Distributed Ledger
  Platform , with its focus on enterprise use and true data segregation, was
  chosen as the platform to build a system that seeks to achieve the purpose
  mentioned in Chapter~\ref{introduction}."}
\end{quote}

\section{Trust in a Network}

Blockchain implementations are an emerging structure for distributed computing
systems that provide an accurate and unchangeable history of records written to
a publicly available ledger, even when there is no implicit trust relationship
between the parties involved~\cite{Barclay2017}.

Banks used to keep track of their financial transactions by writing on a book
usually located at the central bank. This book, often called ledger, is written
on whenever a transaction occurs. The ledger acts as a mean of storing all the
transaction details between the bank and other entities. Nowadays banks do not
use the ledger in a book format. Instead the ledger is referred to as the
structure that holds all the transaction information the bank possesses. It is
a structure that keeps the original purpose of recording all the transactions
that are made.

Imagine the following situation- Joe is on vacation and needs to borrow money
from Jane, his wife. Joe calls Jane to ask for some money and Jane tells him it
will send the money right away. Jane then proceeds to call her account manager
to transfer some of her money to Joe. Finally Jane calls Joe to tell him that
she made the request to send money to him.  As seen on
Figure~\ref{fig:centralizedvsdescentralized} Joe and Jane need to use and trust
the account manager and the bank as a middle man in order to complete this
transaction. If the bank was ever to be unavailable, the bank's database was
corrupted or if someone with  privileged access was able to intercept the
transactions from inside the bank then all transactions between Joe and Jane
would fail creating additional costs to all parties involved. 

\begin{figure}[h]
  \centering
  \includegraphics[width=1\linewidth]{imgs/blockchainvscentralizedNetwork.png}
  \caption{\label{fig:centralizedvsdescentralized} A comparison between a
  Centralized Banking System and a Distributed Ledger. (Source: Finance \&
  Development, 2016)}
\end{figure}

It was for a long time necessary, to establish trust between two entities, a
middle-man with a neutral stake. While the ledger is also at the core of the
Blockchain, this technology aims to solve the dependency placed upon third
parties using decentralization and aims to make two different entities trust
each other through constant replication of the ledger, a security mechanism
called consensus.  While consensus has a system performance impact due to the
necessary replication of data, it is a mechanism that establishes a set of
rules that defines if a chain of blocks is considered to be valid or not.
Different Blockchain implementations often use a variety consensus protocols to
balance this trade-off.

\section{Permissionless and Permissioned Blockchain Implementations}

There are three types of computer networks, as seen on Figure
\ref{fig:typesofnetworks}. As previously mentioned in this Chapter, Blockchain
was created out of a desire to solve problems that are displayed in the
centralized financial systems that society is currently based upon.

\begin{figure}[h]
	\centering
	\includegraphics[width=1\linewidth]{imgs/typesofnetworks.png}
  \caption{\label{fig:typesofnetworks} A comparison between different types of
  networks. (Source: Eric Grange, 2016)}
\end{figure}

Put simply, a centralized system is one that is governed by a hierarchical
authority; examples of such being banks, credit card company’s, etc. If you
want to use a Visa card you must request access from Visa and be approved. At
any time your access to that line of credit and your funds may be made
unavailable to you and your access permanently revoked~\cite{Dreifuerst2018}.

In contrast, distributed systems are based upon the philosophy that processing
is shared across multiple nodes even if the decisions themselves may still be
centralized and use complete system state knowledge of the network. Finally, a
decentralized system is one where no single node can make a decision
individually, instead relying on the other participants to reach an agreement
and make a decision, as no single node has a complete system state knowledge.
With this in mind, a decentralized system is seen as a subset of a distributed
system.

A Blockchain is distributed by design but can be based on either a
decentralized network or a distributed network creating two major
implementation categories.

Permisionless Blockchain implementations, like the Bitcoin's and Ethereum's
Blockchain for example, have no barrier to entry. This means that anyone can,
in theory, participate in the network, write into it as a result of mining and
store data in the ledger sharing the work needed to maintain the network.
Permisionless implementations have some strengths; such as being completely
open, transactions are transparent while also being able to offer anonymity or
pseudo-anonymity and also take away the need for system administrators or
central servers since the network is based on peer-to-peer technology, creating
reduced costs to maintain and deploy \textbf{Ðapps}. On the other hand, they
are slow because every node must participate in consensus, they operate without
clear legal rules and are trust-free, meaning that there is no responsible entity
if data loss or damages affect systems based on this implementation.

Permisionless Blockchain implementations were the first to appear and while
some industries saw benefits in using the technology, some saw drawbacks to
adopting it in enterprise-grade systems~\cite{Gopinath2016} due to this
unregulated nature.

The permissioned variant of this technology has some clear advantages for
enterprise; it is faster because consensus is done by a set of nodes instead of
the entire network, can fall back on the legal system because it features an
identity service and is immediately trustworthy if the entities that manage the
network are considered themselves trustworthy because it is auditable and there
is a legal responsible entity or entities that manage the network. However,
costs when compared to the permissionless variant are higher due to having the
need for a system administrator and servers to manage the network, featuring a
private membership meaning that they are closed to the general public and
managed by a set of entities and are a compromise between the original vision
of a completely decentralized network and enterprise needs and concerns. 

Enterprises benefit greatly from the immutability of the Blockchain
architecture, in that all records cannot be changed. By adding authorised
identity services onto Blockchain, they can meet the regulatory needs of their
industries, by allowing the network to be auditable and assets to be traceable,
falling back to laws if a dispute between participating entities
occurs~\cite{Barclay2017}.

With these two variants in mind, Section~\ref{choosingHyperledger} provides an
insight into the characterists deemed important to achieve the estabilished
purposes that this project set out to achieve and Section~\ref{choosePlatform}
explains the decision processes behind the choice of a permissioned
implementation of the Blockchain technology being used to build a system for
managing patients in Healthcare.

\section{System Requirements and Starting Blocks}\label{choosingHyperledger}

After considering the project goals of investigating the suitability of a
Blockchain based system to manage patients in Healthcare, it was decided to
build a prototype of a system that would represent the network, albeit on a
smaller scale. The insights gained from developing a simple working system
would enable benefits and risks of the approach to be identified, and
opportunities for further research to be laid out.

The requirements for this project are deemed to be as follows:

\renewcommand{\labelenumi}{\Roman{enumi}.}
\begin{enumerate}
  \item The system must allow a patient to opt into the network securely.
  \item The system must allow a patient to record his medical data under the
    approval of an administrator.
  \item The system must keep information confidential, transparent and have
    high availability.
  \item The system must provide the patient with the ability to share his data
    with another entity participating in the network, for example sharing
    information with a doctor.
  \item The system must allow data  of patient to be erased in some manner if
    he wishes to do so, in order to comply with European privacy laws.
\end{enumerate}

This section explores some Blockchain deployments and discusses the key
benefits and drawbacks of these, in regards to the suitability of usage as a
platform for developing the system.

Blockchain platforms often have different goals even tough they originate from
the realization that full centralization has major drawbacks. Ranging from open
networks, such as Ethereum which anyone can join and use, to permissioned
distributed ledgers, which can be run publicly or privately but are only open
to access and participation through a membership service, such as Hyperledger
Fabric and Hyperledger Indy.

\subsection{A Decentralized Open Platform - Ethereum}

Ethereum is a permissionless Blockchain implementation. It is a platform that
lets anyone build and use decentralized applications commonly named
\textbf{Ðapps}. It is an open-source project developed primarily by the
Ethereum Foundation and was designed to be adaptable and flexible, in contrast
to Bitcoin's Blockchain that only records financial
transactions~\cite{EthereumDocs2018}.

It features a friendly programming language called Solidity that is influenced
by C++, Python and Javascript that is designed to allow an easy way for
developers to create new applications on the Ethereum platform with code of
arbitrary algorithmic complexity in a turing complete language. Smart Contract
application code targets the Ethereum Virtual Machine, which is then deployed
to the Blockchain via a local Ethereum node~\cite{Wood2017,Barclay2017}.

At the heart of Ethereum is the Ethereum Virtual Machine (\textbf{evm}) as seen
on Figure~\ref{fig:evm} and, like any Blockchain, Ethereum also includes a
peer-to-peer network protocol. The Ethereum Blockchain database is maintained
and updated by many nodes connected to the network. Each and every node of the
network runs the EVM and executes the same instructions in order to maintain
consensus across the entire Blockchain. Decentralized consensus gives Ethereum
a high degree of fault tolerance, ensures zero downtime, and makes data stored
on the Blockchain forever unchangeable and
censorship-resistant~\cite{EthereumDocs2018}.

\begin{figure}[h]
  \centering
  \includegraphics[width=1\linewidth]{imgs/ethereumVirtualMachine.png}
  \caption{\label{fig:evm} A diagram of where \textbf{evm} fits into the
  Ethereum Platform (Original: Vaibhav Saini, 2018)}
\end{figure}

Users must pay a small transaction fee to the network each time they execute a
transaction. This protects the Ethereum Blockchain from frivolous or malicious
computational tasks, like \textbf{ddos} attacks or infinite loops. The sender
of a transaction must pay for each step of the “program” they activated,
including computation and memory storage. These fees are paid in amounts of
Ethereum’s native value-token, ether, and then these transaction fees are
collected by the nodes that validate the network commonly called miners - which
are nodes in the network that receive, propagate, verify, and execute
transactions. Ethereum currently uses a proof of work (\textbf{pow}) based
consensus algorithm but plans to change to a proof of stake (\textbf{pos})
algorithm due to environmental and financial concerns as well as reduced
centralization risks~\cite{EthereumDocs2018,EthereumPOSFAQ2018}.

Ethereum has a live production network called “mainnet” available for any
developer to deploy applications to, as well as three test networks. "Ropsten”
is based on a \textbf{pow} algorithm while "Rinkeby" and "Kovan" are based on a
Proof of Authority~\footnote{In Proof of Authority based networks, transactions
and blocks are validated by approved accounts, known as validators. Validators
run software allowing them to put transactions in blocks. The process is
automated and does not require validators to be constantly monitoring their
computers. It does, however, require maintaining the authority node
uncompromised.} (\textbf{poa}) and all of them are publicly available and free
to use~\cite{Barclay2017,EthereumTestNetworks2018}.

Ethereum has had some unforeseen problems along the way, namely the DAO heist
where a hacker took advantage of a bug in a smart contract to steal a great sum
of money. With Ethereum frequently reaching full transaction capacity, scaling
solutions are the next big investment~~\cite{ethereumScalability2018}.

There are a few proposed solutions by Buterin - sharding aims to avoid every
node processing all data in order to verify and process a transaction. When
transactions are initiated they will not be directed to all the nodes but would
instead only be directed to those depending on the shard in question.  Another
solution is off chain computation where a layer apart from the Blockchain is
created and where all the computation or solving of a complex mathematical
equation takes place. This would not only take the load off the Ethereum
Blockchain but also help decrease the cost of transaction verification and
processing. This mechanism would ensure that the tasks that account for slower
transaction speeds on the Ethereum’s Blockchain do not affect the whole
network. Finally, to avoid every node having the need to download the entirety
of the Blockchain's data, the complete picture can be stored on cloud and each
node only has to store and load relevant
data~\cite{ethereumBlogScalability2018}.

\subsection{A Permissioned Distributed Ledger Platform - Hyperledger Fabric}

Hyperledger Fabric is a platform for distributed ledger solutions featuring a
modular architecture. It provides developers with a permissioned platform
targeted at business and enterprise use cases that supports pluggable
implementations of different components to accommodate the complexity and
intricacies that exist across the economic ecosystem. It is an open source
project initially commited by IBM  and estabilished under the Linux Foundation,
being developed by over 44 organizations and more than 250
members~\cite{HyperledgerFabricDocs2017,HyperledgerGrowth2018}.

It supports the creation of smart contracts, commonly called "chaincode" in
Fabric, that are authored in general-purpose programming languages such as
Java, Go and Node.js rather than constrained domain-specific languages
(\textbf{dsl}). Chaincode runs in a secured Docker container isolated from the
peer process which consented its installation. Chaincode also initializes and
manages ledger state through transactions as well as the world state.

At the heart of Fabric is the permissioned distributed ledger that provides a
way to secure the interactions among a group of entities that have a common
goal but which may not fully trust each other. By relying on the identities of
the participants, a permissioned ledger platform can use a more traditional
crash fault tolerant (\textbf{cft}) or byzantine fault tolerant (\textbf{bft})
consensus protocols that do not require mining or an associated currency in
order to achieve consensus.

\begin{figure}[h]
  \centering
  \includegraphics[width=1\linewidth]{imgs/executeOrderValidate.png}
  \caption{\label{fig:executeorder} Execute-order-validate architecture of
  Fabric (Source: IBM, 2018)}
\end{figure}

Fabric introduces the execute-order-validate Blockchain architecture as shown
on Figure~\ref{fig:executeorder} and does not follow the standard order-execute
design illustrated on Figure~\ref{fig:orderexecute} \cite{Androulaki2018}. 

\begin{figure}[h]
  \centering
  \includegraphics[width=0.8\linewidth]{imgs/orderExecuteArchitecture.png}
  \caption{\label{fig:orderexecute} Order-execute architecture in Replicated
  Services Like Ethereum (Source: IBM, 2018)}
\end{figure}

The order-execute architecture is conceptually simple, leading it to be
currently widely implemented in replicated services such as Blockchain. In this
architecture the transactions are executed sequentially on all peers which
limits the maximum number of simultaneous transactions that can be achieved.
Additionally a Denial of Service (\textbf{dos}) attack can be mounted just by
deploying a slow performing smart contract or one with an infinite loop to the
network since the Blockchain forms a distributed computing engine.  To cope
with this issue, public programmable Blockchains with an associated
cryptocurrency, account for the execution cost of executing of the program.

Chaincode in Fabric consists of two components - the code itself, which
describes the logic of the program running in the execution phase, and the
endorsement policy that describes how a specific chaincode transaction is
validated. For example, a typical endorsement policy lets the chaincode specify
the endorsers for a transaction in the form of a set of peers that are
necessary for endorsement and subsequent successful
validation~\cite{Androulaki2018}.

In Fabric all nodes that participate in the network have an identity, as
provided by a modular membership service provider (\textbf{msp}).  A
\textbf{msp} is a component that aims to offer the abstraction of a membership
operation architecture meaning that all identities are only allowed to
participate if verified to be considered trustworthy.  The \textbf{msp}
maintains the identities of all nodes in the system and is responsible for
issuing credentials that are used for node authentication and authorization. An
\textbf{msp} may define their own notion of identity, and the rules by which
those identities are governed and authenticated using signature generation and
verification~\cite{HyperledgerFabricDocs2017}.

Fabric also assigns different roles to peers. Nodes in Fabric network take up
one of three roles:

\begin{itemize}
  \item Clients submit transaction proposals for execution, help orchestrate
    the execution phase, and broadcast transactions for ordering.

  \item Peers execute transaction proposals and validate transactions.  All
    peers maintain the ledger, where all transactions  are recorded in the form
    of a hash chain, as well as the state, a succinct representation of the
    latest ledger state. Not all peers execute all transactions.

  \item Ordering Service Nodes (\textbf{osn}) or orderers are the nodes that
    collectively form the ordering service. In short, the ordering service
    establishes the total order of all transactions in Fabric, where each
    transaction contains state updates and dependencies computed during the
    execution phase, along with cryptographic signatures of the endorsing peers
    defined in the endorsing policy of the transaction. Orderers are entirely
    unaware of the application state, and do not participate in the execution
    nor in the validation of transactions. This design choice renders consensus
    in Fabric as modular as possible and simplifies the replacement of
    consensus protocols in Fabric. 
\end{itemize}

Looking ahead, Hyperledger Fabric will continue to focus on privacy and
confidentiality with v1.2 being recently released, v1.3 and 1.4 expected to be
out this year with further emphasis on these aspects in a regular quarterly
cadence~\cite{hyperledgerRoadmap2018}.

\section{Choosing a Platform}\label{choosePlatform}

Ethereum is a popular platform on its own right and has certainly paved the way
for Blockchain to be used as a platform that can be extended and built upon. It
has a growing learning ecosystem and community. It is easy to start interacting
with the network as anyone is able to simply download a client and connect to
it.  Thanks to the Solidity smart contract language being targeted for the
specific purpose of authoring smart contracts it is a platform easy to develop
for after the initial learning barrier of the \textbf{dsl}. 

Ethereum is being used in a great deal of projects around the world proving its
stability and suitability in a wide variety of use cases. On the other hand,
handling patients medical data is a great responsibility as hospitals and
clinics must obey the regulatory laws regarding privacy and usage of this data.

It is also worth noting that while Ethereum can handle private data exchange by
building upon it, as show by Barclay, it was not designed with this intent in
mind, therefore these middle ground solutions can prove to be unwise to use at
scale given Ethereum's past problems with scalability.

Fabric, like Ethereum, was built with the intention of being a general
purpose use Blockchain. It provides developers with the tools needed to build
any system they can imagine but is clearly focused on making organizations feel
at ease by providing an identity service and a controlled environment,
therefore avoiding the same fate as \textbf{IoT} devices where the lack of
security regulations and ambiguity in how data collected by these devices is
handled has stopped these to be used in any official capacity.

Fabric also has good amount of development tools that are now maturing and a
good learning environment with ample documentation about every aspect important
for a developer looking to get started into it. Fabric is being backed by the
Linux Foundation and IBM that lends credibility to the project.

Regarding Fabric's features, it lends itself very well to fulfill the project
requirements. With Fabric's channels and private data segregation at peer level
it lends itself well to fulfill all the requirements that were laid out for
this project. Adding to this, many Blockchain based projects in the Healthcare
field are using permissioned networks due to the concerns regarding the privacy
of the patients while retaining the key benefits of Blockchain such as
immutability and decentralization.

Ultimately it was decided to use Hyperledger Fabric as the platform on which to
build the prototype project upon.
