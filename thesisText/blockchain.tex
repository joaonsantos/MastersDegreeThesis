\chapter{Blockchain: Overview and Use Cases}

\section{Blockchain for Identity Management in Healthcare: Use Cases}

Some companies have already started developing Blockchain applications in the
Healthcare field and established some key partnerships.

Many Blockchain-based solutions are still very early on development or
deployment.  One exception is Guardtime, that has fully deployed their system
in 2008, started cooperating in 2011 and in 2016 announced a partnership with
the Estonian Government, where a million patient records are now secured by the
strategy and, until today, still proves the resilience of the Blockchain
technology, as well as, other advances in cryptography.  Now other companies
like Verizon are becoming interested in this technology for their own purposes.
\cite{GuardTime2018,EstonianGovernmentGuardTime2016}

Another company, Gem, is collaborating with Phillips Healthcare to explore
options in this area, and is opting to solve the interoperability problem with
an additional layer of abstraction they call GemOS.  Factom, another
Blockchain-based service, has also announced a partnership with a major US
medical services provider
HealthNautica.\cite{BlockchainCompHealth2017,FactomPartnership2017}

The use of the Blockchain technology in the health field is expanding. Just
recently a new platform appeared, called Medichain that allows patients to
store their own data in a secure way and give anonymized access to this data to
specialists. Giving data allows for users to gain tokens that represent value.
\cite{MediChain2018}

