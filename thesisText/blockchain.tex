\chapter{Blockchain: A Practical Overview and Use Cases}

\begin{quote} \emph{"The Blockchain, initially used to solve problems in
  centralized financial systems, has inherent characteristics that, depending
  on the implementation, make it suitable for a variety of use cases. This
  Chapter will provide a more in-depth exploration of the Blockchain technology
  showing some of the considerations that were taken into account, after
  analyzing the alternative implementations previously mentioned in
  Chapter~\ref{background}, to choose the Hyperledger Fabric Distributed Ledger
  Platform to achieve the purpose mentioned in Chapter~\ref{introduction}.
  Finally some pratical implementations of Blockchain based solutions created
  to solve problems in the Healthcare field or problems managing entities are
presented that provide context and show the current state of this technology in
a production environment, providing an overview of both its shortcomings and
successes thus far."} \end{quote}

\section{Trust in a Network}

Blockchain implementations are an emerging structure for distributed computing
systems that provide an accurate and unchangeable history of transactions
written to a publicly available ledger or record, even when there is no trust
relationship between the parties involved~\cite{Barclay2017}.

Banks used to keep track of their financial transactions by writing on a book
called the ledger. The ledger would be written on when a new transaction
occurred, storing all details of the transactions that occurred between the
bank and other entities. Nowadays the ledger is not a book, instead being
database storing information in a server, with the same function of recording
all the transactions that are made.

Imagine the following, Joe is on vacation and needs to borrow money from Jane,
his wife. Joe calls Jane to ask for some money and Jane tells him it will send
the money right away. Jane then proceeds to call her account manager in the
bank to transfer money to Joe. Finally Jane calls Joe to tell him the transfer
went through.  As seen on Figure~\ref{fig:centralizedvsdescentralized} Joe and
Jane need to use and trust the bank as a middle man in order to complete this
transaction. If the bank was ever to be unavailable, the database was corrupted
or if someone with  privileged access to the central database and malicious
intent was able to intercept the transactions from inside the bank then all
transactions between Joe and Jane would fail creating additional costs to all
parties involved. 

\begin{figure}[h]
  \centering
  \includegraphics[width=1\linewidth]{imgs/blockchainvscentralizedNetwork.png}
  \caption{\label{fig:centralizedvsdescentralized} A comparison between a
  Centralized Banking System and a Distributed Ledger. (Source: Finance \&
  Development, 2016)}
\end{figure}

It was for a long time necessary, to establish trust between two entities, a
middle-man with a neutral stake in the transaction. While the ledger is also at
the core of the Blockchain, this technology aims to solve the dependency placed
upon third parties using decentralization and aims to make two different
entities trust each other through constant replication and through a process
called consensus.  Consensus is a mechanism that establishes a set of rules
that define if a chain of blocks is considered valid or not and depending on
the Blockchain implementation works in a different manner. With this in mind,
Blockchain implementations can be categorized into two distinct categories.

\section{Permissionless and Permissioned Blockchain Implementations}

There are three types of systems in computing, as seen on Figure
\ref{fig:typesofnetworks}. Each has their own advantages and disadvantages,
each suiting a specific use case. As previously mentioned in this Chapter,
Blockchain was created out of a desire to solve problems that are displayed in
the centralized financial systems that society is currently based upon.

\begin{figure}[h]
	\centering
	\includegraphics[width=1\linewidth]{imgs/typesofnetworks.png}
  \caption{\label{fig:typesofnetworks} A comparison between different types of
  networks. (Source: Eric Grange, 2016)}
\end{figure}

Put simply, a centralized system is one that is governed by a hierarchical
authority; examples of such being banks, credit card company’s, etc. If you
want to use a Visa card you must request access from Visa and be approved. At
any time your access to that line of credit and your funds may be made
unavailable to you and your access permanently revoked~\cite{Dreifuerst2018}.

In contrast, distributed systems are based upon the philosophy that processing
is shared across multiple nodes even if the decisions themselves may still be
centralized and use complete system state knowledge of the network. Finally, a
decentralized system is one where no single node can make a decision
individually, instead relying on the other participants to reach an agreement
and make a decision, as no single node has a complete system state knowledge.
With this in mind, a decentralized system is seen as a subset of a distributed
system.

A Blockchain is distributed by nature but internally how the system handles the
distributed nature creates two major implementation categories.

Permisionless Blockchain implementations, like the Bitcoin's and Ethereum's
Blockchain for example, have no barrier to entry. This means that anyone can,
in theory, participate in the network, write into it as a result of mining and
store data in the ledger. Permisionless implementations have some strengths;
such as being completely open, transactions are transparent while offering
anonymity or pseudo-anonymity and also take away the need for system
administrators or central servers since the network is based on peer-to-peer,
creating reduced costs to maintain and deploy \textbf{Ðapps}. On the other
hand, they are slow because every node must participate in consensus, they
operate without clear legal rules and are trust-free, meaning that no one is
really responsible if some bugs or corruption happened to cause damages to
systems based on this implementation.

Permisionless Blockchains were the first to appear and while some industries
saw benefits in using the technology, some saw drawbacks to using it in
enterprise-grade systems~\cite{Gopinath2016}.

The permissioned variant of this technology has some clear advantages for
enterprise; it is faster because consensus is done by a set of nodes instead of
the entire network, can fall on legal back on the legal system because it
features an identity service and is immediately trustworthy if the entities
that manage the network are considered themselves trustworthy because it is
auditable and there is a legal responsible entity or entities that manage the
network. However, costs when compared to the permissionless variant are higher
due to having the need for a system administrator and servers to manage the
network, feature a private membership meaning that they are closed to the
general public and managed by a set of entities and are a compromise between
the original vision of a distributed network and enterprise needs and concerns. 

Enterprises benefit greatly from the immutability of the Blockchain
architecture, in that all records cannot be changed. By adding authorised
identity services onto Blockchain, they can meet the regulatory needs of their
industries, by allowing the network to be auditable and assets to be traceable,
falling back to laws if a dispute between participating entities
occurs~\cite{Barclay2017}.

With these two variants in mind, Section~\ref{choosingHyperledger} provides an
insight into the characterists deemed important to achieve the estabilished
purposes that this project set out to achieve and explains the decision
processes behind the choice of a permissioned implementation of the Blockchain
technology being used to build a system for managing patients in Healthcare. In
Section~\ref{blockchainHealthcare} some previous Blockchain based solutions in
the Healthcare context are analyzed to take away key learnings that may be used
to improve or be aware of in designing the system.

\section{System Requirements and Starting Blocks}\label{choosingHyperledger}

After considering the project goals of investigating the suitability of a
Blockchain based system to manage patients in Healthcare, it was decided to
build a prototype of a system that would represent the network, albeit on a
smaller scale. The insights gained from developing a simple working system
would enable benefits and risks of the approach to be identified, and
opportunities for further research to be laid out.

The requirements for this project are deemed to be as follows:

\renewcommand{\labelenumi}{\Roman{enumi}.}
\begin{enumerate}
  \item The system must allow a patient to opt into the network securely.
  \item The system must allow a patient to record his medical data under the
    approval of an administrator.
  \item The system must keep information confidential, transparent and have
    high availability.
  \item The system must provide the patient with the ability to share his data
    with another entity participating in the network, for example sharing
    information with a doctor.
  \item The system must allow data  of patient to be erased in some manner if
    he wishes to do so, in order to comply with European privacy laws.
\end{enumerate}

This section  explores some Blockchain deployments and discusses the key benefits and drawbacks of these, in regards to the suitability of usage as a platform
for developing the system.

Blockchain platforms often have different goals even tough they originate from
the realization that full centralization has major drawbacks. Ranging from open
networks, such as Ethereum which anyone can join and use, to permissioned
Blockchains, which can run publicly or privately but are only open to access
and participation through a membership service, such as the Hyperledger Fabric
and Hyperledger Indy.

\subsection{An Open Platform - Ethereum}

Ethereum is a permissionless Blockchain implementation. It is a platform that lets anyone build and use decentralized applications commonly named \textbf{Ðapps}. It is an open-source project developed primarily by the Ethereum Foundation and was designed to be adaptable and flexible, in contrast to Bitcoin's Blockchain that only records financial transactions~\cite{EthereumDocs2018}.

 It features a friendly programming language called Solidity that is influenced by C++, Python and Javascript that is designed to allow an easy way for developers to create new applications on the Ethereum platform with code of arbitrary algorithmic complexity in a turing complete language. Smart Contract application code targets the Ethereum Virtual Machine, which is then
deployed to the Blockchain via a local Ethereum node~\cite{Wood2017,Barclay2017}.

At the heart of Ethereum is the Ethereum Virtual Machine (\textbf{evm}) and, like any Blockchain, Ethereum also includes a peer-to-peer network protocol. The Ethereum Blockchain database is maintained and updated by many nodes connected to the network. Each and every node of the network runs the EVM and executes the same instructions in order to maintain consensus across the entire Blockchain. Decentralized consensus gives Ethereum a high degree of fault tolerance, ensures zero downtime, and makes data stored on the Blockchain forever unchangeable and censorship-resistant~\cite{EthereumDocs2018}.

Users must pay a small transaction fee to the network each time they execute a transaction. This protects the Ethereum Blockchain from frivolous or malicious computational tasks, like \textbf{ddos} attacks or infinite loops. The sender of a transaction must pay for each step of the “program” they activated, including computation and memory storage. These fees are paid in amounts of Ethereum’s native value-token, ether, and then these transaction fees are collected by the nodes that validate the network commonly called miners - which are nodes in the network that receive, propagate, verify, and execute transactions. Ethereum currently uses a \textbf{pow} based consensus algorithm but plans to change to a proof of stake (\textbf{pos}) algorithm due to environmental and financial concerns as well as reduced centralization risks~\cite{EthereumDocs2018,EthereumPOSFAQ2018}.

Ethereum has a live
production network called “mainnet” available for any developer to deploy applications to,
as well as three test networks. "Ropsten” is based on a \textbf{pow} algorithm while "Rinkeby" and "Kovan" are based on a Proof of Authority (\textbf{poa}) and all of them are publicly available and free to use~\cite{Barclay2017,EthereumTestNetworks2018}.

Ethereum has had some unforeseen problems along the way, namely the DAO heist. Also with the Ethereum blockchain reaching 1 million transactions per day, and both Ethereum and other blockchain projects frequently reaching their full transaction capacity, the need for scaling progress is becoming more and more clear and urgent. With sharding the aim is to make sure that every node doesn’t have to process all the data to verify and process a transaction. The complete blockchain state can be separated and divided into different shards. Once the blockchain is divided into shards the need to store all the data by every node can be compromised as every node as only a part of the state would be required to be stored by particular nodes. Now when the transactions are initiated on the blockchain they will not be directed to all the nodes but only to those depending on the shard they affect. It is categorized in such a manner that each shard only processes a part of the entire state.

Off chain computation is a concept that allows for a layer apart from the Blockchain where all the computation or solving of complex mathematical equation takes place. This would not only take the load off the Ethereum Blockchain but also help decrease the cost of transaction verification and processing. This works in a way that not every node in the Blockchain participates in the computations rather particular participants in the blockchain perform the complex computations along with a deposit. If the solution is correct then the participant gets the reward and also keeps the deposit. If the solution is incorrect the deposit is forfeited. The offchain computation calls for a verification and therefore there are verifiers who verify the solutions. This mechanism would ensure that the tasks that account for slower transaction speeds on the Ethereum’s blockchain are performed separately by adding a seperate layer to the blockchain. 

One of the biggest challenges in achieving fast transaction speed for Ethereum is to cut down the data and load which is increasing on the nodes. Currently every node has to store all the data on the blockchain. Therefore a decentralized storage mechanism that share the load of the data could help make the transaction processing more efficient and quicker. A decentralized data storing mechanism which can be integrated to the blockchain allowing the nodes to store only that data which is frequently and locally used will take the load off the nodes and make transaction processing faster. The complete blockchain data can be stored on cloud and each node only has to store and load relevant data for particular transaction processing which are more frequent. This kind of storage mechanism can make Ethereum scalable and there’s a need to focus on such solutions.
%TODO
% Talk about Ethereum vs Fabric vs Indy

\subsection{An Overview of Hyperledger Fabric}
%TODO
