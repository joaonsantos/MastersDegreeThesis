\section{Identity in Healthcare}

Originally records of a patient were stored in paper, a physical format.
Thanks to the advent of the computers more and more records are stored on a
digital format and the Electronic Health Record (EHR) was created.
\cite{Marquez2017}  This benefits handling of information between the patient
and the medical professionals and medical institutions.\cite{ONCoordinator2017}
But first we must discuss what is defined as identity in this specific case.

Identity is a construct that depends on the context.  Identity can be defined
as the characteristics determining who or what a person is.  In this paper we
define identity as the set of characteristics that determine who is the patient
in the given Healthcare ecosystem they belong to, such as, the name, the age,
the cellphone number, the gender and the birth date of the patient.  Electronic
Health Records encapsulate this information in digital format, however, they
are usually represented in a format according to the Information System they
were designed to work with.

To enable interoperability, standards for EHRs were created and many failed to
bring the much needed consensus that was required for interoperability between
different Information Systems in different institutions. \cite{Eichelberg2006}
Health Level 7 has done much work to be recognized in many countries and is
quickly being implemented in many countries to allow for joint efforts between
organizations. \cite{HL7Anual2016}

Even with these advances in mind, the nature of many clinics and hospitals
Information Systems makes the management of their patients identity a very
cumbersome, costly and risky affair to handle.  Security in a connected age,
where internet is easily available, is lagging behind and presenting some
problems.  There is also the question of transparent use of information by the
organizations that store it.
