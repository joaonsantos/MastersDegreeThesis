%!TEX root = main.tex
\chapter{State of the Art}
\section{Background} \label{background}

While Blockchain is not a new concept at this point, it is an evolving
technology that is being used to solve old problems with new approaches. This
section will explore the Blockchain technology origins and history, some of its
different implementations and a brief history to the identity problem is
presented.

\subsection{Blockchain Technology}

A Blockchain can be many things. It can refer to the Bitcoin Blockchain,
alternative implementations or forks of the Bitcoin Blockchain called Altchains
or even platforms that allow execution of code in an autonomous manner, exactly
as it was programmed, with no human intervention.  It is a continuously growing
list of records, written in the ledger, a structure where records are written,
that is being replicated across a network of devices in opposition to having a
single central record history, making it a good example of a distributed
database.  \cite{Wood2017}

The main design goal of the Blockchain is security and to fulfill this purpose
it uses techniques such as cryptography and digital signatures to not only
verify the authenticity of records but also read or write access to the
network.

Unlike a conventional central data storage, where only a single entity keeps a
copy of the underlying database, the ledger of the Blockchain is replicated
across any number of nodes.  Not every participant has the same ability to
interact with the ledger and in this respect a Blockchain can be permissionless
or permissioned. In a permissionless Blockchain every node of the network can
write in the Blockchain whereas in a permissioned Blockchain only a select
group of entities have access to writing in the ledger, making the permissioned
version, by default, secure if the entities themselves are secure and
considered trustworthy.

How does a permissionless Blockchain maintain security if every participant has
access to writing on it, including potentially malicious parties?

Take for example the Bitcoin Blockchain that uses a peer-to-peer network to
avoid meddling from a financial institution or a third party in a financial
transaction. Given that participating nodes in the network can belong to
different and often competing parties, there is no implied trust between them,
so the Blockchain needs a mechanism to ensure the integrity of the ledger and
prevent malicious meddling from interested parties or to avoid a central
authority.\cite{Barclay2017}

To solve this problem, consensus mechanisms are used differently, depending on
its implementation, but having, at its core, a solution to create immutable
records and ensure security.  In Bitcoin Blockchain’s case, consensus is
reached by the longest chain rule where the longest chain not only serves as
proof of the sequence of events witnessed, but as proof that it came from the
largest pool of computing power.\cite{Baars2016}

While the first Blockchain was conceptualized as the public ledger for the
Bitcoin cryptocurrency in 2008 by Satoshi Nakamoto and implemented in 2009,
many are now using it as a foundation across many application areas such as
identity management, traceability and asset management.  Thanks to the roaring
success of Bitcoin and the increasingly apparent use cases that the Blockchain
can provide, the public awareness of it is rising and it is quickly becoming a
technological foundation in our economic and social systems.
% Need References for this

\subsubsection{Ethereum}

Bitcoin is getting media coverage almost everyday and public awareness in
cryptocurrencies in general is rising.  Some people are considering
cryptocurrencies and the Blockchain, to be essentially the same technology and,
while that may have been somewhat true not so long ago, Blockchain technology
is starting to be used in a plethora of ways.

Ethereum is an open-source platform based on the Blockchain technology that
enables developers to build and deploy Decentralized Applications
(\textit{DAPPs}).  Ethereum is being developed by the Ethereum Foundation and
was first discussed by Buterin in 2013.  Ethereum intends to provide a
Blockchain with a built-in programming language that is used to create
\textit{Smart contracts}.  \cite{Wood2017}

These contracts are used to describe the logic of any system that developers
can imagine and, when created, can then be deployed to the Blockchain where
they execute as “autonomous agents”.  Thanks to these tools it is safe to say
that long gone are the days where building Blockchain applications required a
complex background in coding cryptography, mathematics as well as significant
resources.\cite{Wood2017,BlockGeeks2017}

Ethereum Blockchain is a permissionless Blockchain, and thus, it must have a
consensus mechanism to ensure the validation process of every record and, in
turn, ensure security and immutability. While other implementations of the
Blockchain have different consensus mechanics, in Ethereum’s case, all
participants have to reach consensus over the order of all transactions that
have taken place. If a definitive order cannot be established then a
double-spend might have occurred.

\subsubsection{Fabric}

Hyperledger Fabric (HLF) is part of the
\href{http://www.hyperledger.org/projects/fabric}{Hyperledger} project started
in December 2015 by the Linux Foundation, and is an open-source
developer-focused community of communities focused on the development of
enterprise-grade, open-source Blockchain-based solutions.  Fabric is an
implementation of a Distributed Ledger Platform (DLP) under the Hyperledger
umbrella.  \cite{Cachin2016}

HLF’s initial commit was contributed by IBM and written in Go language.  It is
a permissioned Blockchain and its main design goal was to surpass previous
Blockchain implementation limitations, such as, lack of true private
transactions and confidential contracts.

This is achieved thanks to assigning peers in the network three distinct roles
and by offering the ability to create channels each with its own private
ledger.  A peer can have the role of endorser, committer or consenter or
sometimes multiple roles.  HLF is intended as a foundation for developing
applications in a modular fashion, opting for a plug-and-play approach to
various components. \cite{HyperledgerFabricDocs2017}

HLF, as discussed, also allows the creation of smart contracts which can be
written in Chaincode.  As this Blockchain's key operational requirement is
privacy, true private transactions and confidential contracts can exist and are
a great asset for a business environment where sensitive information is
necessary and disclosed often.  Thanks to its modular approach consensus
protocols are no longer hard-coded and trust models can be repurposed.

\subsubsection{Burrow}

Hyperledger Burrow (HLB) is also part of the Hyperledger project and its
development started in 2014 by Monax and sponsored by Intel. It is a
permissionable smart contract machine written in Go and offers a modular
Blockchain client with a permissioned smart contract interpreter built, in
part, to the specification of the Ethereum Virtual Machine (EVM) and the client
has, essentially, three main components, the consensus engine, the permissioned
EVM and the Remote Procedure Call (RPC) gateway.
\cite{Kuhlman2017,HyperledgerBurrow2017}

HLB has its own Consensus Engine, the Byzantine fault-tolerant Tendermint
protocol.  The Tendermint protocol is an open-source effort that allows high
performance in solving the consensus problem and also has a flexible interface
for building arbitrary applications above the consensus, as well as, a suite of
tools for deployments and their management. \cite{Buchman2016}
%
%#===========================Identity in
%Healthcare===================================#%

\subsection{Identity in Healthcare} Originally records of a patient were stored
in a physical format.  Thanks to the advent of the computers more and more
records are stored on a digital format and the Electronic Health Record (EHR)
was created.  This benefits handling of information between the patient and the
medical professionals and medical institutions. But first we must discuss what
is defined as identity in this specific case.

Identity is a construct that depends on the context.  Identity can be defined
as the characteristics determining who or what a person is.  In this paper we
define identity as the set of characteristics that determine who is the patient
in the given Healthcare ecosystem they belong to, such as, the name, the age,
the cellphone number, the gender and the birth date of the patient.  Electronic
Health Records encapsulate this information in digital format, however, they
are usually represented in a format according to the Information System they
were designed to work with.

To enable interoperability, standards for EHRs were created and many failed to
bring the much needed consensus that was required for interoperability between
different Information Systems in different institutions.  Health Level 7 has
done much work to be recognized in many countries and is quickly being
implemented in many countries to allow for joint efforts between organizations.

Even with these advances in mind, the nature of many clinics and hospitals
Information Systems makes the management of their patients identity a very
cumbersome, costly and risky affair to handle.  Security in a connected age,
where internet is easily available, is lagging behind and presenting some
problems.  There is also the question of transparent use of information by the
organizations that store it.
%
%#===========================Related Work===================================#%

\subsection{Blockchain for Identity Management in Healthcare: Use Cases} Some
companies have already started developing Blockchain applications in the
Healthcare field and established some key partnerships.

Many Blockchain-based solutions are still very early on development or
deployment.  One exception is Guardtime, that has fully deployed their system
in 2008, started cooperating in 2011 and in 2016 announced a partnership with
the Estonian Government, where a million patient records are now secured by the
strategy and, until today, still proves the resilience of the Blockchain
technology, as well as, other advances in cryptography.  Now other companies
like Verizon are becoming interested in this technology for their own purposes.
\cite{GuardTime2018,EstonianGovernmentGuardTime2016}

Another company, Gem, is collaborating with Phillips Healthcare to explore
options in this area, and is opting to solve the interoperability problem with
an additional layer of abstraction they call GemOS.  Factom, another
Blockchain-based service, has also announced a partnership with a major US
medical services provider
HealthNautica.\cite{BlockchainCompHealth2017,FactomPartnership2017}

The use of the Blockchain technology in the health field is expanding. Just
recently a new platform appeared, called Medichain that allows patients to
store their own data in a secure way and give anonymized access to this data to
specialists. Giving data allows for users to gain tokens that represent value.
\cite{MediChain2018}
