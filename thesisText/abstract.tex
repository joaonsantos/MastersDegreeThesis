\begin{tueABSTRACT}

  Bitcoin served as the catalyst for creating a solution to secure digital
  transactions without requiring a trusted third party to be involved. To solve
  this problem, the mechanisms now associated with a Blockchain were
  conceptualized and implemented to serve as the backbone for the Bitcoin
  network. More specifically, it was used as a security tool making Bitcoin a
  more transparent and reliable form of cash, a digital cryptographic currency.
  Even tough Bitcoin ended up not fulfilling its intended purpose as a
  currency, the Blockchain technology has enabled further avenues for
  innovation and creativity.

  Blockchain has since been used as the backbone for various cryptocurrencies
  networks. Some implementations of this technology allow the execution of
  code, also known as "smart contracts". Smart contracts are executed in an
  autonomous manner, with no human intervention. These can be used to solve a
  new set of problems due to their transparent behavior, lack of human
  intervention and distributed nature. 

  Blockchain technology allows the creation of systems that introduce a number
  of benefits over traditional data handling used in today's Healthcare
  Information Systems. Costs and risks associated with these systems can be
  reduced and information can become transparent and trustworthy to all
  participants.
  
  The Hyperledger Fabric Network with true private transactions and advanced
  security mechanisms was used to serve as the basis for the system proposed in
  this dissertation. Moreover, a client application was also created that
  interacts with smart contracts to manipulate the ledger.
  
  The work discussed in this dissertation shows that a Blockchain system based
  on Hyperledger Fabric is suitable for managing patients identity, in
  Healthcare. Even tough the feature set of this Blockchain is very focused in
  privacy and security, some additional measures regarding confidentiality of
  data had to be implemented.  Regardless, a system was built successfully that
  met the requirements. The implementation of this system would provide
  transparency, immutability and additional security for patients and medical
  staff alike. 

\end{tueABSTRACT}
