\begin{tueABSTRACT}

  Bitcoin served as the catalyst for creating a solution to secure digital
  transactions without requiring a trusted third party to be involved. To solve
  this problem the mechanisms now associated with a Blockchain were defined and
  implemented to serve as the backbone for the Bitcoin network. More
  specifically it was used as a security tool making Bitcoin a more tranparent
  and reliable form of cash in an online form. Even tough today, it is clear
  that it currently cannot fulfill its intended original purpose, it
  nonetheless, enable further innovation and creativity to solve both new sets
  of problems as well as old problems in a new way.

  The Blockchain was innitially used as the backbone for various
  cryptocurrencies networks and nowadays even provides a platform that allows the 
  execution of code in an autonomous manner exactly as it was programmed, with
  no human intervention. These smart contracts are used to solve yet another set of
  problems due to their transparency and distributed nature. 

  Blockchain technology allows the creation of systems that introduce a number of
  benefits over traditional methods used in today's Healthcare systems.  Costs
  and risks associated with these systems can be reduced and information can
  become transparent and trustworthy to all participants. In this article the
  technological foundations that enable this change are explored and analyzed.
  The Hyperledger Fabric Network with true private transactions and advanced
  security mechanisms was used to serve as the basis for this system.  An
  application was created that uses smart contracts to manipulate the ledger. In
  this paper we present this system and its impact in Healthcare.

\end{tueABSTRACT}
