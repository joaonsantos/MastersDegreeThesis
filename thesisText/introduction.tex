\chapter{Introduction}
\label{introduction}

\begin{quote} 
  \emph{This Chapter introduces the main topics and technologies covered by
  this dissertation. Healthcare and its relationship with technology is
  presented. The current flaws associated with patients identity data
  management are described. The Blockchain technology is introduced as a
  potential solution to some of these problems.} 
\end{quote}

The aim of this dissertation is to create a solution for managing the identity
of patients in the Healthcare environment by using Blockchain technology, and
in turn evaluate the use of this technology in this specific use case.  Health
is intrinsically linked with technology as new technologies enable safer and
better treatments. Healthcare organizations now store patients data on a
digital format. The Electronic Health Record (EHR) is an abstract concept
representing the patients digitally stored clinical data and their identity in
a medical and clinical context.

Standards are an important aspect to take into account when designing an
information system because they allow interoperability between different
organizations. The Health Level 7 (HL7) Fast Healthcare Interoperability
Resources (FHIR) standard (see Section~\ref{blockchainHealthcare}), is being
built primarily by the Health Level Seven organization. Over the last few years
it has seen a significant growth in usage and is an international standard with
partnerships worldwide. HL7 Portugal is now starting its operations and is
building a community to support the use of this standard in
Portugal~\cite{HealthLevel7}.

Blockchain is often known as the technology behind the Bitcoin cryptocurrency.
Bitcoin depends on two complementary technologies, digital tokens and a
Blockchain, that when orchestrated together facilitate trust, immutability and
resiliency~\cite{Evans2016}.

A Blockchain runs on a network of computers and has a list of records that are
replicated across the participating peers. Blockchain, as we know today, was
conceptualized as the public ledger~\footnote{A ledger is defined as an object
in which items are regularly recorded, originally business activities and money
received or paid, but in reality, it can be used to store any type of record.}
for the Bitcoin cryptocurrency in 2008 by Satoshi Nakamoto~\cite{Nakamoto2008},
a pen name of, a still unknown to this day, individual or organization of
individuals.

Traditional Healthcare databases and architectures are increasingly vulnerable
and a target to groups of malicious actors that possess the technical expertise
to deny services with Distributed Denial of Service (DDOS) attacks~\footnote{A
Distributed Denial of Service attack is an attempt to make an online service
unavailable by overwhelming it with traffic from multiple sources.} or cause a
data breach~\footnote{A data breach is the intentional or unintentional release
of secure or private/confidential information to an untrusted
environment.}~\cite{mcCoy2018}. 

Making matters worse other problems spring to mind. The data that comprises the
identity of a patient is often fragmented across multiple Healthcare
organizations, in such a way that, to get a true overview of the patients
history and diagnosis there would be a need to merge all the pieces of
information stored in data systems that are hosted in architecturally different
Healthcare information systems. Transparency is also a concern, as a patient
does not currently possess the means to track how his medical data is being
handled.

As more information becomes available, new insights can be extracted by
Healthcare professionals that lead to an overall improvement of the patients
interaction with the Healthcare ecosystem. However, maintaining a high amount
of data secure is a costly and risky matter for every party involved. Security
and privacy are a top concern regarding sensitive data. 

This thesis provides an insight into the design and implementation of a
Blockchain based system for managing the identity of patients in an Healthcare
setting and its subsequent evaluation. The creation of this system and its
subsequent evaluation could provide interesting conclusions to medical staff as
well as patients, regarding its potential implementation and deployment in the
field.

In this document different Blockchain implementations are explored to get an
overview of their feature set and focus. Considering a set of defined
requirements a platform is chosen, in order to evaluate the suitability of this
technology in the Healthcare field. More precisely, in
Chapter~\ref{background}, a brief introduction to Blockchain and its most
prominent implementations is presented. The technology is further explored in
Chapter~\ref{blockchain} and a number of real world use cases of this
technology in the Healthcare field are explored.  In Chapter~\ref{development}
a Blockchain platform is chosen in order to build a prototype system to
evaluate the usability of this technology in the Healthcare field. Insight is
given into the system design, implementation and evaluation. Finally, in
Chapter~\ref{Conclusion} some conclusions are presented and potential future
work is discussed.
