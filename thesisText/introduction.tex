\chapter{Introduction}\label{introduction}

\begin{quote} 
  \emph{The aim of this work is to build a Blockchain based system to manage
  the identity of patients and find out the suitability of creating such a
  system in the Healthcare environment.  Traditional databases alone can become
  vulnerable and a target to groups of malicious actors. Also the data that
  forms the identity of a patient is often fragmented across multiple
  Healthcare organizations, in such a way that, to a get an overview of the
  patients history and diagnosis you would need to merge together all the
  pieces of information stored in these data systems. Blockchain has a set of
  features that could improve upon these aspects if used together with existing
  systems. This thesis provides an insight into the design and implementation
  of a Blockchain based system for managing the identity of patients in a
  Healthcare context and its subsequent evaluation.  The system should allow a
  patient to be able to manage his data and control its access.  It can be used
  to handle the patient’s medical identity, for example, in hospitals or
  Healthcare clinics alongside the current information systems available in
  nowadays regular medical environments. The creation of this system and its
  subsequent evaluation will provide interesting conclusions to medical staff
  as well as patients, regarding its potential implementation and deployment in
  the field.} 
\end{quote}

Health is intrinsically linked with technology as new technologies enable safer
and better treatments. It is also worth noting that computing devices and
networks are now easily available and widespread to the general population in
more develop countries. Healthcare organizations now store patients data on a
digital format. The Electronic Health Record (EHR) is an abstract concept
representing the patients digitally stored clinical data and their identity in
a medical context.

Electronic Health Records  have lacked a standard for many years. Fortunately
there has been some progress regarding this matter recently as discussed
shortly.  Standards are an important aspect to take into account when designing
an information system because they allow interoperability between different
organizations with no additional work. The Health Level 7 (HL7) standard is
being built primarily by Health Level Seven organization. Over the last few
years it has seen a significant growth in usage and is an international
standard with partnerships worldwide. HL7 Portugal is starting its operations
and is building a community to support the use of this standard in
Portugal~\cite{HealthLevel7}.

Blockchain is often known as the technology behind the Bitcoin cryptocurrency.
Bitcoin depends on two complementary technologies, digital tokens and
Blockchain, that when orchestrated together facilitate trust, immutability and
resiliency~\cite{Evans2016}.

A Blockchain runs on a network of computers and has a list of records that are
replicated across the participating peers. Blockchain, as we know today, was
conceptualized as the public ledger~\footnote{A ledger is defined as an object
in which items are regularly recorded, originally business activities and money
received or paid, but in reality, it can be used to store any type of record.}
for the Bitcoin cryptocurrency in 2008 by Satoshi Nakamoto, a pen name of, a
still unknown to this day, individual or organization of individuals. The
network was implemented in 2009 and many are now finding it has a much broader
potential across many fields, with some implementations even resembling a
programming platform to execute code in an autonomous
manner~\cite{Nakamoto2008}.

Traditional databases and architectures can become vulnerable and a target to
groups of malicious actors that possess the technical expertise to deny
services with Distributed Denial of Service (DDOS) attacks~\footnote{A
Distributed Denial of Service attack is an attempt to make an online service
unavailable by overwhelming it with traffic from multiple sources.} or cause a
data breach~\footnote{A data breach is the intentional or unintentional release
of secure or private/confidential information to an untrusted environment.}. 

Making matters worse other problems spring to mind. The data that forms the
identity of a patient is often fragmented across multiple Healthcare
organizations, in such a way that, to get a true overview of the patients
history and diagnosis you would need to merge together all the pieces of
information stored in data systems that are hosted in architecturally different
and sometimes competing Healthcare information systems. Transparency is also a
concern, as a patient does not currently possess the means to track how his
medical data is being handled.

As more information becomes available, new insights can be extracted by
Healthcare professionals that lead to an overall improvement of the patients
interaction with the Healthcare ecosystem. However, maintaining a high amount
of data secure is a costly and risky matter for every party involved. Security
and privacy should be a top concern regarding this sensitive data. 

In this document different Blockchain implementations are explored to get an
overview of their feature set and focus in order to evaluate the suitability of
this technology in the Healthcare field. More precisely, in
Chapter~\ref{background}, a brief introduction to Blockchain and its most
prominent implementations. The technology is further explored in
Chapter~\ref{blockchain} and a number of real world use cases of this
technology in the Healthcare field are explored.  In Chapter~\ref{development}
a Blockchain platform is chosen in order to build a prototype system to
evaluate the usability of this technology in the Healthcare field. The system
is designed, implemented, built and evaluated. Finally, in
Chapter~\ref{Conclusion} some conclusions are presented and potential future
work is discussed.
