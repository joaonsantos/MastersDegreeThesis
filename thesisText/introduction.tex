\chapter{Introduction}\label{introduction}

\begin{quote} \emph{"This paper provides an insight into the design and
  implementation of a Blockchain based system for managing the identity of a
  patient in a Healthcare context. Such a system could be used as a complement
  to current Information Systems in order to provide a higher degree of
  resiliency and trust. The system should allow a patient to be able to manage
  his data and control its access. It can be used to handle the patient’s
  medical identity, for example, in hospitals or clinics and would help solve
the aforementioned problems in how data is handled in the Information Systems
available in nowadays regular medical environments."} \end{quote}

Health is intrinsically linked with technology as new technologies enable safer
and better treatments. It is also worth noting that computing devices and
networks are now easily available and widespread to the general population in
more develop countries. Healthcare organizations now store patients data on a
digital format. The Electronic Health Record (\textbf{ehr}) is an abstract
concept representing the patients digitally stored clinical data and their
identity in a medical context.

Electronic Health Records (\textbf{ehr}) for many years have lacked a standard,
even tough, more recently there has been some progress regarding this matter.
Standards are an important step to implement because they allow
interoperability between different organizations. The Health Level 7
(\textbf{hl7}) standard, being developed by Health Level Seven organization, is
growing in use and is represented internationally. Health Level Seven Portugal
is starting its operations and is building a community to support the use of
this standard in Portugal~\cite{HealthLevel7}.

Blockchain is often known as the technology behind the Bitcoin Cryptocurrency.
The phenomenon we know as Bitcoin depends on two complementary technologies,
digital tokens and Blockchain, that together facilitate trust, immutability and
resiliency~\cite{Evans2016}.

A Blockchain runs on a network of computers and has a list of records that are
replicated across the participating peers. Blockchain, as we know today, was
conceptualized as the public ledger~\footnote{A ledger is defined as a book in
which items are regularly recorded, especially business activities and money
received or paid.} for the Bitcoin cryptocurrency in 2008 by Satoshi Nakamoto,
a pen name of, a still unknown to this day, individual or organization of
individuals. The network was implemented in 2009 and many are now finding it
has a much broader potential across many fields, with some implementations even
resembling a programming platform to execute code in an autonomous
manner~\cite{Nakamoto2008}.

Traditional databases and architectures can become vulnerable and a target to
groups of malicious actors that possess the technical expertise to deny
services with Distributed Denial of Service (\textbf{ddos}) attacks~\footnote{A
Distributed Denial of Service (\textbf{ddos}) attack is an attempt to make an
online service unavailable by overwhelming it with traffic from multiple
sources.} or cause a data breach~\footnote{A data breach is the intentional or
unintentional release of secure or private/confidential information to an
untrusted environment.}. 

Making matters worse other problems spring to mind. The data that forms the
identity of a patient is often fragmented across multiple Healthcare
organizations, in such a way that, to get a true overview of the patients
history and diagnosis you would need to merge together all the pieces of
information stored in data systems that are hosted in architecturally different
and sometimes competing Healthcare Information Systems. Transparency is also a
concern, as a patient does not currently possess the means to track how his
medical data is being handled by the medical services he used.

As more information becomes available, new insights can be extracted by health
professionals that lead to an overall improvement of the patients interaction
with the Healthcare ecosystem. However, maintaining this huge amount of data
secure is a costly and risky matter for every party involved. Security and
privacy should be a top concern regarding this sensitive data. 

In this article different Blockchain implementations are explored to get an
overview of their capabilities and purposes. Also, some use cases of Blockchain
based systems used in the Healthcare field are presented. More precisely, in
Section~\ref{background}, a brief introduction to Blockchain is made followed
by an introduction to its most prominent implementations. Then a number of
real-world use cases of this technology in the healthcare field are explored.
In Section~\ref{HLFHealthcare} technical details of the system will be
presented. Finally, in Section~\ref{conclusion}, some conclusions are observed
regarding the change enabled by these advances.
