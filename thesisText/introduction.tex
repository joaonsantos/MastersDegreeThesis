%!TEX root = main.tex
\chapter{Introduction}

Health is becoming more digital thanks to the widespread availability of
computing devices.  More and more medical records are stored on a digital
format.  For storing patient clinical data and their identity in a medical
context, the Electronic Health Record (EHR) was created.
 
While all this information should benefit both patient and health professionals
alike, it is not being handled in an effective manner due to problems caused,
in part, due to the fragmentation of the patients identity that naturally
occurs in today's Health Information Systems.

Health is an important topic, for everyone. Healthcare should strive to provide
the best service it can for everyone and everyone should have access to a
quality service. EHR are being generated at an ever increasing rate but most of
the data is not used in a way that puts the patient's privacy and trust at the
forefront.

The purpose of the work presented in this paper is to create and implement a
Blockchain based system for Identity Management in the Healthcare domain. The
patient will be able to manage his data and control its access. Such a system
would be suited to handle the patient’s identity, for example, in hospitals or
clinics and would be able to solve many problems in how data is traditionally
handled in the Information Systems (IS) available in a regular medical
environment.

Blockchain is known as the technology behind the Bitcoin Cryptocurrency,
altough nowadays it is being used for many more purposes that are explored in
the following sections, and its main design goal is to provide security and
immutability to an agreed upon list of records.

A Blockchain runs on a network of computers and the list of records is
replicated in some manner depending on the Blockchain implementation. The first
Blockchain was conceptualized as the public ledger for the Bitcoin
cryptocurrency in 2008 by Satoshi Nakamoto, a pen name of, a still unknown to
this day, individual or organization of individuals.  The network was
implemented in 2009 and many are now finding it has a much broader potential
across many fields, with some implementations even resembling a programming
platform to execute code in an autonomous manner.  \cite{Nakamoto2008}

A single universal way to identify a person in a given environment is clearly
something we should strive towards as seen in, for example, the \textit{Cartão
do Cidadão}, a portuguese identification document that replaces four other
identification documents, streamlining portuguese civilian identification.
This also allows many businesses to tailor their services to this document
making it easier on both parts and eliminating unnecessary costs and risks.

Electronic Health Records (EHR) have seen some progress made regarding the
standards that allow for interoperability between different organizations
thanks to the Health Level 7 (HL7) standard.  While this standard is growing in
use and is represented internationally, Portugal has just started the work
required to implement it.  \cite{HealthLevel7}

In an effort to make the identity of a patient more secure and transparent a
Blockchain can be used to create a system that puts at the forefront of its
design the patients, breaking conventions in traditional patient data handling.

In this article different Blockchain implementations are explored and related
work in this field is presented.  More precisely, in Section~\ref{background},
a brief introduction to Blockchain is made followed by an introduction to its
most prominent implementations. Then a number of real-world use cases of this
technology in the healthcare field are explored. In Section~\ref{HLFHealthcare}
technical details of the system will be presented.  Finally, in
Section~\ref{conclusion},  some conclusions are observed regarding the change
enabled by these advances.
